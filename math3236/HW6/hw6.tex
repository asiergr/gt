\documentclass{article}
\usepackage[utf8]{inputenc}
\usepackage{amsmath}
\usepackage{amssymb}
\usepackage{amsfonts}
\usepackage{amsthm}
\usepackage{parskip}
\usepackage{bm}

\newcommand{\N}{\mathbb{N}}
\newcommand{\Z}{\mathbb{Z}}
\newcommand{\Q}{\mathbb{Q}}
\newcommand{\R}{\mathbb{R}}
\newcommand{\C}{\mathbb{C}}

\renewcommand{\P}[1]{\mathbb{P}\left(#1\right)}
\newcommand{\E}[1]{\mathbb{E}\left[#1\right]}
\newcommand{\normal}{\mathcal{N}}
\newcommand{\var}[1]{\text{var}\left[#1\right]}
\newcommand{\gammafn}[1]{\Gamma\left(#1\right)}
\newcommand{\randsamp}{X_1,\dots,X_n}
\newcommand{\mgf}{moment generating function }
\newcommand{\pdf}{p.d.f. }
\newcommand{\pmf}{p.m.f. }
\newcommand{\cdf}{c.d.f. }
\newcommand{\clt}{central limit theorem}
\newcommand{\mle}{M.L.E. }
\DeclareMathOperator*{\Binomial}{Binomial}

\newenvironment{hwproof}[1]
{
    #1
    \begin{proof}
}{
    \end{proof}
}

\title{HW5}
\author{Asier Garcia Ruiz}


\begin{document}
\maketitle
\section*{Problem 1. Section 9.7, problem 9.}
\begin{hwproof}
    {
        Consider again the conditions of Exercise 7, but suppose now that it is
        desired to test the following hypotheses:
        \begin{align*}
            H_0 & :\ \sigma_1^2 = \sigma_2^2,    \\
            H_1 & :\ \sigma_1^2 \neq \sigma_2^2, \\
        \end{align*}
        Suppose also that the statistic $V$ is defined by Eq. (9.7.4), and it is
        desired to reject $H_0$ if either $V \leq c_1$ or $V \geq c_2$, where
        the constants $c_1$ and $c_2$ are chosen so that when $H_0$ is true,
        $\P{V \leq c_1} = \P{V \geq c_2} = 0.025$. Determine the values of $c_1$
        and $c_2$ when $m=16$ and $n=10$, as in Exercise 7.
    }
\end{hwproof}

\section{Problem 2. Section 9.7, problem 15.}
\begin{hwproof}
    {
        Prove Theorem 9.7.5. Also, compute the p-value for Example 9.7.4 using the
        formula in Eq. (9.7.8).
    }
\end{hwproof}

\section{Problem 3. Section 10.1, problem 3.}
\begin{hwproof}
    {
        Investigate the “randomness” of your favorite pseudorandom number generator
        as follows. Simulate 200 pseudo-random numbers between 0 and 1 and divide
        the unit interval into $k = 10$ intervals of length 0.1 each. Apply the
        $\chi^2$ test of the hypothesis that each of the 10 intervals has the
        same probability of containing a pseudo-random number.
    }
\end{hwproof}

\section*{Problem 4. Section 10.1, problem 9.}
\begin{hwproof}
    {
        The 50 values in Table 10.5 are intended to be a random sample from the
        standard normal distribution.

        \textbf{a.}
        Carry out a $\chi^2$ test of goodness-of-fit by dividing the real line
        into five intervals, each of which has probability 0.2 under the standard
        normal distribution.
    }
\end{hwproof}

\begin{hwproof}
    {
        \textbf{b.}
        Carry out a $\chi^2$ test of goodness-of-fit by dividing the real line
        into 10 intervals, each of which has probability 0.1 under the standard
        normal distribution.
    }
\end{hwproof}

\section*{Problem 5. Section 10.2, problem 4.}
\begin{hwproof}
    {
        Consider again the sample consisting of the heights of 500 men given in
        Exercise 8 of Sec. 10.1. Suppose that before these heights were grouped
        into the intervals given in that exercise, it was found that for the
        500 observed heights in the original sample, the sample mean was
        $\bar{X}_n = 67.6$ and the sample variance was $S^n_2/n = 1.00$.
        Test the hypothesis that these observed heights form a random sample from
        a normal distribution.
    }
\end{hwproof}
\section*{Problem 6. Section 10.2, problem 6.}
\begin{hwproof}
    {
        Rutherford and Geiger (1910) counted the numbers of alpha particles
        emitted by a certain mass of polonium during 2608 disjoint time periods,
        each of which lasted 7.5 seconds. The results are given in Table 10.9.
        Test the hypothesis that these 2608 observations form a random
        sample from a Poisson distribution.
    }
\end{hwproof}
\section*{Problem 7. Section 10.3, problem 6.}
\begin{hwproof}
    {
        Suppose that a store carries two different brands, A and B, of a certain
        type of breakfast cereal. Suppose that during a one-week period the
        store noted whether each package of this type of cereal that was purchased
        was brand A or brand B and also noted whether the purchaser was a man or
        a woman. (A purchase made by a child or by a man and a woman together was
        not counted.) Suppose that 44 packages were purchased, and that the results
        were as shown in Table 10.19. Test the hypothesis
        that the brand purchased and the sex of the purchaser are independent.
    }
\end{hwproof}
\section*{Problem 8. Section 10.4, problem 4.}
\begin{hwproof}
    {
        Suppose that five persons shoot at a target. Suppose also that for
        $i=1,\dots,5$, person $i$ shoots $n_i$ times and hits the target $y_i$ times,
        and that the values of $n_i$ and $y_i$ are as given in Table 10.27. Test the
        hypothesis that the five persons are equally good marksmen.
    }
\end{hwproof}
\section*{Problem 9. Section 10.4, problem 6.}
\begin{hwproof}
    {
        Suppose that 100 students in a physical education class shoot at a target with a
        bow and arrow, and 27 students hit the target. These 100 students are then given
        a demonstration on the proper technique for shooting with the bow and arrow. After
        the demonstration, they again shoot at the target. This time 35 students hit
        the target. What additional information, if any, is needed in order to investigate
        the hypothesis that the demonstration was helpful?
    }
\end{hwproof}
\section*{Problem 10. Section 10.5, problem 6}
\begin{hwproof}
    {
        It was believed that a certain university was discriminating against women in
        its admissions policy because 30 percent of all the male applicants to the
        university were admitted, whereas only 20 percent of all the female applicants
        were admitted. In order to determine which of the five colleges in the
        university were most responsible for this discrimination, the admissions
        rates for each college were analyzed separately. Surprisingly, it was found
        that in each college the proportion of female applicants who were admitted
        to the college was actually larger than the proportion of male applicants who
        were admitted. Discuss and explain this result.
    }
\end{hwproof}
\end{document}