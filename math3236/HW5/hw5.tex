\documentclass{article}
\usepackage[utf8]{inputenc}
\usepackage{amsmath}
\usepackage{amssymb}
\usepackage{amsfonts}
\usepackage{amsthm}
\usepackage{parskip}
\usepackage{bm}

\newcommand{\N}{\mathbb{N}}
\newcommand{\Z}{\mathbb{Z}}
\newcommand{\Q}{\mathbb{Q}}
\newcommand{\R}{\mathbb{R}}
\newcommand{\C}{\mathbb{C}}

\renewcommand{\P}[1]{\mathbb{P}\left(#1\right)}
\newcommand{\E}[1]{\mathbb{E}\left[#1\right]}
\newcommand{\normal}{\mathcal{N}}
\newcommand{\var}[1]{\text{var}\left[#1\right]}
\newcommand{\gammafn}[1]{\Gamma\left(#1\right)}
\newcommand{\randsamp}{X_1,\dots,X_n}
\newcommand{\mgf}{moment generating function }
\newcommand{\pdf}{p.d.f. }
\newcommand{\pmf}{p.m.f. }
\newcommand{\cdf}{c.d.f. }
\newcommand{\clt}{central limit theorem}
\newcommand{\mle}{M.L.E. }
\DeclareMathOperator*{\Binomial}{Binomial}

\newenvironment{hwproof}[1]
{
    #1
    \begin{proof}
}{
    \end{proof}
}

\title{HW4}
\author{Asier Garcia Ruiz}

\begin{document}
\maketitle
I worked on this homework with Daniel Yankin and Armaan Lala.

\section{Section 9.2, problem 7.}
\begin{hwproof}
    {Suppose that $\randsamp$ form a random sample from the normal distribution
        with known mean $\mu$ and unknown variance $\sigma^2$, and the following
        simple hypotheses are to be tested:
        \begin{align*}
            H_0: & \ \sigma^2 = 2, \\
            H_1: & \ \sigma^2 = 3, \\
        \end{align*}

        \textbf{a.} Show that among all the test procedures for which $\alpha(\delta) \leq 0.05$,
        the value of $\beta(\delta)$ is minimised by a procedure that rejects $H_0$ when
        $\sum_{i=1}^n (X_i - \mu)^2 > c$.
    }
    Let $f_i(\bm{x})$ be the \pdf when $H_i$ is true. Then we have
    \begin{align*}
        f_0(\bm{x}) & = \frac{1}{(4\pi)^{n/2}}\exp\left(-\frac{1}{4}\sum_{i=1}^n(x_i-\mu)^2\right), \\
        f_1(\bm{x}) & = \frac{1}{(6\pi)^{n/2}}\exp\left(-\frac{1}{6}\sum_{i=1}^n(x_i-\mu)^2\right).
    \end{align*}
    Now we see that the likelihood ration $f_1(\bm{x}) / f_0(\bm{x})$ is
    \begin{equation*}
        \frac{f_1(\bm{x})}{f_0(\bm{x})} =
        \frac{2}{3}^{n/2} \exp\left(\frac{1}{12}\sum_{i=1}^n(x_i - \mu)^2\right).
    \end{equation*}
    and thus
    \begin{equation*}
        \log\frac{f_1(\bm{x})}{f_0(\bm{x})} =
        \frac{1}{12}\sum_{i=1}^n(x_i - \mu)^2 + \frac{n}{2} \log\frac{2}{3}.
    \end{equation*}
    By the Neymann-Pearson lemma, we reject $H_0$ if $\frac{f_1(\bm{x})}{f_0(\bm{x})} > c$.
    We let
    \begin{equation*}
        k = 12(c - \frac{n}{2} \log\frac{2}{3}).
    \end{equation*}
    Now we have that we will reject $H_0$ if the likelihood ratio is larger than $k$. So,
    we pick $c$ such that
    \begin{equation*}
        \P{\sum_{i=1}^n(X_i - \mu) > k | H_0} = 0.05.
    \end{equation*}

\end{hwproof}

\begin{hwproof}
    {
        \textbf{b.} For $n=8$, find the value of the constant $c$ that appears in part
        (a).
    }
    When $H_0$ is true then $Y = \sum_{i=1}^n(X_i - \mu) / 2$ has the $\chi^2(n)$ distribution.
    Thus, we have that
    \begin{equation*}
        \P{\sum_{i=1}^n(X_i - \mu) > k | H_0} = \P{Y > \frac{k}{2}}.
    \end{equation*}
    We have that $n=8$ and want $\P{Y > \frac{k}{2}} = 0.05$. Consulting the table at the back
    of the book we find that $k/2 = 15.51$ and thus $k = 31.02$.
\end{hwproof}

\section{Section 9.2, problem 12.}
\begin{hwproof}
    {
        Let $\randsamp$ be a random sample from the exponential distribution
        with unknown parameter $\theta$. Let $0 < \theta_0 < \theta_1$ be two
        possible values of the parameter. Suppose that we wish to test the
        following hypotheses:
        \begin{align*}
            H_0: & \ \theta = \theta_0, \\
            H_1: & \ \theta = \theta_1.
        \end{align*}
        For each $\alpha \in (0,1)$, show that among all tests $\delta$
        satisfying $\alpha(\delta) \leq \alpha_0$, the test with the
        smallest probability of type II error will reject $H_0$ if
        $\sum_{i=1}^n X_i < c$, where $c$ is the $\alpha_0$
        quantile of the gamma distribution with parameters
        $n$ and $\theta_0$
    }
    Let $f_i(\bm{x})$ be the \pdf when $H_i$ is true. Then we have
    \begin{align*}
        f_0(\bm{x}) & = \theta_0^n\exp\left(-\theta_0 \sum_{i=1}^n x_i\right), \\
        f_1(\bm{x}) & = \theta_1^n\exp\left(-\theta_1 \sum_{i=1}^n x_i\right), \\
    \end{align*}
    Now we see that the likelihood ration $f_1(\bm{x}) / f_0(\bm{x})$ is
    \begin{equation*}
        \frac{f_1(\bm{x})}{f_0(\bm{x})} =
        \frac{\theta_1}{\theta_0} \exp\left((\theta_0 - \theta_1) \sum_{i=1}^n x_i\right).
    \end{equation*}
    and thus
    \begin{equation*}
        \log\frac{f_1(\bm{x})}{f_0(\bm{x})} =
        (\theta_0 - \theta_1) \sum_{i=1}^n x_i + \log\frac{\theta_1}{\theta_0}.
    \end{equation*}
    By the Neymann-Pearson lemma, we reject $H_0$ if $\frac{f_1(\bm{x})}{f_0(\bm{x})} > c$.
    We let
    \begin{equation*}
        k = (\theta_0 - \theta_1)(c - \log\frac{\theta_1}{\theta_0}).
    \end{equation*}
    Now we have that we will reject $H_0$ if the likelihood ratio is larger than $k$. So,
    we pick $c$ such that
    \begin{equation*}
        \P{\sum_{i=1}^n X_i > k | H_0} = \alpha_0.
    \end{equation*}
    We can see that $\sum_{i=1}^n X_i$ is nothing but a Gamma distribution with parameters $n$
    and $\theta_0$ (property of the sum of exponential distributions). Therefore, $k$
    must be the $\alpha_0$ quantile of this distribution.

\end{hwproof}
\section{Section 9.3, problem 5.}
\begin{hwproof}
    {
        Suppose that $\randsamp$ form a random sample from a distribution that belongs to an
        exponential family, as defined in Exercise 23 of Sec. 7.3, and the \pdf or the
        p.f. of this distribution is $f(\bm{x}|\theta)$, as given in that exercise. Suppose also
        that $c(\theta)$ is a strictly increasing function of $\theta$. Show that the joint \pdf
        or the joint p.f. of $\randsamp$ has a montone likelihood ratio in the statistic
        $\sum_{i=1}^n d(X_i)$.
    }
    A distribution $f(x|\theta)$ belong to an exponential family if it can be written as
    $f(x|\theta) = a(\theta)b(x)\exp[c(\theta)d(x)]$ for $\theta \in \Omega$ and
    all values of $x$.

    Let $\theta_1, \theta_2 \in \Omega$ such that $\theta_1 < \theta_2$. Now
    consider the likelihood ratio
    \begin{align*}
        \frac{f_n(\bm{x}|\theta_2)}{f_n(\bm{x}|\theta_1)} & =
        \frac{a^n(\theta_2)b^n(\bm{x})\exp\left[\sum_{i=1}^n c(\theta_2)d(\bm{x}_i)\right]}{a^n(\theta_1)b^n(\bm{x})\exp\left[\sum_{i=1}^n c(\theta_1)d(\bm{x}_i)\right]}, \\
                                                          & =
        \left(\frac{a(\theta_2)}{a(\theta_1)}\right)\exp\left[(c(\theta_2) - c(\theta_1)) \sum_{i=1}^n d(\bm{x}_i)\right].
    \end{align*}
    Clearly we see that the likelihood ration depends only on $\bm{x}$ through the function
    $\sum_{i=1}^n d(X_i)$. Now, since $c(\theta)$ is strictly increasing in terms of $\theta$
    and $\theta_2 > \theta_1$, we have that $c(\theta_2) > c(\theta_1)$. Therefore,
    $c(\theta_2) - c(\theta_1) > 0$ and the ratio is an increasing monotone ratio of
    $\sum_{i=1}^n d(X_i)$. Thus, the \pdf or p.f. is has a monotone likelihood ratio in this
    statistic.
\end{hwproof}

\section{Section 9.3, problem 8.}
\begin{hwproof}
    {
        Suppose that $\randsamp$ form a random sample from the normal distribution with known
        mean $\theta$ and unknown variance $\sigma^2$, and suppose that it is desired to test
        the following hypotheses:
        \begin{align*}
            H_0 : & \ \sigma^2 \leq 2, \\
            H_1 : & \ \sigma^2 > 2.
        \end{align*}
        Show that there exists a UMP test of these hypotheses at every significance
        $\alpha_0 (0 < \alpha_0 < 1)$.
    }
    First we need to calculate the likelihood ratio. Consider $\sigma_1^2, \sigma_2^2 \in \Omega$
    such that $\sigma_2^2 > \sigma_1^2$. We let $y = \sum_{i=1}^n (x_i - \mu)^2$, now we can compute
    \begin{align*}
        \frac{f_n(\bm{x}|\sigma_2^2)}{f_n(\bm{x}|\sigma_1^2)} & =
        \frac{\frac{1}{(2\pi)^{n/2}\sigma_2^n}\exp(-\frac{y}{2\sigma_2^2})}{\frac{1}{(2\pi)^{n/2}\sigma_1^n}\exp(-\frac{y}{2\sigma_1^2})},                                            \\
                                                              & = \frac{\sigma_1^n}{\sigma_2^n}\exp\left(\frac{1}{2}\left(\frac{1}{\sigma_1^2} - \frac{1}{\sigma_2^2}\right)y\right).
    \end{align*}
    When $0 < \sigma_1^2 < \sigma_2^2$ this is clearly an increasing function of $y$.
    Thus, the joint \pdf has a monotone likelihood ratio in the statistic
    $y = (\sum_{i=1}^nx_i - \mu)^2$. Now, using Thm. 9.3.1 we know the test procedure
    that rejects $H_0$ if $y \geq c$, for some constant $c$, is a UMP test of this
    hypothesis at a significance level $\alpha_0$.
\end{hwproof}

\section{Section 9.5, problem 18.}
\begin{hwproof}
    {
        Suppose that $\randsamp$ form a random sample from a normal distribution with unknown
        mean $\mu$ and known variance 1. Suppose also that the following hypotheses are to be
        tested:
        \begin{align*}
            H_0 : & \ \mu \leq 0, \\
            H_1 : & \ \mu > 0.
        \end{align*}
        Let $\delta^*$ denote the UMP test of these hypotheses at the level of significance
        $\alpha_0 = 0.025$, and let $\pi(\mu | \delta^*)$ denote the power function of $\delta^*$.

        \textbf{a.} Determine the smallest value of the sample size $n$ for which
        $\pi(\mu | \delta^*) \geq 0.9$ for $\mu \geq 0.5$.
    }
    From Example 9.3.8 we know that $\delta^*$ is the UMP procedure that rejects
    $H_0$ when $\bar{X}_n \geq c$. Furthermore, we know the size of the test is
    $\alpha_0 = \P{\bar{X}_n \geq c | \mu = 0}$. Now, we must have
    \begin{equation*}
        \P{\bar{X}_n \geq c | \mu = 0} = \P{\sqrt{n}\bar{X}_n \geq \sqrt{n}c} = 0.025.
    \end{equation*}
    But, if $\mu = 0$ then $\sqrt{n}\bar{X}_n$ follows the standard normal distribution.
    Thus, $\P{\sqrt{n}\bar{X}_n \geq 1.96 | \mu = 0} = 0.025$. Then we clearly have
    that $\sqrt{n}c = 0.025$.

    Now, when $\mu = 0.5$ the random variable $Y = \sqrt{n}(\bar{X}_n - 0.5)$ is
    standard normal. So we have that
    \begin{align*}
        \pi(0.5 | \delta^*) & = \P{Y \geq 1.96 - 0.5n^{1/2}},                            \\
                            & =  \P{Y \leq 0.5n^{1/2} - 1.96} = \phi(0.5n^{1/2} - 1.96).
    \end{align*}
    Therefore, to get $\pi(0.5 | \delta^*) \geq 0.9$ we need
    $0.5n^{1/2} - 1.96 \geq 1.282$, or equivalently $n \geq 42.04$.
    Thus, we need $n=43$. Since the power function is increasing on $\mu$ then
    we also have that $\pi(0.5 | \delta^*) \geq 0.9$ for $\mu > 0.5$.
\end{hwproof}

\begin{hwproof}
    {
        \textbf{b.} Determine the smallest value of $n$ for which
        $\pi(\mu | \delta^*) \leq 0.001$ for $\mu \leq -0.1$.
    }
    Similarly, when $\mu = -0.1$ the random variable $Z = \sqrt{n}(\bar{X}_n + 0.1)$
    has standard normal distribution. Therefore
    \begin{equation*}
        \pi(-0.1 | \delta^*) = \P{Z \geq 1.96 + 0.1n^{1/2}}
        = 1 - \phi(1.96 + 0.1n^{1/2}).
    \end{equation*}
    We know that $\phi(3.10) = 0.999$. Therefore we need $1.96 + 0.1n^{1/2} \geq 3.10$
    or equivalently $n \geq 129.96$. Therefore we need a sample size of $n=130$ in
    order to have $\pi(-0.1 | \delta^*) \leq 0.001$. Since the power function is also
    strictly increasing in terms of $\mu$ it is also true that
    $\pi(-0.1 | \delta^*) \leq 0.001$ for $\mu < -0.1$.
\end{hwproof}

\section{Section 9.6, problem 2.}
\begin{hwproof}
    {
        Suppose that a certain drug A was administered to eight patients selected at random,
        and after a fixed time period, the concentration of the drug in certain body
        cells of each patient was measured in appropriate units. Suppose that these
        concentrations for the eight patients were found to be as follows:
        \begin{gather*}
            1.23, 1.42, 1.41, 1.62, 1.55, 1.51, 1.60, \text{ and } 1.76.
        \end{gather*}
        Suppose also that a second drug B was administered to six different patients
        selected at random, and when the concentration of drug B was measured in a
        similar way for these six patients, the results were as follows:
        \begin{gather*}
            1.76, 1.41, 1.87, 1.49, 1.67, \text{ and } 1.81.
        \end{gather*}
        Assuming that all the observations have a normal distribution with a common
        unknown variance, test the following hypotheses at the level of significance
        0.10: The null hypothesis is that the mean concentration of drug A among
        all patients is at least as large as the mean concentration of drug B.
        The alternative hypothesis is that the mean concentration of drug B is larger
        than that of drug A.
    }
    Let $\mu_1, \mu_2$ represent the means of samples A and B respectively. Then
    we are testing the hypothesis.
    \begin{align*}
        H_0: & \ \mu_1 \geq \mu_2, \\
        H_1: & \ \mu_1 < \mu_2.
    \end{align*}
    Now, using Thm. 9.6.1 we define a two-sample t statistic $U$ defined by
    Eq. 9.6.2 in the book. In these samples we have that for drug A the sample size
    is $m = 8$ so $\bar{x}_m = 1.51$ and $S_X^2 = 0.1808$. For drug B we have a
    sample size of $n=6$ so $\bar{y}_n = 1.67$ and $S_Y^2 = 0.1677$. The
    statistic $U$ has the t distribution with $m + n -2 = 12$ degrees of freedom
    when $\mu_1 = \mu_2$. Now, we will reject $H_0$ when $U \leq -T^{-1}(1 - \alpha_0)$.
    We calculate $U = -1.69$ and since $\alpha_0 = 0.10$ we get
    $-T_{12}^{-1}(0.9) = -1.356$. Therefore, we clearly reject $H_0$.
\end{hwproof}

\section{Section 9.6, problem 5.}
\begin{hwproof}
    {
        Consider again the conditions and observed values of Exercise 2. However,
        suppose now that each observation for drug A has an unknown variance
        $\sigma_1^2$, and each observation for drug B has an unknow variance
        $\sigma_2^2$, but it is known that $\sigma_2^2 = (6/5)\sigma_1^2$. Test the
        hypotheses described in Exercise 2 a the level of significance 0.10.
    }
    The variance of the distribution plays no role in the hypothesis test. Given that the
    statistic $U$ depends only on the sample variances and means, we have no need
    to change the procedure. Therefore, we have again that $U = -1.672$ and
    we should reject $U$ if it is less than -1.356. Therefore, we reject $H_0$.
\end{hwproof}

\section{Section 9.6, problem 10.}
\begin{hwproof}
    {
        Lyle et al. (1987) ran an experiment to study the effect of a calcium supplement
        on the blood pressure of African American males. A group of 10 men received a
        calcium supplement, and another group of 11 men received a placebo. The
        experiment lasted 12 weeks. Both before and after the 12-week period, each man had
        his systolic blood pressure measured while at rest. The changes (after minus
        before) are given in Table 9.2. Test the null hypothesis that the mean change
        in blood pressure for the calcium supplement group is lower than the mean change
        in blood pressure for the placebo group. Use level $\alpha_0 = 0.1$.
    }
    Let $\mu_1$ be the average blood pressure for the calcium supplement group
    and $\mu_2$ the average blood pressure for the placebo group.
    We have testing the hypothesis
    \begin{align*}
        H_0: & \ \mu_1 < \mu_2,    \\
        H_1: & \ \mu_1 \geq \mu_2.
    \end{align*}
    Now, using Thm. 9.6.1 we define a two-sample t statistic $U$ defined by
    Eq. 9.6.2 in the book. In these samples we have that for the calcium supplement group
    the sample size is $m = 10$ so $\bar{x}_m = 109.9$ and $S_X^2 = 546.9$.
    For the placebo group we have a sample size of $n=11$ so $\bar{y}_n = 113.9$ and
    $S_Y^2 = 1282.9$. The
    statistic $U$ has the t distribution with $m + n -2 = 19$ degrees of freedom
    when $\mu_1 = \mu_2$. Now, we will reject $H_0$ when $U > T^{-1}(1 - \alpha_0)$.
    We calculate $U = 0.9350$ and since $\alpha_0 = 0.10$ we get
    $T_{12}^{-1}(0.9) = 1.328$. Therefore, since $U < 1.328$ we do not reject $H_0$.
\end{hwproof}

\end{document}