\documentclass{article}
\usepackage[utf8]{inputenc}
\usepackage{amsmath}
\usepackage{amssymb}
\usepackage{amsfonts}
\usepackage{amsthm}
\usepackage{bm}
\usepackage{mathtools}
\usepackage{parskip}

\usepackage{listings}
\usepackage{graphicx}
\graphicspath{{./}}

\newcommand{\R}{\mathbb{R}}
\renewcommand{\O}[1]{\mathcal{O}(#1)}
\DeclareMathOperator{\sign}{sign}

\newenvironment{answer}{\textit{Answer:}}{}

\title{Midterm 2}
\author{Asier Garcia Ruiz }

\begin{document}
\maketitle

1. (1 point) Explain the connection between finding roots and solving nonlinear
systems.

\begin{answer}
    Solving a system of nonlinear equations
    \begin{equation*}
        \bm{f(x)} = \bm{g(x)}
    \end{equation*}
    is equivalent to solving
    \begin{equation*}
        \bm{h(x)} = 0,
    \end{equation*}
    where $\bm{h(x)} = \bm{f(x)} - \bm{g(x)}$.
    Which is nothing but finding the roots (points at which all equations in
    the system are zero) of $\bm{h(x)}$.
\end{answer}

2. (2 points) If $u(x) = f(x)/f'(x)$, where $f(x)$ has a root of multiplicity
$k$ at $x = \alpha$. Show that $u(\alpha) = 0$ but $u'(\alpha) \neq 0$.

\begin{answer}
    Clearly, if $f(\alpha) = 0$ then
    \begin{equation*}
        u(\alpha) = \frac{f(\alpha)}{f'\alpha} = \frac{0}{f'(\alpha)} = 0
    \end{equation*}
    provided that $f$ is differenitable and non-zero at $\alpha$. Now, we calculate $u'(x)$ as
    \begin{equation*}
        u'(x) = \left[\frac{f(x)}{f'(x)}\right]' =
        \frac{f'(x)^2 - f(x)f''(x)}{f'(x)^2}.
    \end{equation*}
    Evaluating at $x = \alpha$
    \begin{equation*}
        u'(\alpha) = \frac{f'(\alpha)^2 - f(\alpha)f''(\alpha)}{f'(\alpha)^2}
        = \frac{f'(\alpha)^2}{f'(\alpha)^2} = 1
    \end{equation*}
    provided $f'(\alpha)$ and $f''(\alpha)$ exist and is not zero.
\end{answer}

3. (2 points) A modified Newton method for roots $\alpha$ with multiplicity
$k > 1$ is
\begin{equation*}
    x_{n+1} = x_n - k \frac{f(x_n)}{f'(x_n)}.
\end{equation*}
Use both this method and regular Newton method to compute the root
$\alpha = 1$ of the function
\begin{equation*}
    f(x) = 1 - xe^{1-x}
\end{equation*}
Write a code for this. At what rate does the iteration converge for both the
modified and standard Newton method? What method is more accurate?

\begin{answer}
    The code listings are left in Appendix A. We must determine the multiplicity
    of $\alpha = 1$ for use in the imporoved Newton method. We first note that
    locally any function behaves like a polynomial. We also note that $f$ is
    sufficiently smooth, thus in a neighborhood of
    $\alpha = 1$ we can differentiate $f$. As with polynomials, the root has
    multiplicity $k$ where $f^{(k)}(\alpha) \neq 0$. We see that
    \begin{equation*}
        f''(x) = -(x-2)e^{1-x}
    \end{equation*}
    and $f''(1) = 1 \neq 0$. Hence, $k = 1$ for $\alpha = 2$.

    Given this, we now plot the graph for convergence to the root given
    randomised starting points within a neighborhood of $\alpha$
    for both methods.
\end{answer}

4. (1 point) Explain the connection between optimization and training neural
networks.

\begin{answer}
    When training a neural network we want to optimise the weights such that the
    prediction is closest to the real value. That is, we want to have a set of weights
    $w_i$ such that the output (given an input) is closest to some true value.
    This is an optimisation problem since we have to adjust these weights
    according to some rule to reduce the error value.
\end{answer}

5. (2 points) Let $x$ represent a distance the $t$ represent time. Consider the
data
\begin{center}
    \begin{tabular}{c c c c c c c}
        x & \vline  \ 0 & 0.25 & 0.50 & 0.75 & 1.00 & 1.25  \\
        t & \vline \ 0  & 25.0 & 49.4 & 73.0 & 96.4 & 119.4
    \end{tabular}
\end{center}
(a) (1 point) Find the cubic polynomial that interpoles this data at
$x = 0, 0.5, 0.75, 1.25$

\begin{answer}
    To do this we can simply solve the system
    \begin{equation*}
        \begin{bmatrix}
            1 & t_1 & t_1^2 & t_1^3, \\
            1 & t_2 & t_2^2 & t_2^3, \\
            1 & t_3 & t_3^2 & t_3^3, \\
            1 & t_4 & t_4^2 & t_4^3, \\
        \end{bmatrix}\bm{x} =
        \begin{bmatrix}
            0 \\ 0.5 \\ 0.75 \\ 1.25
        \end{bmatrix}
    \end{equation*}
    with $t_1 = 0, t_2 = 49.4, t_3 = 73.0, t_4 = 119.4$.

    Solving this system results in the coefficients
    $x_1 = 0, x_2 = 9.52*10^{-3}, x_3 = 1.5*10^{-5}, x_4=-5.95*10^{-8}$.
    for the polynomial $p(t) = x_1 + x_2 t + x_3t^2 + x_4t^4$.
\end{answer}

(b) (1 point) Use the polynomial to estimate the speed $\frac{dx}{dt}$ at
$x = 1.25$.

\begin{answer}
    First we find
    \begin{equation*}
        p'(t) = x_2 + x_3 t + x_4 t^2.
    \end{equation*}
    Then simply plugging in $t=119.4$, the point corresponding to $x=1.25$
    we get that the approximate speed is
    \begin{equation*}
        p'(119.4) = 1.04*10^{-3}.
    \end{equation*}
\end{answer}

6. (2 points) Construct a natural cubic spline that interpolates the gamma
function $\Gamma(x)$ with knots $x = 1.0, 1.2, 1.4, 1.6, 1.8, 2.0$.
Plot the actualy gamma function versus your spline on $x \in [1,2]$.
Compute the difference (error) between your spline and the actual gamma
function at $x = 1.1$.

\begin{answer}
    We have a series of of points $x_i$, $i = 0,...,5$. Hence, we have five
    intervals $[x_i, x_j], i < j$ and thus five polynomials
    \begin{equation*}
        p_i(x) = \alpha_{i1} + \alpha_{i2} t + \alpha_{i3} t^2 + \alpha_{i4}t^4,
        \quad i = 0,\dots,5.
    \end{equation*}


\end{answer}

7. (2 points) The improved trapezoidal rule has the form
\begin{equation*}
    T_n^c(f) = T_n(f) - \frac{1}{12}h^2(f'(b) - f'(a))
\end{equation*}
Apply the improved and standard trapezoidal rules ot approximate the integral
\begin{equation*}
    I = \int_1^2 e^{-x^2} \ dx = 0.1353572580.
\end{equation*}
with different numbers of sub-divisions (i.e different node spacings $h$).
Use at least 10 different $h$ values (including $h = 1/4$). Use these data to
estimate the rate of convergence of both the standard and improved methods.

\begin{answer}
    The code listing is left in Appendix B. Give
\end{answer}

Now, using the improved trapezoidal method

8. (1 point) Suggest a method for numerically approximating the derivative of
a function whose value is given only at a discrete set of data points.
For this problem, what would be the effect of noisy data, and how would you cope
with it in your numerical method?

\begin{answer}

\end{answer}

\section*{Appendix A}
\begin{lstlisting}[language=Python]
    
\end{lstlisting}

\end{document}