\documentclass{article}
\usepackage[utf8]{inputenc}
\usepackage{amsmath}
\usepackage{amssymb}
\usepackage{amsfonts}
\usepackage{amsthm}
\usepackage{bm}

\newcommand{\N}{\mathbb{N}}
\newcommand{\Z}{\mathbb{Z}}
\newcommand{\Q}{\mathbb{Q}}
\newcommand{\R}{\mathbb{R}}
\newcommand{\C}{\mathbb{C}}
\newcommand{\ra}{\xrightarrow{}}

\title{HW1}
\author{Asier Garcia Ruiz }
\date{August 2021}

\begin{document}
    \maketitle
    1. (1 point) If $0 < \epsilon <\epsilon_{mach}$ similar in size, explain why computing 
    $$z = x + \left(\frac{1}{\epsilon}\right)y$$ could be a problem for floating point arithmetic.\vspace{0.5cm}
    \textit{Answer}: Because $\epsilon < \epsilon_{mach}$ the computer can no longer guarantee
    the precision of $\epsilon$. When we divide by it, not only do we get a very large number
    but also a number that is imprecise, because again the precision cannot be guaranteed.

    \vspace{1cm}
    2. Two alternative series approximations for $e^{-x}$ are
    $$e^{-x} = 1 - x + \frac{1}{2}x^2 - \frac{1}{6}x^3 + ...$$ and 
    $$e^{-x} = \frac{1}{1 + x + x^2/2 = x^3/6 + ...}$$
    Of these two, which approach is more prone to finite-precision floating point rounding errors?
    Explain.
    
    \vspace{0.5cm}
    \textit{Answer:} The second approach is more prone to finite-precision floating point rounding
    errors. This is because in the second approach, we are dividing $1$ by a possibly very large
    number, meaning that the result $\epsilon$ could be such that $\epsilon < \epsilon_mach$. This
    implies a loss of precision on the value of $e^{-x}$.

    \vspace{1cm}
    3. Write  a  Gaussian  elimination  code  (no  pivoting)  that  solves  the  linear  system
    $$\bm{A}=\begin{pmatrix}
        1 & 1/2 &  1/3 \\
        1/2 & 1/3 & 1/4 \\
        1/3 & 1/4 & 1/5 \\
    \end{pmatrix}
    \qquad \bm{b} = \begin{pmatrix}
        7/6 \\ 5/6\\ 13/20
    \end{pmatrix}$$
    Include a listing of your code and your answerx.  What is the residual $r$.

    \vspace{0.5cm}
    4. (a) (1 point) Use Gaussian elimination (no pivoting) to solve the linear system $\bm{A}\bm{x}=\bm{b}$
    with $$\bm{A}=\begin{pmatrix}
        0.0001 & 1 \\
        1 & 1 \\
    \end{pmatrix} \qquad \bm{b} = \begin{pmatrix}
        1 \\ 2
    \end{pmatrix}$$
    with three digit decimal (base-10) arithmetic (by hand).  You can assume the machine chops.

    \vspace{0.5cm}
    \textit{Answer:} We will need to find two elimination matrices. The first one $\bm{M}_1 = 
    \begin{smallmatrix}
        
    \end{smallmatrix}
    
\end{document}