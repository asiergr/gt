\documentclass{article}
\usepackage[utf8]{inputenc}
\usepackage{amsmath}
\usepackage{amssymb}
\usepackage{amsfonts}
\usepackage{amsthm}
\usepackage{bm}
\usepackage{mathtools}

\title{HW1}
\author{Asier Garcia Ruiz }
\date{September 2021}

\begin{document}
    \maketitle
    1. (1 point) If $0 < \epsilon <\epsilon_{mach}$ similar in size, explain why computing 
    $$z = x + \left(\frac{1}{\epsilon}\right)y$$ could be a problem for floating point arithmetic.\vspace{0.5cm}
    \textit{Answer}: Because $\epsilon < \epsilon_{mach}$ the computer can no longer guarantee
    the precision of $\epsilon$. When we divide by it, not only do we get a very large number
    but also a number that is imprecise, because again the precision cannot be guaranteed.

    \vspace{1cm}
    2. Two alternative series approximations for $e^{-x}$ are
    $$e^{-x} = 1 - x + \frac{1}{2}x^2 - \frac{1}{6}x^3 + ...$$ and 
    $$e^{-x} = \frac{1}{1 + x + x^2/2 = x^3/6 + ...}$$
    Of these two, which approach is more prone to finite-precision floating point rounding errors?
    Explain.
    
    \vspace{0.5cm}
    \textit{Answer:} The second approach is more prone to finite-precision floating point rounding
    errors. This is because in the second approach, we are dividing $1$ by a possibly very large
    number, meaning that the result $\epsilon$ could be such that $\epsilon < \epsilon_mach$. This
    implies a loss of precision on the value of $e^{-x}$.

    \vspace{1cm}
    3. Write  a  Gaussian  elimination  code  (no  pivoting)  that  solves  the  linear  system
    $$\bm{A}=\begin{pmatrix}
        1 & 1/2 &  1/3 \\
        1/2 & 1/3 & 1/4 \\
        1/3 & 1/4 & 1/5 \\
    \end{pmatrix}
    \qquad \bm{b} = \begin{pmatrix}
        7/6 \\ 5/6\\ 13/20
    \end{pmatrix}$$
    Include a listing of your code and your answerx.  What is the residual $r$.

    \vspace{0.5cm}
    4. (a) (1 point) Use Gaussian elimination (no pivoting) to solve the linear system $\bm{A}\bm{x}=\bm{b}$
    with $$\bm{A}=\begin{pmatrix}
        0.0001 & 1 \\
        1 & 1 \\
    \end{pmatrix} \qquad \bm{b} = \begin{pmatrix}
        1 \\ 2
    \end{pmatrix}$$
    with three digit decimal (base-10) arithmetic (by hand).  You can assume the machine chops.

    \vspace{0.5cm}
    \textit{Answer:} Because our machine approximates to three decimal digits, matrix $\bm{A}$
    is interpreted as $\bm{A} = \begin{psmallmatrix}
        0 & 1 \\
        1 & 1 \\
    \end{psmallmatrix}$
    This means that one of pivots is 0 for the Gaussian elimination. Since pivotting is not
    allowed, we cannot proceed.

    \vspace{0.5cm}
    (b) Repeat with partial pivoting. Comment on your results.

    \vspace{0.5cm}
    Now we have partial pivoting, so our matrix gets its rows swapped, resulting in
    the system $$\begin{pmatrix}
        1 & 1 \\
        0.0001 & 1 \\
    \end{pmatrix} \bm{x} = \begin{pmatrix}
        2 \\ 1 \\
    \end{pmatrix}$$

    \vspace{1cm}
    5. Use a computer to determine the condition number of $\bm{A}$ from problem 4 
    (you can use full precision, e.g. double precision).  Explain how you coded it up 
    (I don’t want tosee the code itself).
    
    \vspace{0.5cm}
    \textit{Answer:} \\
    (a) The condition number for $\bm{A}$ is $\approx 2.6184$.
    because this is a $2\times 2$ matrix the inverse $\bm{A}^{-1}$ can be trivially
    calculated in constant time. Hence, the condition number calculated by
    $||\bm{A}||*||\bm{A}^{-1}||$. To find these norms, because they are subordinate
    to the vector norm, we simply generate a constant amount of random vectors and 
    pick the maximal $\frac{||\bm{A}\bm{x}||}{||\bm{x}||}$. \\
    (b) (b) (1 point) Do you consider this a good or bad condition number?
    Explain your reasoning.

    \vspace{0.5cm}
    I would consider this a good condition number. This is because it is only $\approx 2.5$
    times larger than the condition number for the identity matrix. Considering that the
    upper bound for this value is infinity (i.e it is unbounded), this is a pretty good value.

    \vspace{1cm}
    (c) f the condition number is bad, how could you change $\bm{A}$ to make it better? If the 
    condition number is good, how could you change $\bm{A}$ to make it worse (not that one 
    would want to do this)?

    \vspace{0.5cm}
    \textit{Answer:} To make this value worse we could make $a_{11} = 1$ or make 
    it very close to $1$. Because we know singular matrices have $\infty$ as their condition
    number, this would raise it. 
\end{document}