\documentclass[twoside]{article}

% Some basic packages
\usepackage[utf8]{inputenc}
\usepackage[T1]{fontenc}
\usepackage{textcomp}
\usepackage{url}
\usepackage{graphicx}
\usepackage{float}
\usepackage{booktabs}
\usepackage{enumitem}

% Don't indent paragraphs.
\usepackage{parskip}

% Hide page number when page is empty
\usepackage{emptypage}
\usepackage{subcaption}
\usepackage{multicol}
\usepackage{xcolor}

% Math stuff
\usepackage{amsmath, amsfonts, mathtools, amsthm, amssymb}
% Non-ams math stuff
\usepackage{derivative, physics}
% Fancy script capitals
\usepackage{mathrsfs}
\usepackage{cancel}

% Bold math
\usepackage{bm}

% Cool command shortcuts
\newcommand\N{\ensuremath{\mathbb{N}}}
\newcommand\R{\ensuremath{\mathbb{R}}}
\newcommand\Z{\ensuremath{\mathbb{Z}}}
\renewcommand\O{\ensuremath{\emptyset}}
\newcommand\Q{\ensuremath{\mathbb{Q}}}
\newcommand\C{\ensuremath{\mathbb{C}}}

% Operators
\DeclareMathOperator*{\argmax}{arg\,max}
\DeclareMathOperator*{\argmin}{arg\,min}

% Easily typeset systems of equations (French package)
\usepackage{systeme}

%Make implies and impliedby shorter
\let\implies\Rightarrow
\let\impliedby\Leftarrow
\let\iff\Leftrightarrow
\let\epsilon\varepsilon
\let\phi\varphi

% Add \contra symbol to denote contradiction
\usepackage{stmaryrd} % for \lightning
\newcommand\contra{\scalebox{1.5}{$\lightning$}}

% horizontal rule
\newcommand\hr{
    \noindent\rule[0.5ex]{\linewidth}{0.5pt}
}

% For box around Definition, Theorem, \ldots
\usepackage{mdframed}
\mdfsetup{skipabove=1em,skipbelow=0em}
\theoremstyle{definition}
\newmdtheoremenv[nobreak=true]{definition}{Definition}
\newmdtheoremenv[nobreak=true]{lemma}{Lemma}
\newmdtheoremenv[nobreak=true]{proposition}{Proposition}
\newmdtheoremenv[nobreak=true]{theorem}{Theorem}
\newmdtheoremenv[nobreak=true]{corollary}{Corollary}
\newmdtheoremenv{conjecture}{Conjecture}
\newtheorem*{remark}{Remark}
\newtheorem*{problem}{Problem}
\newtheorem*{eg}{Example}
\newtheorem*{question}{Question}
\newtheorem*{intuition}{Intuition}
\newtheorem*{claim}{Claim}

% End example environments with a small diamond (just like proof
% environments end with a small square)
\usepackage{etoolbox}
\AtEndEnvironment{eg}{\null\hfill$\diamond$}%

% Fix some spacing
% http://tex.stackexchange.com/questions/22119/how-can-i-change-the-spacing-before-theorems-with-amsthm
\makeatletter
\def\thm@space@setup{%
  \thm@preskip=\parskip \thm@postskip=0pt
}

% Exercise 
% Usage:
% \oefening{5}
% \suboefening{1}
% \suboefening{2}
% \suboefening{3}
% gives
% Oefening 5
%   Oefening 5.1
%   Oefening 5.2
%   Oefening 5.3
\newcommand{\exercise}[1]{%
    \def\@exercise{#1}%
    \subsection*{Exercise #1}
}

\newcommand{\subexercise}[1]{%
    \subsubsection*{Exercise \@exercise.#1}
}

% \lecture starts a new lecture (les in dutch)
%
% Usage:
% \lecture{1}{di 12 feb 2019 16:00}{Inleiding}
%
% This adds a section heading with the number / title of the lecture and a
% margin paragraph with the date.

% I use \dateparts here to hide the year (2019). This way, I can easily parse
% the date of each lecture unambiguously while still having a human-friendly
% short format printed to the pdf.

\usepackage{xifthen}
\def\testdateparts#1{\dateparts#1\relax}
\def\dateparts#1 #2 #3 #4 #5\relax{
    \marginpar{\small\textsf{\mbox{#1 #2 #3 #5}}}
}

\def\@lecture{}%
\newcommand{\lecture}[3]{
    \ifthenelse{\isempty{#3}}{%
        \def\@lecture{Lecture #1}%
    }{%
        \def\@lecture{Lecture #1: #3}%
    }%
    \subsection*{\@lecture}
    \marginpar{\small\textsf{\mbox{#2}}}
}


% For page size and geometry
\usepackage{geometry}

% These are the fancy headers
\usepackage{fancyhdr}
\pagestyle{fancy}

% LE: left even
% RO: right odd
% CE, CO: center even, center odd
% My name for when I print my lecture notes to use for an open book exam.
% \fancyhead[LE,RO]{Gilles Castel}

\fancyhead[RO,LE]{\@lecture} % Right odd,  Left even
\fancyhead[RE,LO]{}          % Right even, Left odd

\fancyfoot[RO,LE]{\thepage}  % Right odd,  Left even
\fancyfoot[RE,LO]{}          % Right even, Left odd
\fancyfoot[C]{\leftmark}     % Center

\makeatother

% Todonotes and inline notes in fancy boxes
\usepackage{todonotes}
\usepackage{tcolorbox}

% Fix some stuff
% %http://tex.stackexchange.com/questions/76273/multiple-pdfs-with-page-group-included-in-a-single-page-warning
\pdfsuppresswarningpagegroup=1


% name
\author{Asier García Ruiz}


\begin{document}
\section{Collaborators}
I collaborated on this with Ayush Panda.

\section{}
\exercise{1}
The matrix needed is
\begin{equation*}
	A =
	\begin{bmatrix}
		w_{22} & 0 & 0      & 0 & 0 & 0      & 0 & 0      \\
		0      & 0 & w_{20} & 0 & 0 & 0      & 0 & 0      \\
		0      & 0 & 0      & 0 & 0 & w_{02} & 0 & 0      \\
		0      & 0 & 0      & 0 & 0 & 0      & 0 & w_{00} \\
	\end{bmatrix}
\end{equation*}

\exercise{2}
The affine transformation $A$ can be written as
\begin{equation*}
	A =
	\begin{bmatrix}
		w_{00} & 0      & 0      & 0      \\
		w_{01} & 0      & 0      & 0      \\
		0      & w_{00} & 0      & 0      \\
		0      & w_{01} & 0      & 0      \\
		w_{10} & 0      & 0      & 0      \\
		w_{11} & 0      & 0      & 0      \\
		0      & w_{10} & 0      & 0      \\
		0      & w_{11} & 0      & 0      \\
		0      & 0      & w_{00} & 0      \\
		0      & 0      & w_{01} & 0      \\
		0      & 0      & 0      & w_{00} \\
		0      & 0      & 0      & w_{01} \\
		0      & 0      & w_{10} & 0      \\
		0      & 0      & w_{11} & 0      \\
		0      & 0      & 0      & w_{10} \\
		0      & 0      & 0      & w_{11} \\
	\end{bmatrix}
\end{equation*}

\exercise{3}
The affine transformation for a convolutional layer with kernel size
$(4, 1, 1, 1)$ is 
\begin{equation*}
    A =
    \begin{bmatrix}
        w_{00} &0&0&0\\ 
        0&w_{10}&0&0\\ 
        0&0& w_{10}&0\\ 
        0&0&0&w_{11}\\
        w_{00} &0&0&0\\ 
        0&w_{10}&0&0\\ 
        0&0& w_{10}&0\\ 
        0&0&0&w_{11}\\
        w_{00} &0&0&0\\ 
        0&w_{10}&0&0\\ 
        0&0& w_{10}&0\\ 
        0&0&0&w_{11}\\
        w_{00} &0&0&0\\ 
        0&w_{10}&0&0\\ 
        0&0& w_{10}&0\\ 
        0&0&0&w_{11}\\
    \end{bmatrix}.
\end{equation*}
The affine transformation for the convolution with kernel size $(1, 1, 2, 2)$
is the same as the one on Exercise 2. It is pretty evident that there 
transformations are equivalent as you can get one by simply swapping rows 
of the other. The only that changes is the order of the elements when 
flattened in row major.

\section{Logic and XOR}
\exercise{4}
We have that $\bm{w} = [w_{1}, w_{2}]^{T} \in \R^{2}, b \in \R$ and $\bm{x} \in \{0,1\}^{2}$. We also define
\begin{equation*}
	f(\bm{x}) =
	\begin{cases}
		1 \text{ if } \bm{w}^{T}\bm{x} + b \geq 0,\\
		0  \text{ if } \bm{w}^{T}\bm{x} + b < 0.
	\end{cases}
\end{equation*}

To find $\bm{w}_{AND}, b_{AND}$ (which we will denote as $\bm{w}, b$ for simplicity)
we require that
\begin{align*}
	\bm{w}^{T} [0, 0] + b < 0 \     & \implies \ b < 0,                    \\
	\bm{w}^{T} [0, 1] + b < 0 \     & \implies \ w_{2} + b < 0,            \\
	\bm{w}^{T} [1, 0] + b < 0 \     & \implies \ w_{1} + b < 0,            \\
	\bm{w}^{T} [1, 1] + b \geq 0 \  & \implies \ w_{1} + w_{2} + b \geq 0.
\end{align*}
Combining these, we get that
\begin{equation*}
	w_{1} + w_{2} + 3b < 0 \quad \text{and} \quad w_{1} + w_{2} + b \geq 0.
\end{equation*}
There are an infinite amount of choices that satisfy this.
One such example is $w_{1} = w_{2} = 1, b = -2$.

To find $\bm{w}_{OR}, b_{OR}$ (which we will denote as $\bm{w}, b$ for simplicity)
we require that
\begin{align*}
	\bm{w}^{T} [0, 0] + b < 0 \     & \implies \ b < 0,                    \\
	\bm{w}^{T} [0, 1] + b \geq 0 \  & \implies \ w_{2} + b \geq 0,         \\
	\bm{w}^{T} [1, 0] + b \geq 0 \  & \implies \ w_{1} + b \geq 0,         \\
	\bm{w}^{T} [1, 1] + b \geq 0 \  & \implies \ w_{1} + w_{2} + b \geq 0.
\end{align*}
Combining these, we get that
\begin{equation*}
	b < 0 \quad \text{and} \quad 2w_{1} + 2w_{2} + 3b \geq 0.
\end{equation*}
There are an infinite amount of choices that satisfy this.
One such example is $w_{1} = w_{2} = 2, b = -1$.

\exercise{5}
To find $\bm{w}_{XOR}, b_{XOR}$ (which we will denote as $\bm{w}, b$ for simplicity)
we require that
\begin{align*}
	\bm{w}^{T} [0, 0] + b < 0 \     & \implies \ b < 0,                 \\
	\bm{w}^{T} [0, 1] + b \geq 0 \  & \implies \ w_{2} + b \geq 0,      \\
	\bm{w}^{T} [1, 0] + b \geq 0 \  & \implies \ w_{1} + b \geq 0,      \\
	\bm{w}^{T} [1, 1] + b \geq 0 \  & \implies \ w_{1} + w_{2} + b < 0.
\end{align*}
Combining these, we get that
\begin{equation*}
	w_{1} + w_{2} + 2b < 0 \quad \text{and} \quad w_{1} + w_{2} + 2b \geq 0.
\end{equation*}
Which is clearly impossible. Therefore XOR cannot be represented using a linear model of this form.

\section{Piece-wise Linearity}
\exercise{6}
We have the function
\begin{equation*}
	h(x) = W^{(3)}\max \{0, W^{(2)} \max \{0, W^{(1)}x + b^{(1)}\} + b^{(2)}\} + b^{(3)}
\end{equation*}
with the given weights. We can actually unfold these piece wise components as such
\begin{align*}
	h(x)
	 & = W^{(3)}\max \{0, W^{(2)} \max \{0, W^{(1)}x + b^{(1)}\} + b^{(2)}\} + b^{(3)},           \\
	 & =  W^{(3)}\max \{0, W^{(2)} \max \{0, [1.5x, 0.5x + 1]^{T}\} + b^{(2)}\} + b^{(3)},        \\
	 & =
	\begin{cases}
		W^{(3)} \max \{0, W^{2}[0, 0]^{T} + b^{(2)}\} + b^{(3)}           & \text{if } x \leq -2,     \\
		W^{(3)} \max \{0, W^{2}[0, 0.5x + 1]^{T} + b^{(2)}\} + b^{(3)}    & \text{if } -2 < x \leq 0, \\
		W^{(3)} \max \{0, W^{2}[1.5x, 0.5x + 1]^{T} + b^{(2)}\} + b^{(3)} & \text{if } x > 0,
	\end{cases} \\
	 & =
	\begin{cases}
		W^{(3)} \max \{0, [0, 1]^{T}\} + b^{(3)}               & \text{if } x \leq -2,     \\
		W^{(3)} \max \{0, [x + 2, 0.5x + 1]^{T}\} + b^{(3)}    & \text{if } -2 < x \leq 0, \\
		W^{(3)} \max \{0, [2.5x + 2, 3.5x + 1]^{T}\} + b^{(3)} & \text{if } x > 0,
	\end{cases}            \\
	 & =
	\begin{cases}
		W^{(3)}[0, 1]^{T} + b^{(3)}                & \text{if } x \leq -2,     \\
		W^{(3)} [x + 2, 0.5x + 1]^{T} + b^{(3)}    & \text{if } -2 < x \leq 0, \\
		W^{(3)} [2.5x + 2, 3.5x + 1]^{T} + b^{(3)} & \text{if } x > 0,
	\end{cases}                        \\
	 & =
	\begin{cases}
		0        & \text{if } x \leq -2,     \\
		1.5x + 3 & \text{if } -2 < x \leq 0, \\
		6x + 3   & \text{if } x > 0.
	\end{cases}
\end{align*}

Therefore,
\begin{equation*}
	h^{\prime}(x) =
	\begin{cases}
		0   & \text{if } x \leq -2,     \\
		1.5 & \text{if } -2 < x \leq 0, \\
		6   & \text{if } x > 0.
	\end{cases}
\end{equation*}
Having found this, we can calculate $W, b$ for the different points.

Let $x = 2$, then $h^{\prime}(2) = 6 = W$ and thus $b = 9$ since $h(2) = 15 = W(2) + b$.

Let $x = -1$, then $h^{\prime}(-1) = 1.5 = W$ and thus $b = 0$ since $h(-1) = 1.5 = W(-1) + b$.

Let $x = 1$, then $h^{\prime}(1) = 6 = W$ and thus $b = 3$ since $h(1) = 9 = W(1) + b$.

\newpage
\section{Depth - Composing Linear Pieces}
\exercise{7}
\begin{proof}
	We will prove that there are $2^{d}$ regions identified by $f_{1}(\cdot)$ by induction.

	Base case: $d = 1$. Let $x \in \R$, then
	\begin{equation*}
		f_{1}(x) = \abs{2x - 1}.
	\end{equation*}
	This identifies the linear regions $(0, \frac{1}{2}), (\frac{1}{2}, 1)$ onto $(0, 1)$.
	This is $2^{1}$ regions, so our base case is met.

	Now, assume that for $x \in \R^{d - 1}$ the function $f_{1}$ identifies $2^{d - 1}$ regions.
	Namely, the regions $((0, \frac{1}{2}), (\frac{1}{2}, 1))^{d - 1}$.
	We now consider $x \in \R^{d}$ and note that $f_{1_{d}} = \abs{2x_{d} - 1}$.
	Therefore, we now identify the regions
	$((0, \frac{1}{2}), (\frac{1}{2}, 1))^{d - 1} \times ((0, \frac{1}{2}), (\frac{1}{2}, 1))
		= ((0, \frac{1}{2}), (\frac{1}{2}, 1))^{d}$.

	Intuitively, we can imagine this as "folding" all the regions in $\R^{d}$ (of which there
	are $2^{d}$) into a subset of the first orthant.
\end{proof}

\exercise{8}
\begin{proof}
	Let $g$ identify $n_{g}$ regions of $(0, 1)^{d}$ onto $(0, 1)^{d}$ and
	$f$ identify $n_{f}$ regions of $(0, 1)^{d}$ onto $(0, 1)^{d}$.
	The total numver of separate input out space neighborhoods that are mapped onto
	a commot neighborhood is given recursively by
	\begin{equation*}
		N_{R}^{l} = \sum_{R^{\prime} \in \mathcal{R}^{l}} N_{R_{\prime}}^{l - 1}
	\end{equation*}
	where $l$ is the layer. Specifially, we have a two layer case so we can easily
	resolve this recurrence as
	\begin{equation*}
		N_{R}^{2} = \sum_{R^{\prime} \in \mathcal{R}^{2}} n_{g} = n_{f}n_{g}.
	\end{equation*}
\end{proof}

\exercise{9}
\begin{proof}
	We will prove this by inducting on $L$.

	Base case: $k = 1$. We have already shows that $f_{1} = h_{1}$ identifies $2^{d}$ regions,
	so we are done.

	We now assume that $h_{k - 1}$ identifies $2^{(k - 1)d}$ regions of $\bm{x}$.
	Now, take $h_{k} = \abs{W_{k}\bm{h}_{k - 1} + \bm{b}_{k}}$. As we showed in Ex. 8, for any
	two regions $h_{k - 1}, h_{k}$ that identify $2^{(k - 1)d}, 2^{d}$ regions of their
	input-output maps respectively, the total regions mapped is
	$2^{(k - 1)d}2^{d} = 2^{kd}$ as needed.

	Therefore, $f(\bm{x})$ identifies $2^{Ld}$ regions of its input.
\end{proof}

\newpage
\section{Implicit Regularizaton of Gradient descent}
\exercise{10}
\begin{proof}
	For this proof, we will assume that $w^{(0)} \in \Span \{\bm{X}\}$.
	We have that $f(\bm{w}) = \norm{\bm{y} - \bm{X}\bm{w}}$, therefore, taking the gradient
	we get $\grad_{\bm{w}}f(\bm{w}) = 2\bm{X}^{T}(\bm{X} \bm{w} - \bm{y})$. We further note that
	since the updates are of the form $\bm{w}^{(t + 1)} = \bm{w}^{(t)} - \eta \grad f(\bm{w}^{(t)})$
	and $\bm{w}^{(t)} \in \Span \{\bm{X}\}$ for all $t$, then all the updates are also
	in the span of $\bm{X}$. Therefore, we can write
	$w^{(t)} = \bm{w}^{(t - 1)} - \bm{X}\bm{u}$ for some $\bm{u}$.

	Now, since $w^{(0)} \in \Span \{\bm{X}\}$ we have that $x^{(0)} = \bm{X}^{T}\bm{v}$
	for some vector $\bm{v}$. Therefore, $w^{(t + 1)} = X^{T}(u + v)$.
	Now, we know that gradient descent will converge to
	$\bm{w}_{gd} = X^{T}\bm{u}_{gd}$ at a point where $\grad_{\bm{w}}f = 0$.

	Putting all this together, we get that
	\begin{gather*}
		2\bm{X}^{T}(\bm{Xw}_{gd} - \bm{y}) = 0,\\
		2\bm{X}^{T}(\bm{XX}^{T}\bm{u}_{gd} - \bm{y}) = 0,\\
		\bm{u}_{gd} = (\bm{XX}^{T})^{-1}\bm{y},\\
		(\bm{X}^{T})^{-1} = (\bm{XX}^{T})^{-1}\bm{y},\\
		\bm{w}_{gd} = \bm{X}^{T}(\bm{XX}^{T})^{-1}\bm{y},
	\end{gather*}
	which is a well known minimum norm solution to the problem $Xw = y$
	(i.e., the solution to a linear regression by least squares).
\end{proof}

\exercise{11}
The optimiztion problem in (24) and (25) is expressing a solution of
the system $\bm{Xw} = \bm{y}$ such that the l-2 norm of $\bm{w}$ is the
smallest it can be. Basically, it is projecting the vector $\bm{y}$ onto 
the span of $\bm{X}$ and finding the solution to that, this 
solution will have the minimum norm, which can also be interpreted as 
regularising the weights.

\exercise{12}
This tells us that gradient descent prefers solutions of minimum norm.
Clearly if the surface is not convex then the solution is not unique,
and gradient descent prefers the solution with minimum norm.

\newpage
\section{Receptive Fields}
\exercise{13}
\subexercise{a}
When working with deep convolutional neural networks we are sometimes 
interested in what affects specific output pixels/decisions. This is determined 
by the combination of the receptive fields that end up affecting that output pixel.
Therefore, it is interesting to know which values in which layers affect the 
subsequent values.

\subexercise{b}
Evidently we have that $r_{2} = 1$.
Now, using the recursion we get 
\begin{align*}
    r_{1} &= s_{2}r_{2} + (k_{2} - s_{2}) = 3,\\ 
    r_{0} &= s_{1}r_{2} + (k_{1} - s_{1}) = 19.
\end{align*}

\subexercise{c}
To calculate the receptive fields we use the recursive formula.
To calculate the size of the feature map we use equation (26).
Combinind the results we get
\begin{align*}
    f_{0} = 259,\ & r_{0} = 163,\\
    f_{1} = 63,\ & r_{1} = 39,\\
    f_{2} = 31,\ & r_{2} = 19,\\
    f_{3} = 31,\ & r_{3} = 15,\\
    f_{4} = 15,\ & r_{4} = 7,\\
    f_{5} = 15,\ & r_{5} = 5,\\
    f_{6} = 15,\ & r_{6} = 3,\\
    f_{7} = 15,\  & r_{7} = 1,\\
\end{align*}


\end{document}
