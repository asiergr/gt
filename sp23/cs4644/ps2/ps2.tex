\documentclass[twoside]{article}

% Some basic packages
\usepackage[utf8]{inputenc}
\usepackage[T1]{fontenc}
\usepackage{textcomp}
\usepackage{url}
\usepackage{graphicx}
\usepackage{float}
\usepackage{booktabs}
\usepackage{enumitem}

% Don't indent paragraphs.
\usepackage{parskip}

% Hide page number when page is empty
\usepackage{emptypage}
\usepackage{subcaption}
\usepackage{multicol}
\usepackage{xcolor}

% Math stuff
\usepackage{amsmath, amsfonts, mathtools, amsthm, amssymb}
% Non-ams math stuff
\usepackage{derivative, physics}
% Fancy script capitals
\usepackage{mathrsfs}
\usepackage{cancel}

% Bold math
\usepackage{bm}

% Cool command shortcuts
\newcommand\N{\ensuremath{\mathbb{N}}}
\newcommand\R{\ensuremath{\mathbb{R}}}
\newcommand\Z{\ensuremath{\mathbb{Z}}}
\renewcommand\O{\ensuremath{\emptyset}}
\newcommand\Q{\ensuremath{\mathbb{Q}}}
\newcommand\C{\ensuremath{\mathbb{C}}}

% Operators
\DeclareMathOperator*{\argmax}{arg\,max}
\DeclareMathOperator*{\argmin}{arg\,min}

% Easily typeset systems of equations (French package)
\usepackage{systeme}

%Make implies and impliedby shorter
\let\implies\Rightarrow
\let\impliedby\Leftarrow
\let\iff\Leftrightarrow
\let\epsilon\varepsilon
\let\phi\varphi

% Add \contra symbol to denote contradiction
\usepackage{stmaryrd} % for \lightning
\newcommand\contra{\scalebox{1.5}{$\lightning$}}

% horizontal rule
\newcommand\hr{
    \noindent\rule[0.5ex]{\linewidth}{0.5pt}
}

% For box around Definition, Theorem, \ldots
\usepackage{mdframed}
\mdfsetup{skipabove=1em,skipbelow=0em}
\theoremstyle{definition}
\newmdtheoremenv[nobreak=true]{definition}{Definition}
\newmdtheoremenv[nobreak=true]{lemma}{Lemma}
\newmdtheoremenv[nobreak=true]{proposition}{Proposition}
\newmdtheoremenv[nobreak=true]{theorem}{Theorem}
\newmdtheoremenv[nobreak=true]{corollary}{Corollary}
\newmdtheoremenv{conjecture}{Conjecture}
\newtheorem*{remark}{Remark}
\newtheorem*{problem}{Problem}
\newtheorem*{eg}{Example}
\newtheorem*{question}{Question}
\newtheorem*{intuition}{Intuition}
\newtheorem*{claim}{Claim}

% End example environments with a small diamond (just like proof
% environments end with a small square)
\usepackage{etoolbox}
\AtEndEnvironment{eg}{\null\hfill$\diamond$}%

% Fix some spacing
% http://tex.stackexchange.com/questions/22119/how-can-i-change-the-spacing-before-theorems-with-amsthm
\makeatletter
\def\thm@space@setup{%
  \thm@preskip=\parskip \thm@postskip=0pt
}

% Exercise 
% Usage:
% \oefening{5}
% \suboefening{1}
% \suboefening{2}
% \suboefening{3}
% gives
% Oefening 5
%   Oefening 5.1
%   Oefening 5.2
%   Oefening 5.3
\newcommand{\exercise}[1]{%
    \def\@exercise{#1}%
    \subsection*{Exercise #1}
}

\newcommand{\subexercise}[1]{%
    \subsubsection*{Exercise \@exercise.#1}
}

% \lecture starts a new lecture (les in dutch)
%
% Usage:
% \lecture{1}{di 12 feb 2019 16:00}{Inleiding}
%
% This adds a section heading with the number / title of the lecture and a
% margin paragraph with the date.

% I use \dateparts here to hide the year (2019). This way, I can easily parse
% the date of each lecture unambiguously while still having a human-friendly
% short format printed to the pdf.

\usepackage{xifthen}
\def\testdateparts#1{\dateparts#1\relax}
\def\dateparts#1 #2 #3 #4 #5\relax{
    \marginpar{\small\textsf{\mbox{#1 #2 #3 #5}}}
}

\def\@lecture{}%
\newcommand{\lecture}[3]{
    \ifthenelse{\isempty{#3}}{%
        \def\@lecture{Lecture #1}%
    }{%
        \def\@lecture{Lecture #1: #3}%
    }%
    \subsection*{\@lecture}
    \marginpar{\small\textsf{\mbox{#2}}}
}


% For page size and geometry
\usepackage{geometry}

% These are the fancy headers
\usepackage{fancyhdr}
\pagestyle{fancy}

% LE: left even
% RO: right odd
% CE, CO: center even, center odd
% My name for when I print my lecture notes to use for an open book exam.
% \fancyhead[LE,RO]{Gilles Castel}

\fancyhead[RO,LE]{\@lecture} % Right odd,  Left even
\fancyhead[RE,LO]{}          % Right even, Left odd

\fancyfoot[RO,LE]{\thepage}  % Right odd,  Left even
\fancyfoot[RE,LO]{}          % Right even, Left odd
\fancyfoot[C]{\leftmark}     % Center

\makeatother

% Todonotes and inline notes in fancy boxes
\usepackage{todonotes}
\usepackage{tcolorbox}

% Fix some stuff
% %http://tex.stackexchange.com/questions/76273/multiple-pdfs-with-page-group-included-in-a-single-page-warning
\pdfsuppresswarningpagegroup=1


% name
\author{Asier García Ruiz}

\title{Deep Learning Problem Set 2}
\begin{document}
\maketitle

\exercise{1}
I collaborated with Ayush Panda on this homework.

\exercise{2}
\subexercise{1}
\begin{proof}
	We are to find the solution $\bm{w}^{*}$ to the following problem
	\begin{equation*}
		\argmin_{\bm{w}} f(\bm{w}^{(t)}) + \langle \bm{w} - \bm{w}^{(t)}, \grad f(\bm{w}^{(t)})
		+ \frac{\lambda}{2} \lVert \bm{w} - \bm{w}^{(t)} \rVert^{2}.
	\end{equation*}
	Since there are no constraints and $f$ is convex, we can simply define the function
	\begin{equation*}
		\mathcal{L}(\bm{w}) = f(\bm{w}^{(t)}) + \langle \bm{w} - \bm{w}^{(t)}, \grad f(\bm{w}^{(t)})
        + \frac{\lambda}{2} \lVert \bm{w} - \bm{w}^{(t)} \rVert^{2}
	\end{equation*}
    and find the minimum by taking the gradient. 

	\begin{claim}
		We claim that
		\begin{equation*}
			\grad \lVert \bm{x} \rVert^{2} = 2\bm{x}.
		\end{equation*}
		\begin{proof}
			From the definiton of the norm
			\begin{equation*}
				f(x) = \lVert x \rVert^{2} = \sum_{j = 1}^{n} x_{j}^{2},
			\end{equation*}
			then
			\begin{equation*}
				\pdv{f}{x_{j}} = \pdv{}{x_{j}} \sum_{j = 1}^{n} x_{j}^{2} = 2x_{j}.
			\end{equation*}
			and the claim follows.
		\end{proof}
	\end{claim}

	Having proven this claim, we proceed to find $\grad \mathcal{L}(\bm{w})$. We write,
	\begin{align*}
		\grad_{\bm{w}} \mathcal{L}(\bm{W}) & = \grad_{\bm{w}}
		\left[ f(\bm{w}^{(t)}) + \langle \bm{w} - \bm{w}^{(t)}, \grad f(\bm{w}^{(t)})\langle
		+ \frac{\lambda}{2} \lVert \bm{w} - \bm{w}^{(t)} \rVert^{2} \right],                               \\
		                                   & = 0 + \grad f(\bm{w}^{(t)}) + \lambda(\bm{w} - \bm{w}^{(t)}). \\
	\end{align*}
	Now, setting this equal to zero, we get
	\begin{equation*}
		\bm{w} = \bm{w}^{(t)} - \frac{1}{\lambda}\grad f(\bm{w}^{(t)}).
	\end{equation*}

	This tells us that the gradient descent method is the solution to this $\argmin$ problem. Meaning that
	assuming that $f$ is convex (or at least locally convex) it will always take us to $\bm{w}^{*}$
	that minimises this loss function, should we choose to use it.

	Clearly we can see that $\eta = \frac{1}{\lambda}$.
\end{proof}

\subexercise{2}
\begin{proof}
	We will prove the inequality
	\begin{equation*}
		\sum_{t=1}^{T} \langle \bm{w}^{(t)} - \bm{w}^{*}, \bm{v}_{t}
		\leq \frac{\lVert \bm{w}^{*} \rVert^{2}}{2 \eta}
		+ \frac{\eta}{2}\sum_{t=1}^{T}\lVert \bm{v}_{t} \rVert^{2}.
	\end{equation*}

	We begin by noting that
	\begin{align*}
		\lVert \bm{w}^{(t+1)} - \bm{w}^{(*)} \rVert^{2}
		 & = \lVert \bm{w}^{(t)} - \eta \bm{v}_{t} - \bm{w}^{*} \rVert^{2},                                     \\
		 & = \lVert \bm{w}^{(t)} - \bm{w}^{*} \rVert^{2}
		+ \eta^{2} \lVert \bm{v}_{t} \rVert^{2} - 2 \eta \langle \bm{w}^{(t)} - \bm{w}^{*}, \bm{v}_{t} \rangle. \\
	\end{align*}
	Which, after rearranging gives us
	\begin{equation*}
		\langle \bm{w}^{(t)} - \bm{w}^{*}, \bm{v}_{t} \rangle
		= \frac{\lVert \bm{w}^{(t)} - \bm{w}^{*} \rVert^{2}}{2 \eta}
		+ \frac{\eta \lVert \bm{v}_{t} \rVert^{2}}{2}
		-  \frac{\lVert \bm{w}^{(t + 1)} - \bm{w}^{*} \rVert^{2}}{2 \eta}.
	\end{equation*}
	Now, we sum over all $t$, and observe that the sum is telescoping. Giving us that
	\begin{equation*}
		\sum_{t = 1}^{T} \langle \bm{w}^{(t)} - \bm{w}^{*}, \bm{v}_{t} \rangle
		= \frac{\lVert \bm{w}^{(1)} - \bm{w}^{*} \rVert^{2}}{2 \eta}
		+ \frac{\eta \lVert \bm{v}_{t} \rVert^{2}}{2}
		-  \frac{\lVert \bm{w}^{(T + 1)} - \bm{w}^{*} \rVert^{2}}{2 \eta}.
	\end{equation*}
	We now see that the last term in the equation is negative, and note that $\bm{w}^{1} = 0$.
	This leads to the final result that
	\begin{equation*}
		\sum_{t=1}^{T} \langle \bm{w}^{(t)} - \bm{w}^{*}, \bm{v}_{t}\rangle
		\leq \frac{\lVert \bm{w}^{*} \rVert^{2}}{2 \eta}
		+ \frac{\eta}{2}\sum_{t=1}^{T}\lVert \bm{v}_{t} \rVert^{2},
	\end{equation*}
	as required.

\end{proof}

\subexercise{3}
\begin{proof}
	We start by noting that since $f$ is convex, we have that
	\begin{equation*}
		f(y) - f(x) \leq \grad f(y)\cdot(y - x),
	\end{equation*}
	where $\cdot$ denotes the inner product.

	Therefore we immediately get that
	\begin{align*}
		f(\bar{\bm{w}}) - f(\bm{w}^{*}) & \leq \grad f\left(\frac{1}{T}\sum_{t=1}^{T} \bm{w}^{t}\right)
		\cdot \left(\frac{1}{T}\sum_{t=1}^{T} [\bm{w}^{T}] - \bm{w}^{*}\right).
		\intertext{Now, note that $\frac{1}{T}\sum_{t=1}^{T} [\bm{w}^{(t)}] - \bm{w}^{*}
			= \frac{1}{T}\sum_{t=1}^{T} [\bm{w}^{(t)}- \bm{w}^{*}]$ and use the fact that
		$\grad $ is a linear operator to get that}
		                                & = \left[\frac{1}{T}\sum_{t=1}^{T} \grad f(\bm{w}^{(t)})\right]
		\cdot \left[\frac{1}{T}\sum_{t=1}^{T} \bm{w}^{(t)} - \bm{w}^{*}\right],                          \\
		                                & = \frac{1}{T}\sum_{t=1}^{T}(\bm{w}^{(t)} - \bm{w}^{*})
		\cdot \grad f(\bm{w}^{(t)}),
	\end{align*}
	as required.
\end{proof}

\begin{proof}
	Now, that we have established these two inequalities, we can use them together
	to find an upper bound on the convergence of gradient descent.

	From both inequalities we have that
	\begin{equation*}
		f(\bar{\bm{w}}) - f(\bm{w}^{*}) \leq
		\frac{1}{T}\sum_{t=1}^{T}\langle \bm{w}^{(t)} - \bm{w}^{*}, \grad f(\bm{w}^{(t)})\rangle
		\leq \frac{\lVert \bm{w}^{*} \rVert^{2}}{2 \eta T}
		+ \frac{\eta}{2T}\sum_{t=1}^{T}\lVert \grad f(\bm{w}^{t}) \rVert^{2}.
	\end{equation*}
	Now, we use the fact that $f$ is $\rho$-Lipschitz and $B$ is an upper bound
	for $\lVert \bm{w}^{*} \rVert$ to get that
	\begin{align*}
		\frac{\lVert \bm{w}^{*} \rVert^{2}}{2 \eta T}
		+ \frac{\eta}{2T}\sum_{t=1}^{T}\lVert \grad f(\bm{w}^{t}) \rVert^{2}
		 & \leq \frac{B^{2}}{2 \eta T} + \frac{\eta}{2T} \sum_{t=1}^{T} \rho^{2}, \\
		 & = \frac{B^{2}}{2 \eta T} + \frac{\eta \rho^{2}}{2}.
		\intertext{Now, we let $\eta = \frac{B}{\rho \sqrt{T}}$,}
		 & = \frac{B \rho}{2\sqrt{T}} + \frac{B \rho}{2\sqrt{T}}
		= \frac{B \rho}{\sqrt{T}} = \mathcal{O}\left(\frac{1}{\sqrt{T}}\right),
	\end{align*}
	as required.

\end{proof}

\exercise{3}
\subexercise{a}
\begin{proof}
	We are to find the solution to
	\begin{equation*}
		\argmin_{\bm{w}} \left[f(\bm{w}^{(t)}) + \langle \bm{w} - \bm{w}^{(t)}, \grad f(\bm{w}^{(t)}) \rangle
			+ \frac{1}{2}(\bm{w} - \bm{w}^{(t)})^{T}\bm{H}(\bm{w} - \bm{w}^{(t)})\right].
	\end{equation*}
	Similarly to 2.1, this is an unconstrained problem, so we can simply define the function
	\begin{equation*}
		\mathcal{L}(\bm{w}) = f(\bm{w}^{(t)}) + \langle \bm{w} - \bm{w}^{(t)}, \grad f(\bm{w}^{(t)}) \rangle
		+ \frac{1}{2}(\bm{w} - \bm{w}^{(t)})^{T}\bm{H}(\bm{w} - \bm{w}^{(t)}),
	\end{equation*}
	take the gradient, and set it equal to zero.

	\begin{claim}
		We claim that
		\begin{equation*}
			\grad \bm{w}^{T}\bm{A}\bm{w} = 2\bm{A}\bm{w}
		\end{equation*}
		for any symmetric $\bm{A}$.

		\begin{proof}
			Let
			\begin{equation}
				f(x) = \bm{w}^{T}\bm{A}\bm{w} = \sum_{i = 1}^{n} \sum_{j = 1}^{n} w_{i}a_{ij}w_{j}
				= \sum_{i = 1}^{n} (a_{ii}w_{i}^{2} + \sum_{j \neq i}w_{i}a_{ij}w_{j}).
			\end{equation}
			Now, we can take the partial derivative with respect to the $k$th element.
			\begin{align*}
				\pdv{}{w_{k}} \left[\sum_{i = 1}^{n} (a_{ii}w_{i}^{2} + \sum_{j \neq i}w_{i}a_{ij}w_{j})\right]
				 & = 2a_{kk}w_{k} + \sum_{j \neq k}w_{j}a_{jk} + \sum_{j \neq k} a_{kj}w_{j}.
				\intertext{now, since $\bm{A}$ is symmetric (i.e. $a_{ij} = a_{ji} \ \forall i, j$),}
				 & = 2a_{kk}w_{k} + \sum_{j \neq k} 2w_{j}a_{jk},                             \\
				 & = \sum_{i = 1}^{n} 2w_{k}a_{ki}.
			\end{align*}
			Concluding the proof of the claim.
		\end{proof}
	\end{claim}

	Now, with the claim proven, we proceed to find the gradient of $\mathcal{L}(\bm{w})$.
	\begin{align*}
		\grad \mathcal{L}(x)
		 & = 0 + \grad f(\bm{w}^{(t)}) + \bm{H}\bm{w}.
	\end{align*}
	Now, setting this equal to zero, we get that
	\begin{equation*}
		\bm{w} = \bm{H}^{-1}\grad f(\bm{w}^{(t)}).
	\end{equation*}
\end{proof}

\subexercise{b}
There are $2(50*50) + 500 = 5500$ parameters. Therefore $H \in \R^{5500 \times 5500}$.

\subexercise{c}
Using Eqn 9 from the problem worksheet and letting $\Delta \bm{w} = r \bm{v}$, rearranging, we get
\begin{align*}
	\bm{Hv} & =
	\frac{1}{r} [\grad_{\bm{w}} (\bm{w} + r \bm{v}) - \grad_{\bm{w}}(\bm{w}) + \mathcal{O}(\lVert r \bm{v} \rVert^{2})], \\
	        & = \frac{1}{r} [\grad_{\bm{w}} (\bm{w} + r \bm{v}) - \grad_{\bm{w}}(\bm{w})] + \mathcal{O}(r),              \\
\end{align*}

\subexercise{e}
\begin{proof}
	We prove two properties of the $\mathcal{R}$ operator.

	\begin{align*}
		\mathcal{R}_{\bm{v}}\{cf(\bm{w})\}
		 & = \pdv{}{r} cf(\bm{w} + r \bm{v}) |_{r = 0},  \\
		 & = c \pdv{}{r} f(\bm{w} + r \bm{v}) |_{r = 0}, \\
		 & = c\mathcal{R}_{\bm{v}}\{f(\bm{w})\}.
	\end{align*}
	as needed.

	We also have that
	\begin{align*}
		\mathcal{R}_{\bm{v}}\{f(\bm{w}g(\bm{w}))\}
		 & = \pdv{}{r} f(\bm{w} + r \bm{v})g(\bm{w} + r \bm{v}) |_{r = 0}, \\
		 & = f(\bm{w} + r \bm{v})\pdv{}{r}g(\bm{w} + r \bm{v}) |_{r = 0} +
		g(\bm{w} + r \bm{v})\pdv{}{r} f(\bm{w} + r \bm{v}) |_{r = 0},      \\
		 & = f(\bm{w})\pdv{}{r}g(\bm{w} + r \bm{v}) |_{r = 0} +
		g(\bm{w})\pdv{}{r} f(\bm{w} + r \bm{v}) |_{r = 0},                 \\
		 & = \mathcal{R}_{\bm{v}}\{f(\bm{w})\}g(\bm{w})
		+ \mathcal{R}_{\bm{v}}\{g(\bm{w})\}f(\bm{w}),                      \\
	\end{align*}
	as needed.

	This concludes our proof.
\end{proof}

\subexercise{e}
\begin{proof}
	Using Eqn 9 again, letting $\Delta \bm{w} = r \bm{v}$,
	and using the limit definition of the derivative, we get
	\begin{align*}
		\bm{Hv}
		 & = \lim_{r \to 0} \frac{1}{r} [\grad_{\bm{w}} (\bm{w} + r \bm{v}) - \grad_{\bm{w}}(\bm{w})], \\
		 & = \pdv{}{r} \grad_{\bm{w}}(\bm{w} + r \bm{v}) |_{r = 0},                                    \\
		 & = \mathcal{R}_{\bm{v}}\{\grad_{\bm{w}}(\bm{w})\}.
	\end{align*}
\end{proof}

\subexercise{f}
With an auto-differentiation library we can find $\grad_{\bm{w}}$ since this is nothing but the derivative of the loss
with respect to the weights. Now, instead of calculating the Hessian, we can simply use the $\mathcal{R}$ operator
on the gradient to find the Hessian. This makes the process of finding the hessian much less expensive.

\exercise{4}
\subexercise{a}


\exercise{5}
\subexercise{a}

\begin{proof}
	To show that $C_{a}S = SC_{a}$ we can simply multiply them. Simply multiplying the matrices we get
	\begin{equation*}
		C_{a}S =
		\begin{bmatrix}
			a_{0}     & a_{n - 1} & \cdots & a_{1}  \\
			a_{1}     & a_{0}     & \cdots & a_{2}  \\
			\vdots    & \vdots    & \ddots & \vdots \\
			a_{n - 1} & a_{n - 2} & \cdots & a_{0}
		\end{bmatrix}
		\begin{bmatrix}
			0 & 0      & \cdots & 1      \\
			1 & 0      & \cdots & 0      \\
			0 & \ddots & \ddots & \vdots \\
			0 & \cdots & 1      & 0
		\end{bmatrix}
		=
		\begin{bmatrix}
			a_{n - 1} & a_{n - 2} & \cdots & a_{0}     \\
			a_{1}     & a_{0}     & \cdots & a_{1}     \\
			\vdots    & \vdots    & \ddots & \vdots    \\
			a_{n - 2} & a_{n - 3} & \cdots & a_{n - 1}
		\end{bmatrix}
		= SC_{a}
	\end{equation*}
\end{proof}

\subexercise{b}
\begin{proof}
	Let $f$ be a linear operator. We want to show that $f(Sx \bm{x}) = Sf(\bm{x}) \iff f(x) = (x * w)_{i} = C_{a}\bm{x}$.

	($\Leftarrow$)\\
	We assume that $f(\bm{x}) = C_{a} \bm{x}$, as we proved in (a) we know that $C_{a}S = SC_{a}$.
	Therefore, we hc ave that
	\begin{equation*}
		f(S \bm{x}) = C_{a}S \bm{x} = SC_{a}\bm{x} = Sf(\bm{x}).
	\end{equation*}
	Thus, $f$ is shift-equivariant.

	($\Rightarrow$)\\
	We assume that $f$ is some shift-equivariant linear operator.
	This implies that $f(S \bm{x}) = AS \bm{x} = SA \bm{x} = Sf(x)$.

	Let $A$ be any matrix, then we have that
	\begin{equation*}
		AS =
		\begin{bmatrix}
			a_{11} & a_{12} & \cdots        & a_{1n} \\
			a_{21} & a_{22} & \cdots        & a_{2n} \\
			\vdots & \vdots & \ddots        & \vdots \\
			a_{n1} & a_{n2} & \cdots a_{nn}
		\end{bmatrix}
		\begin{bmatrix}
			0 & 0      & \cdots & 1      \\
			1 & 0      & \cdots & 0      \\
			0 & \ddots & \ddots & \vdots \\
			0 & \cdots & 1      & 0
		\end{bmatrix}
		=
		\begin{bmatrix}
			a_{12} & a_{13} & \cdots & a_{1n}       & a_{11} \\
			a_{22} & a_{23} & \cdots & a_{n, n - 1} & a_{21} \\
			\vdots & \vdots & \ddots & \vdots       & \vdots \\
			a_{n2} & a_{3n} & \cdots & a_{nn}       & a_{n1}
		\end{bmatrix}
	\end{equation*}
	Now, by our assumption, we know this must be equal to
	\begin{align*}
		SA
		 & =
		\begin{bmatrix}
			0 & 0      & \cdots & 1      \\
			1 & 0      & \cdots & 0      \\
			0 & \ddots & \ddots & \vdots \\
			0 & \cdots & 1      & 0
		\end{bmatrix}
		\begin{bmatrix}
			a_{11} & a_{12} & \cdots        & a_{1n} \\
			a_{21} & a_{22} & \cdots        & a_{2n} \\
			\vdots & \vdots & \ddots        & \vdots \\
			a_{n1} & a_{n2} & \cdots a_{nn}
		\end{bmatrix}, \\
		 & =
		\begin{bmatrix}
			a_{n1}       & a_{n2}       & \cdots & a_{n, n-1}     & a_{nn}       \\
			a_{11}       & a_{12}       & \cdots & a_{1, n - 1}   & a_{1n}       \\
			\vdots       & \vdots       & \ddots & \vdots         & \vdots       \\
			a_{n - 1, 1} & a_{n - 1, 2} & \cdots & a_{n - 1, n-1} & a_{n - 1, n}
		\end{bmatrix}.
	\end{align*}
	We can see now that this implies that $a_{11} = a_{22} = \dots = a_{nn}$,
	$a_{12} = a_{n1} = \dots$. We now see that this is nothing
	but the circulant matrix.

	Therefore, we have that the circulant matrix is the only linear operator that
	is shift-equivariant.
\end{proof}

\subexercise{c}
Shift-equivariance is very powerful in terms of processing spatial or spatio-temporal data.
This is because it implies that the data can be shifted in any direction, and the result
will be the same as if it were shifted \textit{after} the convolution.

This means, in short, that the convolution kernel is not dependent on spatial location,
which makes it generalise easily.
\exercise{6}

\end{document}
