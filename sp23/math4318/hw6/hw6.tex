\documentclass[twoside]{article}

% Some basic packages
\usepackage[utf8]{inputenc}
\usepackage[T1]{fontenc}
\usepackage{textcomp}
\usepackage{url}
\usepackage{graphicx}
\usepackage{float}
\usepackage{booktabs}
\usepackage{enumitem}

% Don't indent paragraphs.
\usepackage{parskip}

% Hide page number when page is empty
\usepackage{emptypage}
\usepackage{subcaption}
\usepackage{multicol}
\usepackage{xcolor}

% Math stuff
\usepackage{amsmath, amsfonts, mathtools, amsthm, amssymb}
% Non-ams math stuff
\usepackage{derivative, physics}
% Fancy script capitals
\usepackage{mathrsfs}
\usepackage{cancel}

% Bold math
\usepackage{bm}

% Cool command shortcuts
\newcommand\N{\ensuremath{\mathbb{N}}}
\newcommand\R{\ensuremath{\mathbb{R}}}
\newcommand\Z{\ensuremath{\mathbb{Z}}}
\renewcommand\O{\ensuremath{\emptyset}}
\newcommand\Q{\ensuremath{\mathbb{Q}}}
\newcommand\C{\ensuremath{\mathbb{C}}}
\newcommand{\pnorm}[2]{\lVert #2 \rVert_{#1}}
\newcommand{\sz}[1]{\lvert #1 \rvert}

% Operators
\DeclareMathOperator{\argmin}{argmin}

% Easily typeset systems of equations (French package)
\usepackage{systeme}

%Make implies and impliedby shorter
\let\implies\Rightarrow
\let\impliedby\Leftarrow
\let\iff\Leftrightarrow
\let\epsilon\varepsilon
\let\phi\varphi

% Add \contra symbol to denote contradiction
\usepackage{stmaryrd} % for \lightning
\newcommand\contra{\scalebox{1.5}{$\lightning$}}

% horizontal rule
\newcommand\hr{
    \noindent\rule[0.5ex]{\linewidth}{0.5pt}
}

% For box around Definition, Theorem, \ldots
\usepackage{mdframed}
\mdfsetup{skipabove=1em,skipbelow=0em}
\theoremstyle{definition}
\newmdtheoremenv[nobreak=true]{definition}{Definition}
\newmdtheoremenv[nobreak=true]{lemma}{Lemma}
\newmdtheoremenv[nobreak=true]{proposition}{Proposition}
\newmdtheoremenv[nobreak=true]{theorem}{Theorem}
\newmdtheoremenv[nobreak=true]{corollary}{Corollaty}
\newmdtheoremenv{conjecture}{Conjecture}
\newtheorem*{remark}{Remark}
\newtheorem*{problem}{Problem}
\newtheorem*{eg}{Example}
\newtheorem*{question}{Question}
\newtheorem*{intuition}{Intuition}

% End example environments with a small diamond (just like proof
% environments end with a small square)
\usepackage{etoolbox}
\AtEndEnvironment{eg}{\null\hfill$\diamond$}%

% Fix some spacing
% http://tex.stackexchange.com/questions/22119/how-can-i-change-the-spacing-before-theorems-with-amsthm
\makeatletter
\def\thm@space@setup{%
  \thm@preskip=\parskip \thm@postskip=0pt
}

% Exercise 
% Usage:
% \oefening{5}
% \suboefening{1}
% \suboefening{2}
% \suboefening{3}
% gives
% Oefening 5
%   Oefening 5.1
%   Oefening 5.2
%   Oefening 5.3
\newcommand{\exercise}[1]{%
    \def\@exercise{#1}%
    \subsection*{Exercise #1}
}

\newcommand{\subexercise}[1]{%
    \subsubsection*{Exercise \@exercise.#1}
}

% \lecture starts a new lecture (les in dutch)
%
% Usage:
% \lecture{1}{di 12 feb 2019 16:00}{Inleiding}
%
% This adds a section heading with the number / title of the lecture and a
% margin paragraph with the date.

% I use \dateparts here to hide the year (2019). This way, I can easily parse
% the date of each lecture unambiguously while still having a human-friendly
% short format printed to the pdf.

\usepackage{xifthen}
\def\testdateparts#1{\dateparts#1\relax}
\def\dateparts#1 #2 #3 #4 #5\relax{
    \marginpar{\small\textsf{\mbox{#1 #2 #3 #5}}}
}

\def\@lecture{}%
\newcommand{\lecture}[3]{
    \ifthenelse{\isempty{#3}}{%
        \def\@lecture{Lecture #1}%
    }{%
        \def\@lecture{Lecture #1: #3}%
    }%
    \subsection*{\@lecture}
    \marginpar{\small\textsf{\mbox{#2}}}
}



% These are the fancy headers
\usepackage{fancyhdr}
\pagestyle{fancy}

% LE: left even
% RO: right odd
% CE, CO: center even, center odd
% My name for when I print my lecture notes to use for an open book exam.
% \fancyhead[LE,RO]{Gilles Castel}

\fancyhead[RO,LE]{\@lecture} % Right odd,  Left even
\fancyhead[RE,LO]{}          % Right even, Left odd

\fancyfoot[RO,LE]{\thepage}  % Right odd,  Left even
\fancyfoot[RE,LO]{}          % Right even, Left odd
\fancyfoot[C]{\leftmark}     % Center

\makeatother

% Todonotes and inline notes in fancy boxes
\usepackage{todonotes}
\usepackage{tcolorbox}

% Fix some stuff
% %http://tex.stackexchange.com/questions/76273/multiple-pdfs-with-page-group-included-in-a-single-page-warning
\pdfsuppresswarningpagegroup=1


% name
\author{Asier García Ruiz}

\title{Analysis II HW 6}

\begin{document}
    \maketitle

    \exercise{1}
    We will first prove that $\cos \frac{\pi}{2} = 0$.
    \begin{proof}
        Using the angle summation identity for $\cos$ we get that 
        \begin{equation*}
            \cos(\frac{\pi}{2}) = \cos(\pi - \frac{\pi}{2})
            = \cos \pi \cos \frac{\pi}{2} - \sin \pi \sin \frac{\pi}{2}
            = \cos \pi \cos \frac{\pi}{2}.
        \end{equation*}
        Rearranging,
        \begin{equation*}
            \cos \frac{\pi}{2} (\cos \pi) + 1 = 0,
        \end{equation*}
        therefore,
        \begin{equation*}
            \cos \frac{\pi}{2} = 0 \quad \text{and} \quad \cos \pi = -1.
        \end{equation*}
    \end{proof}

    Now, we prove that $\cos x > 0, x \in [0, \frac{\pi}{2})$.
    \begin{proof}
        As shown in class, $(\cos x)^{\prime} = -\sin x < 0, x \in [0, \pi)$.
        Therefore, $\cos x$ is strictly decresing on $(0, \pi)$.
        However, as shown before, $\cos \frac{\pi}{2} = 0$, so it must be the case that 
        $\cos x > 0, x \in [0, \frac{\pi}{2})$.
    \end{proof}

    \exercise{2}
    \subexercise{a}
    We start by noting that if we take $\sin^{2} x + \cos^{2} x = 1$ and divide by 
    $\cos^{2}x$ on both sides we get $\tan^{2} x + 1 = \sec^{2} x$.

    Now, using the quotient rule
    \begin{align*}
        \dv{}{x}\tan x
        &= \dv{}{x} \frac{\sin x}{\cos x},\\ 
        &= \frac{\cos^{2} x + \sin^{2} x}{\cos^{2} x},\\ 
        &= \sec^{2}x = 1 + \tan^{2} x,
    \end{align*}
    as needed.

    \subexercise{b}
    Since $\sin x$ is continuous everywhere and $\cos x = 0$ only when $x = \pm \frac{\pi}{2}$,
    and also continuous everywhere, we get that $\tan x$ is continuous in the interval.

    Now, since $(\tan x)^{\prime} = 1 + \tan^{2} x \geq 1$ for all $x \in (-\frac{\pi}{2}, \frac{\pi}{2})$,
    it is also increasing in the interval.

    \subexercise{c}
    By the Inverse Function Thm. we have that
    \begin{equation*}
        (\arctan x)^{\prime} = \frac{1}{1 + (\tan \arctan x)^{2}} = \frac{1}{1 + x^{2}}
    \end{equation*}

    Then taking the derivative of the other side,
    \begin{equation*}
        \dv{}{x} \sum_{n = 0}^{\infty} \frac{(-1)^{n}x^{2n + 1}}{(2n + 1)}
        = \sum_{n = 0}^{\infty} (-1)^{n} x^{2n} = \frac{1}{1 + x^{2}}.
    \end{equation*}
    Therefore, by the uniqueness of power series,
    \begin{equation*}
        \dv{}{x} \arctan x = \dv{}{x} \sum_{n = 0}^{\infty} \frac{(-1)^{n}x^{2n + 1}}{(2n + 1)}
    \end{equation*}
    and
    \begin{equation*}
         \arctan x = \sum_{n = 0}^{\infty} \frac{(-1)^{n}x^{2n + 1}}{(2n + 1)}
    \end{equation*}

    \exercise{3}

    \exercise{4}
    \begin{proof}
        Let $\sum_{i, j} \abs{a_{i,j}}$ converge. Now, let $b_{i,j} = \max(a_{i,j}, 0)$ and
        $c_{i,j} = \max(-a_{i,j}, 0)$. We now have that 
        $0 \leq b_{i,j} \leq \abs{a_{i,j}}$ and $0 \leq c_{i,j} \leq \abs{a_{i,j}}$.
        Therefore, $\sum_{i,j} b_{i,j}, \sum_{i,j} c_{i,j}$ converge (using the comparison test).

        This now implies that 
        \begin{equation*}
        \sum_{i,j} b_{i,j} - \sum_{i,j} c_{i,j} = \sum_{i,j} b_{i,j} - c_{i,j}
        = \sum_{i,j} a_{i,j}
        \end{equation*}
        converges. We can combine the sums because the summations are finite (they converge).
    \end{proof}

    
\end{document}
