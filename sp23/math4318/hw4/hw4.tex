\documentclass[twoside]{article}

% Some basic packages
\usepackage[utf8]{inputenc}
\usepackage[T1]{fontenc}
\usepackage{textcomp}
\usepackage{url}
\usepackage{graphicx}
\usepackage{float}
\usepackage{booktabs}
\usepackage{enumitem}

% Don't indent paragraphs.
\usepackage{parskip}

% Hide page number when page is empty
\usepackage{emptypage}
\usepackage{subcaption}
\usepackage{multicol}
\usepackage{xcolor}

% Math stuff
\usepackage{amsmath, amsfonts, mathtools, amsthm, amssymb}
% Non-ams math stuff
\usepackage{derivative, physics}
% Fancy script capitals
\usepackage{mathrsfs}
\usepackage{cancel}

% Bold math
\usepackage{bm}

% Cool command shortcuts
\newcommand\N{\ensuremath{\mathbb{N}}}
\newcommand\R{\ensuremath{\mathbb{R}}}
\newcommand\Z{\ensuremath{\mathbb{Z}}}
\renewcommand\O{\ensuremath{\emptyset}}
\newcommand\Q{\ensuremath{\mathbb{Q}}}
\newcommand\C{\ensuremath{\mathbb{C}}}

% Operators
\DeclareMathOperator*{\argmax}{arg\,max}
\DeclareMathOperator*{\argmin}{arg\,min}

% Easily typeset systems of equations (French package)
\usepackage{systeme}

%Make implies and impliedby shorter
\let\implies\Rightarrow
\let\impliedby\Leftarrow
\let\iff\Leftrightarrow
\let\epsilon\varepsilon
\let\phi\varphi

% Add \contra symbol to denote contradiction
\usepackage{stmaryrd} % for \lightning
\newcommand\contra{\scalebox{1.5}{$\lightning$}}

% horizontal rule
\newcommand\hr{
    \noindent\rule[0.5ex]{\linewidth}{0.5pt}
}

% For box around Definition, Theorem, \ldots
\usepackage{mdframed}
\mdfsetup{skipabove=1em,skipbelow=0em}
\theoremstyle{definition}
\newmdtheoremenv[nobreak=true]{definition}{Definition}
\newmdtheoremenv[nobreak=true]{lemma}{Lemma}
\newmdtheoremenv[nobreak=true]{proposition}{Proposition}
\newmdtheoremenv[nobreak=true]{theorem}{Theorem}
\newmdtheoremenv[nobreak=true]{corollary}{Corollary}
\newmdtheoremenv{conjecture}{Conjecture}
\newtheorem*{remark}{Remark}
\newtheorem*{problem}{Problem}
\newtheorem*{eg}{Example}
\newtheorem*{question}{Question}
\newtheorem*{intuition}{Intuition}
\newtheorem*{claim}{Claim}

% End example environments with a small diamond (just like proof
% environments end with a small square)
\usepackage{etoolbox}
\AtEndEnvironment{eg}{\null\hfill$\diamond$}%

% Fix some spacing
% http://tex.stackexchange.com/questions/22119/how-can-i-change-the-spacing-before-theorems-with-amsthm
\makeatletter
\def\thm@space@setup{%
  \thm@preskip=\parskip \thm@postskip=0pt
}

% Exercise 
% Usage:
% \oefening{5}
% \suboefening{1}
% \suboefening{2}
% \suboefening{3}
% gives
% Oefening 5
%   Oefening 5.1
%   Oefening 5.2
%   Oefening 5.3
\newcommand{\exercise}[1]{%
    \def\@exercise{#1}%
    \subsection*{Exercise #1}
}

\newcommand{\subexercise}[1]{%
    \subsubsection*{Exercise \@exercise.#1}
}

% \lecture starts a new lecture (les in dutch)
%
% Usage:
% \lecture{1}{di 12 feb 2019 16:00}{Inleiding}
%
% This adds a section heading with the number / title of the lecture and a
% margin paragraph with the date.

% I use \dateparts here to hide the year (2019). This way, I can easily parse
% the date of each lecture unambiguously while still having a human-friendly
% short format printed to the pdf.

\usepackage{xifthen}
\def\testdateparts#1{\dateparts#1\relax}
\def\dateparts#1 #2 #3 #4 #5\relax{
    \marginpar{\small\textsf{\mbox{#1 #2 #3 #5}}}
}

\def\@lecture{}%
\newcommand{\lecture}[3]{
    \ifthenelse{\isempty{#3}}{%
        \def\@lecture{Lecture #1}%
    }{%
        \def\@lecture{Lecture #1: #3}%
    }%
    \subsection*{\@lecture}
    \marginpar{\small\textsf{\mbox{#2}}}
}


% For page size and geometry
\usepackage{geometry}

% These are the fancy headers
\usepackage{fancyhdr}
\pagestyle{fancy}

% LE: left even
% RO: right odd
% CE, CO: center even, center odd
% My name for when I print my lecture notes to use for an open book exam.
% \fancyhead[LE,RO]{Gilles Castel}

\fancyhead[RO,LE]{\@lecture} % Right odd,  Left even
\fancyhead[RE,LO]{}          % Right even, Left odd

\fancyfoot[RO,LE]{\thepage}  % Right odd,  Left even
\fancyfoot[RE,LO]{}          % Right even, Left odd
\fancyfoot[C]{\leftmark}     % Center

\makeatother

% Todonotes and inline notes in fancy boxes
\usepackage{todonotes}
\usepackage{tcolorbox}

% Fix some stuff
% %http://tex.stackexchange.com/questions/76273/multiple-pdfs-with-page-group-included-in-a-single-page-warning
\pdfsuppresswarningpagegroup=1


% name
\author{Asier García Ruiz}


\begin{document}
\exercise{1}
\begin{proof}
	We begin by noting that
	\begin{equation*}
		F(x + 1) = \lim_{N \to \infty} \int_{0}^{N} t^{x + 1}e^{-t} \ dt.
	\end{equation*}
	We let $u = t^{x + 1}, dv = e^{-t}$
	and then we have that $du = (x + 1)t^{x}, v = -e^{-t}$.
	Now, using integration by parts we have that
	\begin{align*}
		F(x + 1)
		 & = \lim_{N \to \infty} \int_{0}^{N} t^{x + 1}e^{-t} \ dt,                        \\
		 & = \lim_{N \to \infty} \left[-t^{x + 1}e^{-t}\right]_{0}^{N}
		+ \int_{0}^{N} (x + 1)t^{x}e^{-t} \ dt,                                            \\
		 & = \lim_{N \to \infty} -N_{x + 1}e^{-N} + (x + 1) \int_{0}^{N} t^{x}e^{-t} \ dt, \\
		 & = (x + 1) \lim_{N \to \infty} \int_{0}^{N} t^{x}e^{-t} \ dt,                    \\
		 & = (x + 1)F(x),
	\end{align*}
	as required.
\end{proof}

\exercise{2}
\begin{proof}
	Assume, for the sake of contradiction, that $F(t)$ is differentiable at $c$.
	Then, using the limit definition of the derivative, we have that
	\begin{equation*}
		F^{\prime}(c) = \lim_{t \to c} \frac{F(t) - F(c)}{t - c}
		= \lim_{t \to c} \frac{1}{t - c}\left[\int_{a}^{t}f(x) \ dx - \int_{a}^{c} f(x) \ dx\right].
	\end{equation*}
	Now we can use l'H\^opital's rule with the fundamental theorem of calculus
	to get that
	\begin{equation*}
		F^{\prime}(c) = \lim_{t \to c} f(t).
	\end{equation*}
	However, we know that $f$ is not continous at $c$, so this limit does not
	exist \contra.

	Therefore, we have that $F(t)$ is not differentiable at $c$ as required.
\end{proof}

\exercise{3}
\begin{proof}
	Assume, for the sake of contradiction, that $\exists c \in [a, b]$ such that
	$f(c) > 0$. Now, by the continuity of $f$, we have that for any $\epsilon > 0$,
	there exists $\delta > 0$ such that $|x - c| < \delta$
	implies $|f(x) - f(c)| < \epsilon$. This implies that
	\begin{equation*}
		\int_{A}^{B} f(x) \ dx > 0
	\end{equation*}
	where $A = \max\{c - \delta, a\}, B = \min\{c + \delta, b\}$. Now, by the properties
	of integrals we have that
	\begin{equation*}
		\int_{a}^{b} f(x) \ dx = \int_{a}^{A} f(x) \ dx + \int_{A}^{B} f(x) \ dx
		+ \int_{B}^{b} \ dx > 0 \quad \contra.
	\end{equation*}
	Therefore, we have that $f(x) = 0$ for all $x \in [a,b]$ as required.
\end{proof}

\exercise{4}
Consider the function
\begin{equation*}
	f(x) =
	\begin{cases}
		1 \text{ if } x \in \Q, \\
		-1 \text{ if } x \not\in \Q.
	\end{cases}
\end{equation*}
This function is not Riemann integrable on $[0, 1]$, but
\begin{equation*}
	f^{2} = 1,
\end{equation*}
is clearly Riemann integrable on $[0, 1]$.

\exercise{5}
\begin{proof}
	We start by noting that the rationals and irrationals are dense in the reals. This means
	that is in every interval $[c, d] \in [0, 1]$ there exists a rational $q \in [c, d]$ and
	and irrational $r \in [c, d]$

	Now, we consider any lower sum of $f$ and note that for every partition $\bm{P}$ of $[0, 1]$
	\begin{equation*}
		\mathcal{L}(f, \bm{P}) = \sum_{I \in \bm{P}} \inf_{I} (f) * |I| = 0,
	\end{equation*}
	since $\inf_{I} (f) = 0$ for all $I \in \bm{P}$.

	Now, we consider any upper sum of $f$, for every partition $P$ of $[0, 1]$
	\begin{equation*}
		\mathcal{U}(f, \bm{P}) = \sum_{I \in \bm{P}} \sup_{I} (f) * |I| > 0
	\end{equation*}
	since $\sup_{I} > 0$ for all $I \in \bm{P}$ and $|I| \neq 0 \ \forall I \in \bm{P}$.

	Therefore, since $\mathcal{L}(f, \bm{P}) \neq \mathcal{U}(f, \bm{P})$ for any $P$, we have that
	$f$ is not Riemann integrable on $[0, 1]$ as required.
\end{proof}
\end{document}
