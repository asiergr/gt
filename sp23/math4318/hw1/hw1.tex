\documentclass{article}
% Some basic packages
\usepackage[utf8]{inputenc}
\usepackage[T1]{fontenc}
\usepackage{textcomp}
\usepackage{url}
\usepackage{graphicx}
\usepackage{float}
\usepackage{booktabs}
\usepackage{enumitem}

% Don't indent paragraphs.
\usepackage{parskip}

% Hide page number when page is empty
\usepackage{emptypage}
\usepackage{subcaption}
\usepackage{multicol}
\usepackage{xcolor}

% Math stuff
\usepackage{amsmath, amsfonts, mathtools, amsthm, amssymb}
% Non-ams math stuff
\usepackage{derivative, physics}
% Fancy script capitals
\usepackage{mathrsfs}
\usepackage{cancel}

% Bold math
\usepackage{bm}

% Cool command shortcuts
\newcommand\N{\ensuremath{\mathbb{N}}}
\newcommand\R{\ensuremath{\mathbb{R}}}
\newcommand\Z{\ensuremath{\mathbb{Z}}}
\renewcommand\O{\ensuremath{\emptyset}}
\newcommand\Q{\ensuremath{\mathbb{Q}}}
\newcommand\C{\ensuremath{\mathbb{C}}}
\newcommand{\pnorm}[2]{\lVert #2 \rVert_{#1}}
\newcommand{\sz}[1]{\lvert #1 \rvert}

% Operators
\DeclareMathOperator{\argmin}{argmin}

% Easily typeset systems of equations (French package)
\usepackage{systeme}

%Make implies and impliedby shorter
\let\implies\Rightarrow
\let\impliedby\Leftarrow
\let\iff\Leftrightarrow
\let\epsilon\varepsilon
\let\phi\varphi

% Add \contra symbol to denote contradiction
\usepackage{stmaryrd} % for \lightning
\newcommand\contra{\scalebox{1.5}{$\lightning$}}

% horizontal rule
\newcommand\hr{
    \noindent\rule[0.5ex]{\linewidth}{0.5pt}
}

% For box around Definition, Theorem, \ldots
\usepackage{mdframed}
\mdfsetup{skipabove=1em,skipbelow=0em}
\theoremstyle{definition}
\newmdtheoremenv[nobreak=true]{definition}{Definition}
\newmdtheoremenv[nobreak=true]{lemma}{Lemma}
\newmdtheoremenv[nobreak=true]{proposition}{Proposition}
\newmdtheoremenv[nobreak=true]{theorem}{Theorem}
\newmdtheoremenv[nobreak=true]{corollary}{Corollaty}
\newmdtheoremenv{conjecture}{Conjecture}
\newtheorem*{remark}{Remark}
\newtheorem*{problem}{Problem}
\newtheorem*{eg}{Example}
\newtheorem*{question}{Question}
\newtheorem*{intuition}{Intuition}

% End example environments with a small diamond (just like proof
% environments end with a small square)
\usepackage{etoolbox}
\AtEndEnvironment{eg}{\null\hfill$\diamond$}%

% Fix some spacing
% http://tex.stackexchange.com/questions/22119/how-can-i-change-the-spacing-before-theorems-with-amsthm
\makeatletter
\def\thm@space@setup{%
  \thm@preskip=\parskip \thm@postskip=0pt
}

% Exercise 
% Usage:
% \oefening{5}
% \suboefening{1}
% \suboefening{2}
% \suboefening{3}
% gives
% Oefening 5
%   Oefening 5.1
%   Oefening 5.2
%   Oefening 5.3
\newcommand{\exercise}[1]{%
    \def\@exercise{#1}%
    \subsection*{Exercise #1}
}

\newcommand{\subexercise}[1]{%
    \subsubsection*{Exercise \@exercise.#1}
}

% \lecture starts a new lecture (les in dutch)
%
% Usage:
% \lecture{1}{di 12 feb 2019 16:00}{Inleiding}
%
% This adds a section heading with the number / title of the lecture and a
% margin paragraph with the date.

% I use \dateparts here to hide the year (2019). This way, I can easily parse
% the date of each lecture unambiguously while still having a human-friendly
% short format printed to the pdf.

\usepackage{xifthen}
\def\testdateparts#1{\dateparts#1\relax}
\def\dateparts#1 #2 #3 #4 #5\relax{
    \marginpar{\small\textsf{\mbox{#1 #2 #3 #5}}}
}

\def\@lecture{}%
\newcommand{\lecture}[3]{
    \ifthenelse{\isempty{#3}}{%
        \def\@lecture{Lecture #1}%
    }{%
        \def\@lecture{Lecture #1: #3}%
    }%
    \subsection*{\@lecture}
    \marginpar{\small\textsf{\mbox{#2}}}
}



% These are the fancy headers
\usepackage{fancyhdr}
\pagestyle{fancy}

% LE: left even
% RO: right odd
% CE, CO: center even, center odd
% My name for when I print my lecture notes to use for an open book exam.
% \fancyhead[LE,RO]{Gilles Castel}

\fancyhead[RO,LE]{\@lecture} % Right odd,  Left even
\fancyhead[RE,LO]{}          % Right even, Left odd

\fancyfoot[RO,LE]{\thepage}  % Right odd,  Left even
\fancyfoot[RE,LO]{}          % Right even, Left odd
\fancyfoot[C]{\leftmark}     % Center

\makeatother

% Todonotes and inline notes in fancy boxes
\usepackage{todonotes}
\usepackage{tcolorbox}

% Fix some stuff
% %http://tex.stackexchange.com/questions/76273/multiple-pdfs-with-page-group-included-in-a-single-page-warning
\pdfsuppresswarningpagegroup=1


% name
\author{Asier García Ruiz}


\title{Homework 1}
\begin{document}
\maketitle
    \exercise{1}

    Prove that the function $f(x) = \abs{x})$ is differentiable for all $x \neq 0$, but is not differentiable for $x = 0$.

    \begin{proof}
        We will begin by showing that $f(x)$ is not differentiable at $x = 0$. We know the definition of the derivative is
        \begin{equation*}
            \lim_{h \to 0} \frac{f(x + h) - f(x)}{h}.
        \end{equation*}

        Now, we interpret $f(x)$ as a piece-wise function.
        \begin{equation*}
            f(x) = \begin{cases}
                x & \text{if } x >= 0,\\
                -x & \text{if } x < 0
                \end{cases}
        \end{equation*}
        Now, let us consider the left and right handed limits of this function. We will start with the left limit
        \begin{equation*}
            \lim_{h \to 0^{-}} \frac{-x - h + x}{h} = -1.
        \end{equation*}

        Now, we consider the right limit.
        \begin{equation*}
            \lim_{h \to 0^{+}} \frac{x + h -x}{h} = 1.
        \end{equation*}

        Since the right and left handed limits at $x = 0$ are different, we have that the function is not differentiable at 
        this point. However, at any other point, the function simply behaves like a linear function, so if has a derivative.
        Namely,

        \begin{equation*}
            f '(x) = \begin{cases}
                1 & \text{if } x > 0,\\
                -1 & \text{if } x < 0.
                \end{cases}
        \end{equation*}
    \end{proof}

    \newpage
    \exercise{2}
    Find where the function
    \begin{equation*}
        f(x) =
        \begin{cases}
            x^{2} & \text{ if } x \text{ is rational},\\
            0  & \text{ if } x \text{ is irrational}.
        \end{cases}
    \end{equation*}
    is (a) continuous, and (b) differentiable.

    \begin{proof}
        We will first prove that this function is continuous at $x=0$. Let $\epsilon > 0$ and $\delta = \sqrt{\epsilon}$.
        Now, we let $\abs{x - 0} < \delta$. We can now see that if $x$ is irrational, then $\abs{f(x) - f(0)} = 0 < \epsilon$
        If $x$ is irrational, then $\abs{f(x) - f(0)} = x^{2} < \delta^{2} = \sqrt{\epsilon}^{2} = \epsilon $.
        So, we always have that $\abs{f(x) - f(0)} < \epsilon$. Therefore, $f$ is continuous at $x = 0$.

        Now, we show that $f$ is continuous nowhere else. Suppose that $x \neq 0$ is rational. We note that the irrationals are dense
        in $\R$, so any open interval in $\R$ contains an irrational number. Now, we let $\epsilon = \frac{x^{2}}{2}$. We let
        $x_{0} \in (x -\delta, x + \delta)$ be irrational. Now, we have that 
        \begin{equation*}
            \abs{f(x) - f(x_{0})} = \abs{x^{2} - 0} = x^{2} \geq \frac{x^{2}}{2} = \epsilon.
        \end{equation*}
        Therefore, $f$ is not continuous when $x \neq 0$ is rational.

        Similarly, we let $x \neq 0$ an irrational number. We note that the rationals are dense in $\R$. We let 
        $\epsilon = x^{2}, \delta > 0$. We let $q$ be a rational in the interval $(x - \delta, x + \delta)$ such that
        $f(q) = q^{2} > x^{2}$. Finally, we get that
        \begin{equation*}
            \abs{f(x) - f(x_{0})} = \abs{0 - q^{2}} = q^{2} > x^{2} = \epsilon.
        \end{equation*}
        Therefore, $f$ is not continous at the irrational $x \neq 0$.
    \end{proof}

    \subexercise{b}
    \begin{proof}
        Clearly since the function is discontinuous at $x \neq 0$, it is not differentiable at $x \neq 0$. Let us check what the 
        limits as $x \to 0$ are. Again, since the irrationals are dense in the reals, we can approach this limit
        from the rational or the irrational numbers. 
        We have that 
        \begin{equation*}
            \lim_{x \to 0, x \in \Q} \frac{f(x) - f(0)}{x - 0} = 0,
        \end{equation*}
        and 
        \begin{equation*}
            \lim_{x \to 0, x \in \R \setminus  \Q} \frac{f(x) - f(0)}{x - 0} = 0.
        \end{equation*}
        
        Therefore, since the limits agree, $f(x)$ is differentiable at 0.
    \end{proof}

    \newpage
    \exercise{3}
    Consider the function
    \begin{equation*}
        f(x) = 
        \begin{cases}
            x^{2}\sin(\frac{1}{x}) & \text{ if } x \neq 0,\\
            0 & \text{ if } x = 0.
        \end{cases}
    \end{equation*}

    \subexercise{a}
    Prove that $x$ is differentiable on $\R$ (you may use differentiation rules for trigonometric, functions)

    \begin{proof}
        We can explicitly calculate the derivative using first principles to get
        \begin{equation*}
            \dv{f}{x}(x) = 
            \begin{cases}
                2x\sin(\frac{1}{x}) - \cos(\frac{1}{x}) & \text{ if } x \neq 0,\\
                0 & \text{ if } x = 0.
            \end{cases}
        \end{equation*}
        \end{proof}

    \subexercise{b}
    \begin{proof}
        We will prove that $\dv{f}{x}$ is not continuous at $x = 0$. To do this we simply look at the limit as $x \to 0$. We have that
        \begin{equation*}
            \lim_{x \to 0} 2x\sin(\frac{1}{x}) + \cos(\frac{1}{x}).
        \end{equation*}
        We simply replace $z = \frac{1}{x}$ to get 
        \begin{equation*}
            \lim_{z \to \infty} 2\frac{\sin(z)}{z} + \cos(z).
        \end{equation*}

        While $\lim_{z \to \infty} 2 \frac{\sin z}{z} = 0$, we have that $\lim_{z \to \infty} \cos z$ does not exist. Therefore,
        $\dv{f}{x}$ is not differentiable at $x = 0$ since the limit does not agree with $f(0)$.
    \end{proof}
    
    \newpage
    \exercise{4}
    \subexercise{a}
    Suppose that $f: [a,b] \to \R$ is differentiable on $[a,b]$. Suppose $f^{\prime} (a) < f^{\prime}(b)$ and 
    $\beta \in (f^{\prime}(a), f^{\prime}(b))$ prove that $g(x) = f(x) - \beta(x - a)$ obtains its minimum in $[a,b]$ at a point in the 
    open interval $(a,b)$.

    \begin{proof}
        We begin by calculating
        \begin{equation*}
            g^{\prime}(x) = f^{\prime}(x) - \beta.
        \end{equation*}
        Now, we see that $g^{\prime}(a) = f^{\prime}(a) - \beta$ and $g^{\prime}(b) = f^{\prime}(b) - \beta$. Now, since 
        $\beta \in (f^{\prime}(a), f^{\prime}(b))$ we know that $g^{\prime}(a) < 0$ and $g^{\prime}(b) > 0$. Meaning that
        the minimum point cannot lie at $x = a,b$. Therefore, it must lie in $(a,b)$.
    \end{proof}

    \subexercise{b}
    Conclude that if $\beta \in (f^{\prime}(a), f^{\prime}(b))$, then there exists $c \in (a,b)$ with $f^{\prime}(c) = \beta$.

    \begin{proof}
        Since we know the stationary point is a minimum, we can simply set $g^{\prime}(x) = 0$ to find it. We get that
        \begin{gather*}
            g^{\prime}(x) = f^{\prime}(x) - \beta = 0,\\
            f^{\prime}(x) = \beta.
        \end{gather*}
        Therefore, we get that there exists $c \in (a,b)$ with $f^{\prime}(c) = \beta$ as needed.
    \end{proof}

    \newpage
    \exercise{5}
    Consider the polynomial
    \begin{equation*}
        p(x) = x^{9} + 3x^{7} + 21x^{3} + 321,
    \end{equation*}
    
    The goal of this question is to show that $p$ has only one root.

    \subexercise{a}
    Suppose $p$ has two roots. Find a contradiction using Rolle's theorem.
    \begin{proof}
        Suppose we have two roots at $x = a,b$. By Rolle's theorem we know $\exists c \in (a,b) \text{ such that } f(c) = 0$.
        Now, we note that 
        \begin{equation*}
            p^{\prime}(x) = 9x^{8} + 21x^{6} + 63x^{2}.
        \end{equation*}
        Therefore, the only place that $p^{\prime}(x) = 0$ is at $x = 0$. This implies that $a < 0 < b$. However, we also note 
        that $p^{\prime}(x) > 0 \ \ \forall x > 0$, so $p(x)$. Therefore, if $p(x)$ has a root at $a$, 
        and only increases starting at 0, it cannot have another root after that, which is where the root would be as per
        our first finding. \contra
    \end{proof}

    \subexercise{b}
    Show that $p$ has at least one root.

    \begin{proof}
        We can see that $p(-2) = -743$ and $p(-1) = 296$. By the Intermediate Value Theorem, $\exists c \in (-2, -1)$ such that
        $p(c) = 0$.
    \end{proof}

\end{document}
