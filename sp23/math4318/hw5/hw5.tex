\documentclass[twoside]{article}

% Some basic packages
\usepackage[utf8]{inputenc}
\usepackage[T1]{fontenc}
\usepackage{textcomp}
\usepackage{url}
\usepackage{graphicx}
\usepackage{float}
\usepackage{booktabs}
\usepackage{enumitem}

% Don't indent paragraphs.
\usepackage{parskip}

% Hide page number when page is empty
\usepackage{emptypage}
\usepackage{subcaption}
\usepackage{multicol}
\usepackage{xcolor}

% Math stuff
\usepackage{amsmath, amsfonts, mathtools, amsthm, amssymb}
% Non-ams math stuff
\usepackage{derivative, physics}
% Fancy script capitals
\usepackage{mathrsfs}
\usepackage{cancel}

% Bold math
\usepackage{bm}

% Cool command shortcuts
\newcommand\N{\ensuremath{\mathbb{N}}}
\newcommand\R{\ensuremath{\mathbb{R}}}
\newcommand\Z{\ensuremath{\mathbb{Z}}}
\renewcommand\O{\ensuremath{\emptyset}}
\newcommand\Q{\ensuremath{\mathbb{Q}}}
\newcommand\C{\ensuremath{\mathbb{C}}}
\newcommand{\pnorm}[2]{\lVert #2 \rVert_{#1}}
\newcommand{\sz}[1]{\lvert #1 \rvert}

% Operators
\DeclareMathOperator{\argmin}{argmin}

% Easily typeset systems of equations (French package)
\usepackage{systeme}

%Make implies and impliedby shorter
\let\implies\Rightarrow
\let\impliedby\Leftarrow
\let\iff\Leftrightarrow
\let\epsilon\varepsilon
\let\phi\varphi

% Add \contra symbol to denote contradiction
\usepackage{stmaryrd} % for \lightning
\newcommand\contra{\scalebox{1.5}{$\lightning$}}

% horizontal rule
\newcommand\hr{
    \noindent\rule[0.5ex]{\linewidth}{0.5pt}
}

% For box around Definition, Theorem, \ldots
\usepackage{mdframed}
\mdfsetup{skipabove=1em,skipbelow=0em}
\theoremstyle{definition}
\newmdtheoremenv[nobreak=true]{definition}{Definition}
\newmdtheoremenv[nobreak=true]{lemma}{Lemma}
\newmdtheoremenv[nobreak=true]{proposition}{Proposition}
\newmdtheoremenv[nobreak=true]{theorem}{Theorem}
\newmdtheoremenv[nobreak=true]{corollary}{Corollaty}
\newmdtheoremenv{conjecture}{Conjecture}
\newtheorem*{remark}{Remark}
\newtheorem*{problem}{Problem}
\newtheorem*{eg}{Example}
\newtheorem*{question}{Question}
\newtheorem*{intuition}{Intuition}

% End example environments with a small diamond (just like proof
% environments end with a small square)
\usepackage{etoolbox}
\AtEndEnvironment{eg}{\null\hfill$\diamond$}%

% Fix some spacing
% http://tex.stackexchange.com/questions/22119/how-can-i-change-the-spacing-before-theorems-with-amsthm
\makeatletter
\def\thm@space@setup{%
  \thm@preskip=\parskip \thm@postskip=0pt
}

% Exercise 
% Usage:
% \oefening{5}
% \suboefening{1}
% \suboefening{2}
% \suboefening{3}
% gives
% Oefening 5
%   Oefening 5.1
%   Oefening 5.2
%   Oefening 5.3
\newcommand{\exercise}[1]{%
    \def\@exercise{#1}%
    \subsection*{Exercise #1}
}

\newcommand{\subexercise}[1]{%
    \subsubsection*{Exercise \@exercise.#1}
}

% \lecture starts a new lecture (les in dutch)
%
% Usage:
% \lecture{1}{di 12 feb 2019 16:00}{Inleiding}
%
% This adds a section heading with the number / title of the lecture and a
% margin paragraph with the date.

% I use \dateparts here to hide the year (2019). This way, I can easily parse
% the date of each lecture unambiguously while still having a human-friendly
% short format printed to the pdf.

\usepackage{xifthen}
\def\testdateparts#1{\dateparts#1\relax}
\def\dateparts#1 #2 #3 #4 #5\relax{
    \marginpar{\small\textsf{\mbox{#1 #2 #3 #5}}}
}

\def\@lecture{}%
\newcommand{\lecture}[3]{
    \ifthenelse{\isempty{#3}}{%
        \def\@lecture{Lecture #1}%
    }{%
        \def\@lecture{Lecture #1: #3}%
    }%
    \subsection*{\@lecture}
    \marginpar{\small\textsf{\mbox{#2}}}
}



% These are the fancy headers
\usepackage{fancyhdr}
\pagestyle{fancy}

% LE: left even
% RO: right odd
% CE, CO: center even, center odd
% My name for when I print my lecture notes to use for an open book exam.
% \fancyhead[LE,RO]{Gilles Castel}

\fancyhead[RO,LE]{\@lecture} % Right odd,  Left even
\fancyhead[RE,LO]{}          % Right even, Left odd

\fancyfoot[RO,LE]{\thepage}  % Right odd,  Left even
\fancyfoot[RE,LO]{}          % Right even, Left odd
\fancyfoot[C]{\leftmark}     % Center

\makeatother

% Todonotes and inline notes in fancy boxes
\usepackage{todonotes}
\usepackage{tcolorbox}

% Fix some stuff
% %http://tex.stackexchange.com/questions/76273/multiple-pdfs-with-page-group-included-in-a-single-page-warning
\pdfsuppresswarningpagegroup=1


% name
\author{Asier García Ruiz}


\begin{document}
    \exercise{1}
    \subexercise{a}
    Consider
    \begin{equation*}
        \sum_{n = 0}^{\infty}x^{n}.
    \end{equation*}

    \subexercise{b}
    Consider
    \begin{equation*}
        \sum_{n = 1}^{\infty} (-1)^{n}\frac{x^{n}}{n}.
    \end{equation*}

    \subexercise{c}
    Consider
    \begin{equation*}
        \sum_{n = 1}^{\infty} (-1)^{n}\frac{x^{2}}{n}.
    \end{equation*}

    \exercise{2}
    \begin{proof}
        We want to show that for some $\epsilon > 0$, when $x \in (-1, 1)$ and $n \geq N$,
        \begin{equation*}
            \left|\sum_{n = 0}^{n - 1} - \frac{1}{1 - x} \right| < \epsilon.
        \end{equation*}

    We begin by noting that the partial sum 
    \begin{equation*}
        S_{n} = \sum_{k = 0}^{n - 1} |x|^{k} = \left| \frac{(1 - x^{n})}{1 - x} \right|
    \end{equation*}
    when $x \in (-1, 1)$. Now we pick some $\epsilon > 0$ and some $n \geq N$.
    We have that 
    \begin{align*}
        \left| \left| \frac{1 - x^{n}}{1 - x} \right| - \frac{1}{1 - x} \right|
        &=
        \intertext{and note that when $x \in (-1, 1), \frac{1}{1 - x} = \left| \frac{1}{1 - x} \right|$.
        So we are left with}
        &= \left| \left| \frac{1 - x^{n}}{1 - x} \right| - \left| \frac{1}{1 - x} \right| \right|,\\
        &\leq \left| \frac{1-x^{n}}{1 - x} - \frac{1}{1 - x} \right|,\\ 
        &= \left|\frac{-x^{n}}{1 - x} \right|,\\ 
        &= \frac{|x|^{n}}{1 - |x|}.
        \intertext{Now, we let $N = \log_{x}(\epsilon(1 - |x|))$,}
        &\leq \frac{\epsilon(1 - |x|)}{1 - |x|} = \epsilon
    \end{align*}
    as needed.

    Therefore, we have absolute convergence as required.

    Now we show that the series does not uniformly converge on $(-1, 1)$.
    Since we have absolute converges, we also know that $S_{n}$ 
    converges pointwise to $\frac{1}{1 - x}$ for $|x| < 1$.
    Now, we consider the subset $0 < x < 1 \subset |x| < 1$. So we have that 
    \begin{equation*}
        \sup_{|x| < 1} \left|\frac{1}{1 - x} - S_{n}(x) \right| \geq \sup_{0 < x < 1} \left|\frac{1}{1 - x} - S^{n}(x)\right|.
    \end{equation*}
    Now, we observe that 
    \begin{equation*}
        \frac{1}{1 - x} - S_{n}(x) > \frac{1}{1 - x} - (k + 1), 0 < x < 1,
    \end{equation*}
    and 
    \begin{equation*}
        \lim_{x \to 1^{-}} \frac{1}{1 - x} = \infty.
    \end{equation*}
    Therefore,
    \begin{equation*}
        \lim_{x \to 1^{-}} \frac{1}{1 - x} - S_{n}(x) = \infty.
    \end{equation*}
    Therefore we have that 
    \begin{equation*}
         \sup_{0 < x < 1} \left|\frac{1}{1 - x} - S^{n}(x)\right| = \infty
    \end{equation*}
    and 
    \begin{equation*}
         \sup_{|x| < 1} \left|\frac{1}{1 - x} S_{n}(x) \right| = \infty
    \end{equation*}
    as needed. Thus, since $n$ is arbitrary, we do not have uniform convergence.
    \end{proof}

    \exercise{3}
    \begin{proof}
        We will start by proving that $f$ is smooth (i.e., infinitely differentiable)
        everywhere.

        We use the power series representation of the exponential function at 0 to find that,
        given some $m \in \N$,
        \begin{equation*}
            \frac{1}{x^{m}} = x\left(\frac{1}{x}\right)^{m + 1} 
            \leq (m + 1)! x \sum_{n = 0}^{\infty}\frac{1}{n!}\left(\frac{1}{x}\right)^{n}
            = (m + 1)!xe^{1/x}, x > 0.
        \end{equation*}

        We now divide by $e^{1/x}$ and take the limit to zero from above
        \begin{equation*}
            \lim_{x \to 0^{+}} \frac{e^{-1/x}}{x^{m}} \leq (m + 1)!\lim_{x \to 0^{+}}x = 0.
        \end{equation*}

        Now, we find the formula for the derivatives of $f$.
        We will show that $f^{(n)}(x) = \frac{p_{n}(x)}{x^{2n}}$ if $x > 0$ and 0 if $x \leq 0$ 
        where $p_{n + 1} = x^{2}p^{\prime}(x) - (2nx - 1)p_{n}(x), p_{1}(x) = 1$
        The case when $x \leq 0$ is trivial, so we focus on the former.

        For the base case we have that 
        $f^{\prime}(0) = \lim_{x \to 0^{+}} = \frac{f(x) - f(0)}{x - 0} = 
        \lim_{x \to 0^{+}} \frac{e^{-1/x}}{x} = 0$.

        Now we have that
        \begin{equation*}
            f^{(n + 1)}(0) = \left(\frac{p^{\prime}_{n}(x)}{x^{2n}} - 2n \frac{p_{n}(x)}{x_{2n + 1}}
                + \frac{p_{n}(x)}{x^{2n+2}}\right)f(x) = \frac{p_{n+1}(x)}{x^{2(n+1)}}f(x).
        \end{equation*}

        For the derivative of $f^{(n)}(0)$ we get that
        \begin{equation*}
            f^{(n)}(0) = \lim_{x \to 0^{+}} = \frac{f^{(n)}(x) - f^{(n)}(0)}{x - 0} = 
            \lim_{x \to 0^{+}} \frac{p_{n}(x)}{x^{2n + 1}}e^{-1/x} = 0.
        \end{equation*}

        Therefore, the derivatives are continuous everywhere, including 0.
        This shows the function is smooth everywhere.

        However, as we have noted, all derivatives at 0 are equal to zero.
        Therefore, the Taylor series of $f$ at the origin converges to 
        zero at all points. This implies that $f$ is not analytic at the origin.
    \end{proof}

    \exercise{4}
    \subexercise{a}
    \begin{proof}
        For the base case we have that
        \begin{equation*}
            |f_{0}(t)| = |f(t)| \leq M
        \end{equation*}
        for some $M \in \R$. We know this is true as it is given to us that
        $f$ is bounded above.

        Now, we assume that
        \begin{equation*}
            |f_{n}(t)| \leq \frac{M}{n!}(t - a)^{n}.
        \end{equation*}

        Consider 
        \begin{align*}
            |f_{n + 1}(t)| 
            &= \left|\int_{a}^{t}f_{n}(x) \ dx\right|,\\ 
            &\leq \int_{a}^{t} |f_{n}(x)| \ dx,\\ 
            &\leq \int_{a}^{t} \frac{M}{n!}(t - a)^{n},\\
            &= \frac{M}{(n + 1)!}(t - a)^{n + 1}
        \end{align*}
        as required. Which concludes our proof.
    \end{proof}

    \subexercise{b}
    \begin{proof}
        We want to show that for some $\epsilon > 0, n \geq N$,
        \begin{equation*}
            |f_{n}(x)| < \epsilon.
        \end{equation*}

        By the result in (a) we already have that
        \begin{align*}
            |f_{n}(x)|
            &\leq \frac{M}{n!}(t - a)^{n},
            \intertext{since $t \in [a,b]$,}
            &\leq \frac{M}{n!}(b - a)^{n},\\
            \intertext{now, we pick $N$ such that $\frac{M}{N!}(b - a)^{N} < \epsilon$
            to get}
            &< \epsilon.
        \end{align*}
        The reason this choice of $N$ works is because $N!$ grows faster than
        any exponential in $N$.

        Therefore, this concludes our proof.
    \end{proof}
\end{document}
