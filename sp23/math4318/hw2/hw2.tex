\documentclass[twoside]{article}

% Some basic packages
\usepackage[utf8]{inputenc}
\usepackage[T1]{fontenc}
\usepackage{textcomp}
\usepackage{url}
\usepackage{graphicx}
\usepackage{float}
\usepackage{booktabs}
\usepackage{enumitem}

% Don't indent paragraphs.
\usepackage{parskip}

% Hide page number when page is empty
\usepackage{emptypage}
\usepackage{subcaption}
\usepackage{multicol}
\usepackage{xcolor}

% Math stuff
\usepackage{amsmath, amsfonts, mathtools, amsthm, amssymb}
% Non-ams math stuff
\usepackage{derivative, physics}
% Fancy script capitals
\usepackage{mathrsfs}
\usepackage{cancel}

% Bold math
\usepackage{bm}

% Cool command shortcuts
\newcommand\N{\ensuremath{\mathbb{N}}}
\newcommand\R{\ensuremath{\mathbb{R}}}
\newcommand\Z{\ensuremath{\mathbb{Z}}}
\renewcommand\O{\ensuremath{\emptyset}}
\newcommand\Q{\ensuremath{\mathbb{Q}}}
\newcommand\C{\ensuremath{\mathbb{C}}}
\newcommand{\pnorm}[2]{\lVert #2 \rVert_{#1}}
\newcommand{\sz}[1]{\lvert #1 \rvert}

% Operators
\DeclareMathOperator{\argmin}{argmin}

% Easily typeset systems of equations (French package)
\usepackage{systeme}

%Make implies and impliedby shorter
\let\implies\Rightarrow
\let\impliedby\Leftarrow
\let\iff\Leftrightarrow
\let\epsilon\varepsilon
\let\phi\varphi

% Add \contra symbol to denote contradiction
\usepackage{stmaryrd} % for \lightning
\newcommand\contra{\scalebox{1.5}{$\lightning$}}

% horizontal rule
\newcommand\hr{
    \noindent\rule[0.5ex]{\linewidth}{0.5pt}
}

% For box around Definition, Theorem, \ldots
\usepackage{mdframed}
\mdfsetup{skipabove=1em,skipbelow=0em}
\theoremstyle{definition}
\newmdtheoremenv[nobreak=true]{definition}{Definition}
\newmdtheoremenv[nobreak=true]{lemma}{Lemma}
\newmdtheoremenv[nobreak=true]{proposition}{Proposition}
\newmdtheoremenv[nobreak=true]{theorem}{Theorem}
\newmdtheoremenv[nobreak=true]{corollary}{Corollaty}
\newmdtheoremenv{conjecture}{Conjecture}
\newtheorem*{remark}{Remark}
\newtheorem*{problem}{Problem}
\newtheorem*{eg}{Example}
\newtheorem*{question}{Question}
\newtheorem*{intuition}{Intuition}

% End example environments with a small diamond (just like proof
% environments end with a small square)
\usepackage{etoolbox}
\AtEndEnvironment{eg}{\null\hfill$\diamond$}%

% Fix some spacing
% http://tex.stackexchange.com/questions/22119/how-can-i-change-the-spacing-before-theorems-with-amsthm
\makeatletter
\def\thm@space@setup{%
  \thm@preskip=\parskip \thm@postskip=0pt
}

% Exercise 
% Usage:
% \oefening{5}
% \suboefening{1}
% \suboefening{2}
% \suboefening{3}
% gives
% Oefening 5
%   Oefening 5.1
%   Oefening 5.2
%   Oefening 5.3
\newcommand{\exercise}[1]{%
    \def\@exercise{#1}%
    \subsection*{Exercise #1}
}

\newcommand{\subexercise}[1]{%
    \subsubsection*{Exercise \@exercise.#1}
}

% \lecture starts a new lecture (les in dutch)
%
% Usage:
% \lecture{1}{di 12 feb 2019 16:00}{Inleiding}
%
% This adds a section heading with the number / title of the lecture and a
% margin paragraph with the date.

% I use \dateparts here to hide the year (2019). This way, I can easily parse
% the date of each lecture unambiguously while still having a human-friendly
% short format printed to the pdf.

\usepackage{xifthen}
\def\testdateparts#1{\dateparts#1\relax}
\def\dateparts#1 #2 #3 #4 #5\relax{
    \marginpar{\small\textsf{\mbox{#1 #2 #3 #5}}}
}

\def\@lecture{}%
\newcommand{\lecture}[3]{
    \ifthenelse{\isempty{#3}}{%
        \def\@lecture{Lecture #1}%
    }{%
        \def\@lecture{Lecture #1: #3}%
    }%
    \subsection*{\@lecture}
    \marginpar{\small\textsf{\mbox{#2}}}
}



% These are the fancy headers
\usepackage{fancyhdr}
\pagestyle{fancy}

% LE: left even
% RO: right odd
% CE, CO: center even, center odd
% My name for when I print my lecture notes to use for an open book exam.
% \fancyhead[LE,RO]{Gilles Castel}

\fancyhead[RO,LE]{\@lecture} % Right odd,  Left even
\fancyhead[RE,LO]{}          % Right even, Left odd

\fancyfoot[RO,LE]{\thepage}  % Right odd,  Left even
\fancyfoot[RE,LO]{}          % Right even, Left odd
\fancyfoot[C]{\leftmark}     % Center

\makeatother

% Todonotes and inline notes in fancy boxes
\usepackage{todonotes}
\usepackage{tcolorbox}

% Fix some stuff
% %http://tex.stackexchange.com/questions/76273/multiple-pdfs-with-page-group-included-in-a-single-page-warning
\pdfsuppresswarningpagegroup=1


% name
\author{Asier García Ruiz}


\begin{document}
    \exercise{1}
    Let $a < b$ and let $f : [a, b ] \to \R$ be a differentiable function. Prove that

    \subexercise{a }
    If $f^{\prime} > 0$ for all $x \in (a, b)$, then $f$ is strictly monotone increasing.
    \begin{proof}
        Assume, for the sake of contradiction, that $\exists x, y \in [a, b]$ such that $x < y$ but $f(x) > f(y)$.
        Now, by the Mean Value Theorem, $\exists c \in [x,y]$ such that 
        \begin{equation*}
            f^{\prime}(c) = \frac{f(y) - f(x)}{y - x}.
        \end{equation*}
        We note, however, that the numerator is negative while the denominator is positive.
        Therefore, $\exists c \in [a, b]$ such that $f^{\prime}(c) < 0$ \contra.
    \end{proof}

    \subexercise{b}  
    If $f^{\prime}(x) = 0$ for all $x \in (a,b)$, then $f $ is a constant function.

    \begin{proof}
        Let $x \in (a,b)$.
        By the Mean Value Theorem we have that $\exists c \in [a, x]$ such that 
        \begin{equation*}
            f^{\prime}(c) = \frac{f(x) - f(a)}{x - a}.
        \end{equation*}
        However, $f^{\prime}(c) = 0$ for all $c \in (a,b)$. Therefore we have that $f(x) = f(a)$ for all $x$.
        Since $f(a)$ is just a constant, we have our result as required.
    \end{proof}

    \newpage
    \exercise{2}
    Suppose $f : (a,b) \to \R$ and $c \in (a,b)$. Prove that if $f^{\prime\prime}(c)$ exists, then
    % https://math.stackexchange.com/questions/221905/check-my-workings-show-that-lim-h-to0-fracfxh-2fxfx-hh2-fx
    % https://math.stackexchange.com/questions/210264/second-derivative-formula-derivation/210273#210273
    \begin{equation*}
        f^{\prime\prime}(c) = \lim_{h\to 0} \frac{f(c + h) + f(c - h) - 2f(c)}{h^{2}}
    \end{equation*}
    (Hint: use Cauchy Extended Mean Value Theorem)

    \begin{proof}
        We will prove this using L'Hopital's theorem. We apply the theorem on the limit and get that
        \begin{equation*}
            \lim_{h \to 0} \frac{f^{\prime}(c + h) + f^{\prime}(c - h)}{2h}.
        \end{equation*}
        Which is nothing but an alternate definition of $f^{\prime\prime}(c)$.
    \end{proof}

    \newpage
    \exercise{3}
    % https://math.stackexchange.com/questions/1560143/convergence-of-sum-n-1-inftyfn-iff-convergence-of-int-1-infty
    Suppose $f : [0, \infty) \to [0, \infty)$ is a bounded decreasing function.
    Prove that 
    \begin{equation*}
        \sum_{n=1}^{\infty}f(n) \text{ converges if and only if } \sup_{N \in \N} \int_{0}^{N} f < \infty.
    \end{equation*}
    (Hint: find a relationship between $\sum_{n=1}^{N} f(n), \sum_{n=1}^{N-1} f(n) \text{ and } \int_{0}^{N}f $)

    \begin{proof}
        We note that since $f$ is decreasing function we have that 
        \begin{equation}
            f(n + 1) < \int_{n}^{n+1} f < f(n).
        \end{equation}
        
        We also see that the integral $\sup_{N \in \N}\int_{1}^{N}$ exists only if the series
        \begin{equation*}
            \int_{1}^{2} f + \int_{2}^{3} f + \dots
        \end{equation*}
        converges.

        Now, from (1) we can actually get both direction of our if and only if statement.
        The left half of the inequality demonstrates that we can compare the series and integral
        as such 
        \begin{equation*}
            S_{n} = \sum_{n = 1}^{\infty} f(n + 1) < \sum_{n = 1}^{\infty} \int_{n}^{n+1} f = I.
        \end{equation*}
        Therefore, $S_{n}$ will converge if $\sup_{N \in N} I$ exists.

        On the other side we have that we can compare
        \begin{equation*}
            I^{\prime} = \sum_{n = 1}^{\infty} < \sum_{n=1}^{\infty} f(n) = S_{n}^{\prime}.
        \end{equation*}
        This proves that $\sup_{N \in \N}I^{\prime}$ will exist if $\sum_{n=1}^{\infty}f(n)$ converges.

    \end{proof}

    \newpage
    \exercise{4}
    (You can use the result of question 3.) Show that 
    \begin{equation*}
        \sum_{n=2}^{\infty} \frac{1}{n(\ln n)^{p}} \text{ converges if and only } p > 1.
    \end{equation*}

    \begin{proof}
        First, we find the integral $I = \int_{0}^{N} \frac{1}{n(\ln n)^{p}} dn$. We let $u = \ln n$ and see that
        \begin{align*}
            \int_{0}^{N} \frac{1}{n(\ln n)^{p}} &= \int_{0}^{\ln(N)} \frac{1}{nu^{p}}\frac{1}{n},\\ 
                                                &= \int_{1}^{\ln N}\frac{1}{u^{p}}, \\
                                                &= \left[\frac{u^{-p + 1}}{-p + 1}\right]_{1}^{\ln N},\\ 
                                                &=  \left[\frac{\ln^{-p + 1}(n)}{-p + 1}\right]_{0}^{N},\\ 
                                                &= \frac{1}{-p + 1}\left[\frac{1}{\ln^{p-1}(N)} - 1\right].
        \end{align*}
        Clearly we cannot have that $p = 1$. We can also see that $\sup_{N \in \N} I < \infty$ only 
        when $p > 1$.

        Now, having found this, we use the result proven in Exercise 3. to the series 
        $\sum_{2}^{\infty} \frac{1}{n(\ln n)^{p}}$ and conclude that it converges if and only if $p < 1$.
    \end{proof}

    \newpage
    \exercise{5}
    Prove that a continuous function $f : [a,b] \to \R $ is uniformly continuous.

    \begin{proof}
        Assume, for the sake of contradiction, that $f$ is not uniformly continuous. This implies that there exists
        some $\epsilon > 0$ such that for ever $\delta > 0$ there exists $x, y \in [a, b]$ such that 
        $\lvert x - y \rvert < \delta$ but $\lvert f(x) - f(y) \rvert > \epsilon$. We pick these $x, y \in [a, b]$ and 
    call them $x_{n}, y_{n}$, since $[a, b]$ is closed and bounded, then there are subsequences $\{x_{n_{k}}\} \{y_{n_{k}}\}$
        of $\{x_{n}\}, \{y_{n}\}$ such that $\{x_{n_{k}}\}$ converges to $c \in [a,b]$. We now observe that 
        \begin{equation*}
            \lvert y_{n_{k}} - c \rvert = \lvert y_{n_{k}} - x_{n_{k}} + x_{n_{k}} - c \rvert 
            < \rvert y_{n_{k}} - x_{n_{k}} \rvert + \lvert x_{n_{k}} - c \rvert.
        \end{equation*}
        Since both of the terms of the right go to zero, we get that $\{y_{n_{k}}\}$ also converges to $c$.

        Now we consider
        \begin{align*}
            \lvert f(x_{n_{k}}) - f(c) \rvert &= \lvert f(x_{n_{k}}) - f(y_{n_{k}}) + f(y_{n_{k}}) - f(c) \rvert, \\
            &\geq \lvert f(x_{n_{k}}) - f(y_{n_{k}}) \rvert - \lvert f(y_{n_{k}}) - f(c), \\
            &\geq \epsilon - \lvert f(y_{n_{k}}) - f(c) \rvert.
        \end{align*}
        Now, if $f$ was continuous at $c$ then we would have that $\lvert f(y_{n_{k}}) - f(c) \rvert = 0$,
        which would be a contradiction.
        Therefore, $f$ is not continuous \contra.
    \end{proof}
\end{document}
