\documentclass[twoside]{article}
% Some basic packages
\usepackage[utf8]{inputenc}
\usepackage[T1]{fontenc}
\usepackage{textcomp}
\usepackage{url}
\usepackage{graphicx}
\usepackage{float}
\usepackage{booktabs}
\usepackage{enumitem}

% Don't indent paragraphs.
\usepackage{parskip}

% Hide page number when page is empty
\usepackage{emptypage}
\usepackage{subcaption}
\usepackage{multicol}
\usepackage{xcolor}

% Math stuff
\usepackage{amsmath, amsfonts, mathtools, amsthm, amssymb}
% Non-ams math stuff
\usepackage{derivative, physics}
% Fancy script capitals
\usepackage{mathrsfs}
\usepackage{cancel}

% Bold math
\usepackage{bm}

% Cool command shortcuts
\newcommand\N{\ensuremath{\mathbb{N}}}
\newcommand\R{\ensuremath{\mathbb{R}}}
\newcommand\Z{\ensuremath{\mathbb{Z}}}
\renewcommand\O{\ensuremath{\emptyset}}
\newcommand\Q{\ensuremath{\mathbb{Q}}}
\newcommand\C{\ensuremath{\mathbb{C}}}

% Operators
\DeclareMathOperator*{\argmax}{arg\,max}
\DeclareMathOperator*{\argmin}{arg\,min}

% Easily typeset systems of equations (French package)
\usepackage{systeme}

%Make implies and impliedby shorter
\let\implies\Rightarrow
\let\impliedby\Leftarrow
\let\iff\Leftrightarrow
\let\epsilon\varepsilon
\let\phi\varphi

% Add \contra symbol to denote contradiction
\usepackage{stmaryrd} % for \lightning
\newcommand\contra{\scalebox{1.5}{$\lightning$}}

% horizontal rule
\newcommand\hr{
    \noindent\rule[0.5ex]{\linewidth}{0.5pt}
}

% For box around Definition, Theorem, \ldots
\usepackage{mdframed}
\mdfsetup{skipabove=1em,skipbelow=0em}
\theoremstyle{definition}
\newmdtheoremenv[nobreak=true]{definition}{Definition}
\newmdtheoremenv[nobreak=true]{lemma}{Lemma}
\newmdtheoremenv[nobreak=true]{proposition}{Proposition}
\newmdtheoremenv[nobreak=true]{theorem}{Theorem}
\newmdtheoremenv[nobreak=true]{corollary}{Corollary}
\newmdtheoremenv{conjecture}{Conjecture}
\newtheorem*{remark}{Remark}
\newtheorem*{problem}{Problem}
\newtheorem*{eg}{Example}
\newtheorem*{question}{Question}
\newtheorem*{intuition}{Intuition}
\newtheorem*{claim}{Claim}

% End example environments with a small diamond (just like proof
% environments end with a small square)
\usepackage{etoolbox}
\AtEndEnvironment{eg}{\null\hfill$\diamond$}%

% Fix some spacing
% http://tex.stackexchange.com/questions/22119/how-can-i-change-the-spacing-before-theorems-with-amsthm
\makeatletter
\def\thm@space@setup{%
  \thm@preskip=\parskip \thm@postskip=0pt
}

% Exercise 
% Usage:
% \oefening{5}
% \suboefening{1}
% \suboefening{2}
% \suboefening{3}
% gives
% Oefening 5
%   Oefening 5.1
%   Oefening 5.2
%   Oefening 5.3
\newcommand{\exercise}[1]{%
    \def\@exercise{#1}%
    \subsection*{Exercise #1}
}

\newcommand{\subexercise}[1]{%
    \subsubsection*{Exercise \@exercise.#1}
}

% \lecture starts a new lecture (les in dutch)
%
% Usage:
% \lecture{1}{di 12 feb 2019 16:00}{Inleiding}
%
% This adds a section heading with the number / title of the lecture and a
% margin paragraph with the date.

% I use \dateparts here to hide the year (2019). This way, I can easily parse
% the date of each lecture unambiguously while still having a human-friendly
% short format printed to the pdf.

\usepackage{xifthen}
\def\testdateparts#1{\dateparts#1\relax}
\def\dateparts#1 #2 #3 #4 #5\relax{
    \marginpar{\small\textsf{\mbox{#1 #2 #3 #5}}}
}

\def\@lecture{}%
\newcommand{\lecture}[3]{
    \ifthenelse{\isempty{#3}}{%
        \def\@lecture{Lecture #1}%
    }{%
        \def\@lecture{Lecture #1: #3}%
    }%
    \subsection*{\@lecture}
    \marginpar{\small\textsf{\mbox{#2}}}
}


% For page size and geometry
\usepackage{geometry}

% These are the fancy headers
\usepackage{fancyhdr}
\pagestyle{fancy}

% LE: left even
% RO: right odd
% CE, CO: center even, center odd
% My name for when I print my lecture notes to use for an open book exam.
% \fancyhead[LE,RO]{Gilles Castel}

\fancyhead[RO,LE]{\@lecture} % Right odd,  Left even
\fancyhead[RE,LO]{}          % Right even, Left odd

\fancyfoot[RO,LE]{\thepage}  % Right odd,  Left even
\fancyfoot[RE,LO]{}          % Right even, Left odd
\fancyfoot[C]{\leftmark}     % Center

\makeatother

% Todonotes and inline notes in fancy boxes
\usepackage{todonotes}
\usepackage{tcolorbox}

% Fix some stuff
% %http://tex.stackexchange.com/questions/76273/multiple-pdfs-with-page-group-included-in-a-single-page-warning
\pdfsuppresswarningpagegroup=1


% name
\author{Asier García Ruiz}


\begin{document}
    \exercise{1}
    \begin{proof}
        Let 
        \begin{equation*}
            g(x) = f(x) - \sum_{k = 0}^{n - 1} \frac{f^{(k)}(a)}{k!}(x - a)^{k}
        \end{equation*}
        and 
        \begin{equation*}
            h(x) = (x - a)^{n}
        \end{equation*}
        We now note that
        \begin{equation*}
            g^{(i)}(a) = f^{(i)}(a) = 0 \qquad  \forall i \in [0, n - 1].
        \end{equation*}
        
        Now, we apply Cauchy's mean value theorem and get
        \begin{equation*}
            \frac{g^{\prime}(c_{1})}{h^{\prime}(c_{2})} = \frac{g(b) - g(a)}{h(b) - h(a)} = \frac{g(b)}{h(b)},
        \end{equation*}
        where $c_{1} \in (a,b)$.

        Now, we apply this again on the interval $(a, c_{1})$ to get
        \begin{equation*}
            \frac{g^{\prime\prime}(c_{2})}{h^{\prime\prime}(c_{2})} 
            = \frac{g^{\prime}(c_{1}) - g^{\prime}(a)}{h^{\prime}(c_{1}) - h^{\prime}(a)} 
            = \frac{g^{\prime}(c_{1})}{h^{\prime}(c_{1})},
        \end{equation*}
        for some $c_{2} \in (a, c_{1})$.

        Applying this $n$ times we get a progression $a < c_{n} < \dots < c_{1} < b$ and
        \begin{equation*}
            \frac{g(b)}{h(b)} = \frac{g^{(n)}(c)}{h^{(n)}(c)},
        \end{equation*}
        for some $c = c_{n} \in (a,b)$.

        Replacing the substituted values we get that
        \begin{equation*}
            \frac{f(b) - \sum_{k = 0}^{n - 1} \frac{f^{(k)}(a)}{k!}(b - a)^{k}}{(b - a)^{n}} = \frac{f^{(n)}(c)}{n!}.
        \end{equation*}
        Which, upon rearragement, yields
        \begin{equation*}
            f(b) = \sum_{k = 0}^{n - 1} \frac{f^{(k)}(a)}{k!}(b - a)^{k} + \frac{f^{(n)}(c)}{n!}(b-a)^{n}
        \end{equation*}
        as desired.
    \end{proof}
    \exercise{2}
    \begin{proof}
        We have that $f$ is Riemann integrable, therefore $\exists \bm{P}, \epsilon > 0$ such that
        \begin{equation*}
            \mathcal{U}(f, \bm{P}) - \mathcal{L}(f, \bm{P}) < \epsilon_{1}
        \end{equation*}

        Now consider the upper sum of $\alpha f$, 
        \begin{align*}
            \mathcal{U}(\alpha f, \bm{P}) &= \sum_{I \in \bm{P}} \sup_{I} (\alpha f) * |I|,\\
                                          &= \alpha \sum_{I \in \bm{P}} \sup_{I} f * |I|,\\
                                          &= \alpha \mathcal{U}(f, \bm{P}).
        \end{align*}
        Since $\inf_{I} (\alpha f) = \alpha \inf_{I} (f) \forall I \in \bm{P}$ we get the similar result
        \begin{equation*}
            \mathcal{L}(\alpha f, \bm{P}) = \alpha \mathcal{L}(f, \bm{P}).
        \end{equation*}

        Now, we let $\epsilon_{2} = \frac{\epsilon_{1}}{\alpha} > 0$ if $\alpha > 0$ or 
        $\epsilon_{2} = -\frac{\epsilon_{1}}{\alpha} > 0$ if $\alpha < 0$. Therefore, now we have that
        \begin{equation*}
            \mathcal{U}(\alpha f, \bm{P}) - \mathcal{L}(\alpha f, \bm{P}) 
        = \alpha \left[\mathcal{U}(f, \bm{P}) - \mathcal{L}(f, \bm{P})\right] < \alpha \epsilon_{1}
        = \epsilon_{2}
        \end{equation*}
        Therefore we have that $\alpha f$ is Riemann integrable.

        Now, to show the value of $\int_{a}^{b} \alpha f$ we consider
        \begin{equation*}
            \int_{a}^{b} \alpha f = \sup_{\bm{P}} \mathcal{L}(\alpha f, \bm{P})
            = \sup_{\bm{P}} \alpha \mathcal{L}(\alpha f, \bm{P})
            = \alpha \sup_{\bm{P}} \mathcal{L}(\alpha f, \bm{P})
            = \alpha \int_{a}^{b} f.
        \end{equation*}
        Concluding our proof.
    \end{proof}

    \exercise{3}
    \begin{proof}
        Since $f$ is Riemann integrable, we have that 
        \begin{align*}
            \int_{a}^{b} f &= \sup_{\bm{P}} \mathcal{L}(f, \bm{P}),\\ 
                           &= \sup_{\bm{P}}\sum_{I \in \bm{P}} \inf_{I} f * |I|.
                           \intertext{Now, since $f(x) \geq 0 \ \forall x \in [a,b]$ and $|I| > 0 \ \forall I$}
                           &\geq 0.
        \end{align*}
    \end{proof}

    \exercise{4}
    \subexercise{a}
    \begin{proof}
        We start by considering the lower sum $\mathcal{L}(f, \bm{P}) = \sum_{I \in \bm{P}} \inf_{I} f * |I|$.
        We observe that for any partition $\bm{P}$ there cannot be an interval consisting
        of a single point. Therefore, for a any $I \in \bm{P}$ we have that
        $\inf_{I} f = 1$. Thus, our lower sum can be resolved to
        \begin{equation*}
            \mathcal{L}(f, \bm{P}) = \mathcal{L}(f, \bm{P}) = \sum_{I \in \bm{P}} \inf_{I} f * |I|
            = \sum_{I \in \bm{P}} |I| = |[1, 3]| = 2,
        \end{equation*}
        as required.
    \end{proof}

    \subexercise{b}
    \begin{proof}
        Let $\epsilon > 0$ and consider a parition $\bm{U} = \{1, 1 + \frac{\epsilon}{n}, 3 - \frac{\epsilon}{n}, 3\}$.
        Now we calculate the upper sum,
        \begin{align*}
            \mathcal{U}(f, \bm{U}) &= \sum_{I \in \bm{U}} \sup_{I} f * |I|,\\
                                   &= 3\frac{\epsilon}{n} + 1(2 - 2\frac{\epsilon}{n}) + 5\frac{\epsilon}{n},\\ 
                                   &= 2 + 6\frac{\epsilon}{n},\\ 
                                   \intertext{picking any $n \geq 6$,}
                                   &\leq 2 + \epsilon.
        \end{align*}
    \end{proof}

    \exercise{5}
    \subexercise{a}
    \begin{proof}
        Suppose that $u = \sup_{A} f, v = \sup_{A} g$. Clearly we have that
        \begin{equation*}
            f(a) + g(a) \leq u + v \qquad \forall a \in A.
        \end{equation*}
        Now, let $w = \sup_{A} f + g$. By the definition of a supremum, this is the
        tightest upper bound for $f(a) + g(a) \ \forall a \in A$. 
        Therefore, we have that $w \leq u + v$ as needed.

        We will follow a similar logic for the infimum.
        Let $u = \inf_{A} f, v = \inf_{A} g$. Clearly we have that
        \begin{equation*}
            u + v \leq f(a) + g(a) \qquad  \forall a \in A.
        \end{equation*}
        Now, let $w = \inf_{A} f + g$. By the definition of an infimum, this is
        the tighest lower bound for $f(a) + g(a) \ \forall a \in A$.
        Therefore, we have that $u + v \leq w$ as needed.
    \end{proof}

    \subexercise{b}
    \begin{proof}
        Now we consider the upper sum of $f + g$ and see that
        \begin{align*}
            \mathcal{U}(f + g, \bm{P}) &= \sum_{I \in \bm{P}}\sup_{I} (f + g) * |I|,\\ 
                                       &\leq \sum_{I \in \bm{P}} (\sup_{I} f + \sup_{I} g)*|I|,\\ 
                                       &= \sum_{I \in \bm{P}} \sup_{A} f * |I| + \sum_{I \in \bm{P}} \sup_{A} g * |I|,\\ 
                                       &= \mathcal{U}(f, \bm{P}) + \mathcal{U}(g, \bm{P}),
        \end{align*}
        as needed.
    \end{proof}
\end{document}
