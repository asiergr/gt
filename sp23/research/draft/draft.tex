\documentclass[twoside]{article}

% Some basic packages
\usepackage[utf8]{inputenc}
\usepackage[T1]{fontenc}
\usepackage{textcomp}
\usepackage{url}
\usepackage{graphicx}
\usepackage{float}
\usepackage{booktabs}
\usepackage{enumitem}

% Don't indent paragraphs.
\usepackage{parskip}

% Hide page number when page is empty
\usepackage{emptypage}
\usepackage{subcaption}
\usepackage{multicol}
\usepackage{xcolor}

% Math stuff
\usepackage{amsmath, amsfonts, mathtools, amsthm, amssymb}
% Non-ams math stuff
\usepackage{derivative, physics}
% Fancy script capitals
\usepackage{mathrsfs}
\usepackage{cancel}

% Bold math
\usepackage{bm}

% Cool command shortcuts
\newcommand\N{\ensuremath{\mathbb{N}}}
\newcommand\R{\ensuremath{\mathbb{R}}}
\newcommand\Z{\ensuremath{\mathbb{Z}}}
\renewcommand\O{\ensuremath{\emptyset}}
\newcommand\Q{\ensuremath{\mathbb{Q}}}
\newcommand\C{\ensuremath{\mathbb{C}}}

% Operators
\DeclareMathOperator*{\argmax}{arg\,max}
\DeclareMathOperator*{\argmin}{arg\,min}

% Easily typeset systems of equations (French package)
\usepackage{systeme}

%Make implies and impliedby shorter
\let\implies\Rightarrow
\let\impliedby\Leftarrow
\let\iff\Leftrightarrow
\let\epsilon\varepsilon
\let\phi\varphi

% Add \contra symbol to denote contradiction
\usepackage{stmaryrd} % for \lightning
\newcommand\contra{\scalebox{1.5}{$\lightning$}}

% horizontal rule
\newcommand\hr{
    \noindent\rule[0.5ex]{\linewidth}{0.5pt}
}

% For box around Definition, Theorem, \ldots
\usepackage{mdframed}
\mdfsetup{skipabove=1em,skipbelow=0em}
\theoremstyle{definition}
\newmdtheoremenv[nobreak=true]{definition}{Definition}
\newmdtheoremenv[nobreak=true]{lemma}{Lemma}
\newmdtheoremenv[nobreak=true]{proposition}{Proposition}
\newmdtheoremenv[nobreak=true]{theorem}{Theorem}
\newmdtheoremenv[nobreak=true]{corollary}{Corollary}
\newmdtheoremenv{conjecture}{Conjecture}
\newtheorem*{remark}{Remark}
\newtheorem*{problem}{Problem}
\newtheorem*{eg}{Example}
\newtheorem*{question}{Question}
\newtheorem*{intuition}{Intuition}
\newtheorem*{claim}{Claim}

% End example environments with a small diamond (just like proof
% environments end with a small square)
\usepackage{etoolbox}
\AtEndEnvironment{eg}{\null\hfill$\diamond$}%

% Fix some spacing
% http://tex.stackexchange.com/questions/22119/how-can-i-change-the-spacing-before-theorems-with-amsthm
\makeatletter
\def\thm@space@setup{%
  \thm@preskip=\parskip \thm@postskip=0pt
}

% Exercise 
% Usage:
% \oefening{5}
% \suboefening{1}
% \suboefening{2}
% \suboefening{3}
% gives
% Oefening 5
%   Oefening 5.1
%   Oefening 5.2
%   Oefening 5.3
\newcommand{\exercise}[1]{%
    \def\@exercise{#1}%
    \subsection*{Exercise #1}
}

\newcommand{\subexercise}[1]{%
    \subsubsection*{Exercise \@exercise.#1}
}

% \lecture starts a new lecture (les in dutch)
%
% Usage:
% \lecture{1}{di 12 feb 2019 16:00}{Inleiding}
%
% This adds a section heading with the number / title of the lecture and a
% margin paragraph with the date.

% I use \dateparts here to hide the year (2019). This way, I can easily parse
% the date of each lecture unambiguously while still having a human-friendly
% short format printed to the pdf.

\usepackage{xifthen}
\def\testdateparts#1{\dateparts#1\relax}
\def\dateparts#1 #2 #3 #4 #5\relax{
    \marginpar{\small\textsf{\mbox{#1 #2 #3 #5}}}
}

\def\@lecture{}%
\newcommand{\lecture}[3]{
    \ifthenelse{\isempty{#3}}{%
        \def\@lecture{Lecture #1}%
    }{%
        \def\@lecture{Lecture #1: #3}%
    }%
    \subsection*{\@lecture}
    \marginpar{\small\textsf{\mbox{#2}}}
}


% For page size and geometry
\usepackage{geometry}

% These are the fancy headers
\usepackage{fancyhdr}
\pagestyle{fancy}

% LE: left even
% RO: right odd
% CE, CO: center even, center odd
% My name for when I print my lecture notes to use for an open book exam.
% \fancyhead[LE,RO]{Gilles Castel}

\fancyhead[RO,LE]{\@lecture} % Right odd,  Left even
\fancyhead[RE,LO]{}          % Right even, Left odd

\fancyfoot[RO,LE]{\thepage}  % Right odd,  Left even
\fancyfoot[RE,LO]{}          % Right even, Left odd
\fancyfoot[C]{\leftmark}     % Center

\makeatother

% Todonotes and inline notes in fancy boxes
\usepackage{todonotes}
\usepackage{tcolorbox}

% Fix some stuff
% %http://tex.stackexchange.com/questions/76273/multiple-pdfs-with-page-group-included-in-a-single-page-warning
\pdfsuppresswarningpagegroup=1


% name
\author{Asier García Ruiz}


\begin{document}
\begin{definition}
	We define a shallow neural network with $N$ neurons to be a function $\mathcal{N} : \R^{d} \to \R$
	of the form
	\begin{equation*}
		\mathcal{N}_{\omega, c} = \sum_{n=1}^{N}c^{n} \sigma(\bm{w}_{n} \cdot [\bm{x}, 1]^{T}),
	\end{equation*}
	where $\sigma(\bm{w}_{n} \cdot [\bm{x}, 1]^{T})$ are the single neurons, $N$ is the width of the network,
	and $\bm{\omega} = (a_{n}, b_{n}) \in \R^{d + 1}$ denoted the nodes with inner
	weights $a_{n} \in \R^{n}$ and $b_{n} \in \R$. The numbers $c_{n} \in \R$ are called
	the outer weights.
\end{definition}

\begin{definition}
	We let $L(\cdot)$ denote the least squares loss function
	\begin{equation*}
		L(\mathcal{N}_{\omega, c} ; \bm{y}) = L_{K, x}(\mathcal{N}_{\omega, c} ; \bm{y})
		:= \frac{1}{2K}\sum_{k = 1}^{K} \lvert \mathcal{N}_{\omega, c}(x_{k}) - y_{k} \rvert^{2}.
	\end{equation*}
\end{definition}

\begin{lemma}\label{lemma:equality-of-problems}
	Let
	\begin{align*}
		I(w, c) & = L(\mathcal{N}_{\omega, c} ; \bm{y}) + \alpha \sum_{n = 1}^{N} \phi(\rvert c_{n} \lvert), \\
		J(w, c) & =  L(\mathcal{N}_{\omega, c} ; \bm{y})
		+ \alpha \sum_{n=1}^{N} \phi\left(\frac{\abs{c_{n}}^{2} + \lVert w_{n} \rVert^{2}}{2}\right).
	\end{align*}

	For any $(\hat{\omega}, \hat{c}) \in (\mathbb{S}^{d} \times \R)^{N}$, there exists a
	$(\bar{\omega}, \bar{c}) \in (\R^{d + 1} \times \R)^{N}$
	such that $I(\hat{\omega}, \hat{c}) = J(\bar{\omega}, \bar{c})$.
\end{lemma}

\begin{proof}
	Let $(\hat{\omega}_n, \hat{c}_n) \in \mathbb{S}_{d} \times \R^{N}$.
	Now, let $\bar{\omega}_n = \frac{\hat{\omega}_n}{\sqrt{|\hat{c}_n |}}$ and
	$\bar{c}_n = \hat{c}_n \cdot \sqrt{|\hat{c}_n|}$.

	Now consider
	\begin{align*}
		I(\hat{\omega}_n, \hat{c}_n) & = \lVert \mathcal{N}_{\hat{\omega}, \hat{c}} - \bm{y} \rVert_{2}
		+ \alpha \sum_{n = 1}^{N} \phi(|\hat{c}_n|).                                                    \\
	\end{align*}
	Note that due to the positive homogeneity of ReLU we then have that
	$\lVert \mathcal{N}_{\hat{\omega}, \hat{c}} - \bm{y} \rVert_{2} = \lVert \mathcal{N}_{\bar{\omega}, \bar{c}} - \bm{y} \rVert_{2} $,
	and also that $|\hat{c}_n | = \frac{|\bar{c}_n|^2 + \lVert \bar{\omega}_n \rVert^2}{2}$.
	Therefore, $I(\hat{\omega}, \hat{c}) = J(\bar{\omega}, \bar{c})$.

\end{proof}

\begin{theorem}
	Let $L$ be a loss function and $D = (\bm{x}_{n}, \bm{y}_{n})_{n=1}^{d}$ be training examples. Then
	\begin{equation}
		\min_{N \in \N, \{c_{n}\} \in \R^{N}\, \{\omega_{n} = (a_{n}, b_{n})\} \in (\mathbb{S}^{d})^{N}}
		I(w, c),
	\end{equation}
	where $\mathbb{S}^{d} = \{(a,b) \in \R^{d + 1} : \lVert {(a,b)}^{2} \rVert_{2} = 1\}$ is the unit sphere in $\R^{d + 1}$.

	Is equivalent to
	\begin{equation}
		\min_{N \in \N, \{(a_{n}, b_{n}, c_{n})\}\in (\R^{d} \times \R \times \R)^{N}}
		J(w, c),
	\end{equation}
	By equivalence we mean that
	if $(\hat{w}, \hat{c})$ is a solution to (1), then, there exists
	$(\bar{w}, \bar{c})$ that is also a solution to (2), and viceversa.
\end{theorem}

\begin{proof}
	Let $g: (\mathbb{S}^{d} \times \R) \to (\R^{d + 1}\times \R)$ and
	$h: (\R^{d + 1}\times \R) \to (\mathbb{S}^{d} \times \R)$.

	We note that due to the increasing nature of $\phi$,
	and the inequality between arithmetic and geometric sums,
	\begin{equation}\label{eq:problem-inequality}
		I(\bm{w}, \bm{c}) \leq J(\bm{w}, \bm{c})
		\quad \forall (\bm{w}, \bm{c}) \in (\R^{d + 1}\times \R).
	\end{equation}

    NOTE: problem part

	$(1) \Rightarrow (2)$.
	Let $(\hat{w}^{*}, \hat{c}^{*})$ minimise $I$. By Eq. (\ref{eq:problem-inequality}) and
	Lemma \ref{lemma:equality-of-problems} we get
	\begin{equation*}
		I(\hat{w}^{*}, \hat{c}^{*}) \leq J(\hat{w}^{*}, \hat{c}^{*})
		\stackrel{?}{=} I(h(\hat{w}, \hat{c})) \leq I(\hat{w}^{*}, \hat{c}^{*}).
	\end{equation*}
	Therefore, $I(\hat{w}^{*}, \hat{c}^{*}) = J(g(\hat{w}^{*}, \hat{c}^{*}))$.

    NOTE: end of problem part


	$(2) \Rightarrow (1)$
	Let $(\bar{\bm{w}}^{*}, \bar{\bm{c}}^{*})$ minimise $J$.
	Now, by Eq. (\ref{eq:problem-inequality}) and Lemma \ref{lemma:equality-of-problems} we have that
	\begin{equation}\label{eq:J-equals-I}
		J(\bar{\bm{w}}^{*}, \bar{\bm{c}}^{*}) \geq I(\bar{\bm{w}}^{*}, \bar{\bm{c}}^{*})
		= J(g(\bar{\bm{w}}^{*}, \bar{\bm{c}}^{*})) \geq J(\bar{\bm{w}}^{*}, \bar{\bm{c}}^{*}).
	\end{equation}
	Therefore, $J(\bar{\bm{w}}^{*}, \bar{\bm{c}}^{*}) = I(\bar{\bm{w}}^{*}, \bar{\bm{c}}^{*})$.

	\begin{claim}
		$(\bar{\bm{w}}^{*}, \bar{\bm{c}}^{*})$ minimises $I$.

		\begin{proof}
			Assume, for the sake of contradiction, that it does not. this implies
			$\exists \hat{\bm{w}} \in (\mathbb{S}^{d} \times \R)$ such that
			\begin{equation*}
				I(\hat{\bm{w}}, \hat{\bm{c}}) < I(\bar{\bm{w}}^{*}, \bar{\bm{c}}^{*}).
			\end{equation*}
			However, from Lemma \ref{lemma:equality-of-problems} and Eq. (\ref{eq:J-equals-I})
			we get that this implies
			\begin{equation*}
				J(g(\hat{\bm{w}}^{*}, \hat{\bm{c}}^{*})) < J(\bar{\bm{w}}^{*}, \bar{\bm{c}}^{*}),
			\end{equation*}
			which is a contradiction as $(\bar{\bm{w}}^{*}, \bar{\bm{c}}^{*})$ minimises $J$.
		\end{proof}
	\end{claim}
	Therefore, we have that if $(\bar{\bm{w}}^{*}, \bar{\bm{c}}^{*})$ minimises $J$, then,
    it is also minimises $I$.
\end{proof}

\end{document}
