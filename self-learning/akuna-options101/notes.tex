\documentclass{article}
\usepackage[utf8]{inputenc}
\usepackage{amsmath}
\usepackage{amssymb}
\usepackage{amsfonts}
\usepackage{amsthm}
\usepackage{parskip}
\usepackage{graphicx}

\graphicspath{ {./images/} }

\title{Akuna Capital Options 101 Notes}
\author{Asier Garcia Ruiz}

\begin{document}
\maketitle

\section{Basics}
\subsection{Terminology}
\textbf{Bid}: The (highest) price for which someone is willing to \textit{buy} something.

\textbf{Offer}: The (lowest) price at which someone is willing to \textit{sell} something.

\textbf{Size}: the number of contracts that one is willing to trade at a given price.

\textbf{Make a market}: to provide a bid and ask price and a quantity/size.

\textbf{Spread}: $|P_{bid} - P_{ask}|$

\textbf{Hedge}: a trade or investment to reduce the risks of another transaction.

\textbf{Index}: an instrument that tracks the performance of a market.
Generally, an index will track the performance of many stocks.
E.g. Nasdaq, S\&P 500, and Down Jones Industrial Average.

\textbf{ETF}: marketable security that tracks a stock index, commodity, or other
basket of assets. Behaves and trades very much like a stock.

\textbf{Commodity}: a raw material (oil, gold) or agricultural product 
(soybeans, corn) that can be bought and sold, normally at one prevailing price

\textbf{Trader}: can define any active market participant.

\textbf{Market Maker}: a specific trader willing to buy or sell an asset at a specific
price at all times. Market makers constantly buy and sell related securities,
with their primary responsibilities being to collect edge and manage risk.

\textbf{Local}: General term for market makers. From pit trading days, as locals 
stood in the pit every day.

\textbf{Broker}: a person or company that acts as an intermediary between buyers 
and sellers

\textbf{Paper}: the interested parties trading against market makers. Term
comes from pit trading days as customer orders came via paper.

\textbf{Hedge Fund/Institution/Bank}: Financial institutions that are active
market participants. These groups are generally the largest paper customers.

\textbf{Retail Client}: Smaller “paper” customers. For example, an individual
trading at home.


\subsection{Options Specific Terms:}

\textbf{Fill}: Completion of a trade. If a market maker trades on their bid or offer,
the market maker may claim he/she “got filled”.

\textbf{Tick size (tick increment)}: the increment between one price level 
and the next smallest price level. Different products have different tick
sizes. E.g, tick increment in USD is 0.01.

\textbf{Queue Priority}: for markets that are determined “price-time”,
if multiple orders are entered for the same price, the participant
who entered his/her order or quote first, will trade first.
This person is said to have queue priority.

\textbf{Settlement time}: the specific time of days options expire,
and futures “settle” for the day. Used to calculate daily
P\&L and mark to market.

\textbf{All-or-None}: an order type that must be executed in its entirety,
or not executed at all.

\textbf{Immediate or Cancel (IOC)}: a type of order that requires all or part
of the order to be executed immediately. Unfilled parts of order are cancelled
sometimes referred to as \textbf{Fill and Kill (FAK)} orders.

\textbf{Good for Day (GFD) order}: a type of order that will remain active until
executed (in part or full) or until the end of the trading day.
It is then cancelled.

\textbf{Good-till-cancelled (GTC)}: a type of order that will remain active until
completed or cancelled by the entering party.

\textbf{Fill or Kill (FOK)}: an order type that must be executed immediately
in its entirety, otherwise the order is cancelled; often with floor trading
market makers have a few seconds to decide to make a trade and can also do a
partial order. Sometimes brokers will use this interchangeably with Fill and Kill
and will fill partial FOK orders.

\textbf{OCO (one cancels the other)}: when one order is executed, the other order
is automatically cancelled. Used to protect from too much exposure in one direction.

\textbf{Contract Size}: Multiplier attached to an option/future. Options
on stock generally have a multiplier of 100 shares. Options on futures
have a multiplier of 1 future. The multiplier on options on a future and the
multiplier on the future can vary.

\textbf{Vol bid, catching a bid, ripping/exploding}: Variety of terms for vol
going up.

\textbf{Vol offered, vol smashed/smoked}: Variety of terms for vol going down.

\textbf{Teenie}: lowest priced options. Generally traded for movement risk purposes.

\textbf{Theoretical Value (Theo)}: based on all inputs, the current value a market
maker believes an option is worth.

\textbf{Sheets (or fair value)}: same as above, but generally when referring to
where something traded.

\textbf{Liquidity}: how easy/hard it is to trade close to fair value.
Generally determined by the number on contracts on the bid/offer, and the width
of the market.

\subsection{How Market Makers Profit}
\begin{itemize}
    \item Calculate a theo \textbf{disseminate} (send out) a bid slightly below and an offer
slightly above.
    \item How to sell something you don't own? You are selling a promise to deliver
the underlying. Some possible outcomes:
\begin{description}
    \item[Stock Option:] deliver/buy underlying shares.
    \item[Cash Settled Option:] debit cash from account and pay buyer\\
    $P_{transaction} - P_{settlement}$. Done automatically by clearing houses.
    \item[Future Settled Options:] Deliver a future. 
\end{description}
\end{itemize}

\textbf{An option on a future:}
\begin{enumerate}
    \item Option contract created.
    \item Future settled.
    \begin{enumerate}
        \item Buyer exercises \textrightarrow\ New future contract created.
        \item No exercise \textrightarrow\ Option expires worthless \& end of trade.
    \end{enumerate}
\end{enumerate}

\subsubsection{Marketplaces}
\begin{itemize}
    \item Akuna trades on electonic exchanges ("on the screens") and open outcry ("trading pits").
    \item They have servers in the centers (\textbf{co-location})to save nanoseconds when
sending/amending quotes to the matching-engine.
    \item \textbf{Over the Counter (OTC)} trades are from one party to another,
    "off-floor". More \textit{counterparty risk}.
    \item On an exchange, all entity trading with, clearing company, and exchange
    would have to default.
    \item \textbf{Singly listed option} only trades on one exchange
    (e.g. Live Cattle only on CME).
    \item AAPL is listed on 12+ exchanges.
    \item Options on futures regulated by Commodity Futures Trading Company (CFTC).
    \item Options on equities by Securities and Exchange Commission (SEC).
    \item Options on futures traded mainly on CME Group (metals, energy, grains,
    treasuries, currencies, SP500, livestock) and ICE (Oil, sugar, coffee,
    cocoa, OJ, Russel 2000 Idx, USD Idx). Options on indices on several exchanges.
    Only CBOE trades SPX and VIX options (licensing). 
\end{itemize}

\subsubsection{Settlement Types}
\begin{itemize}
    \item Most common are stock/cash/futures.
    \item Futures either cash (E-mini S\&P, Lean Hogs, etc) or physical delivery
    (e.g. corn)
    \item $<1\%$ result in physical delivery.
\end{itemize}

\subsubsection{Stocks \& Futures}
\begin{itemize}
    \item The underlyings in options.
    \item Stock is a share in a company to raise capital.
    \item Future is a contract/agreement to buy/sell commodity or asset at a 
    given price in the future. Buyer of future buys underlying and V/V for seller.
\end{itemize}

\subsection*{Options on Listed Futures}
\begin{itemize}
    \item Two types of options: \textit{calls} and \textit{puts}.
    \item \textbf{Calls/Puts:} The \textit{right but not obligation} to buy/sell
     underlying for a specified price (strike price) at or before expiration date.
\end{itemize}

\subsubsection{Contract Specifications}
\begin{itemize}
    \item Details always listed by exchange.
\end{itemize}

\subsubsection{Option, Edge, Multipliers \& Cash}
\begin{itemize}
    \item Tick increments \& multipliers are set by exchange.
    \item $PnL = points \times qty \times multiplier$.
    \item E.g. Each corn futures contract is for 5000 bushels.
\end{itemize}

\subsubsection{Edge vs Cash}
\begin{itemize}
    \item Sometimes we don't care about PnL, but rather edge in points/ticks
    against theo (usually for options).
    \item e.g. bought at 1.00 and sold at 1.04 "made 4 ticks of edge."
\end{itemize}

\section{}
\subsection{Payoff Diagrams}
\begin{itemize}
    \item \textit{A graph that shows profit of an option or option combinations at expiry
for different underlyings.}
\item 2+ options together is a \textbf{spread} or \textbf{combo}.
\end{itemize}

\begin{figure}[h]
    \includegraphics[scale=0.5]{2/payoff1}
    \centering
\end{figure}

\begin{itemize}
    \item Akuna doesn't think of options as payoff diagrams because they're not
    held to expiry.
    \item Also a lot of hedging using the underlying.
\end{itemize}

\subsection{Time Premium and Put-Call-Parity}
\begin{itemize}
    \item \textbf{Instrinsic Value} - amount of the option price based on
    difference between current underlying price and strike. Only applicable for
    ITM option.
    \item \textbf{In-the-Money (ITM)} - an option that has inherent/instrinsic
    value based on location of underlying compared to strike price.
    \item \textbf{Out-of-the-money (OTM)} - an option that has no instrinsic
    value.
    \item An option that isnt ITM is OTM.
    \item For a call $V_{intrinsic} = \max(0, P_{underlying} - P_{strike})$.
    \item For a put $V_{intrinsic} = \max(0, P_{strike} - P_{underlying})$.
    \item \textbf{Extrinsic Value/Time Premium} - Portion of option price 
    attributed to optionality of option contract itself. I.e. There is a chance
    that OTM becomes ITM.
    \item \textit{OTM options only have extrinsic value, ITM have instrinsic and
    extrinsic.}
\end{itemize}

\subsubsection{Forward Pricing}
\begin{itemize}
    \item \textbf{Forward Price}: price of underlying at a future time (option exp)
    accounts for:
    \begin{itemize}
        \item Current (spot) price.
        \item Interest Rate 
        \item Dividends (if applicable).
        \item Carrying costs.
    \end{itemize}
    \item Once this is established, what contributes to extrinsic value is:
    \begin{itemize}
        \item \textbf{Distance to ATM}: how far strike is from forward.
        \item \textbf{Time}: time to expiry (1yr is more expensive that 6mos).
        \item \textbf{Volatility}: Higher implied vol, higher option price.
    \end{itemize}
\end{itemize}

\subsubsection{Put Call Parity (PCP)}
\begin{itemize}
    \item Mathematical link between puts and calls at same strike.
    \item \textbf{The Value of an ITM option is equal to its instrinsic value
    plus the value of the corresponding OTM option.}
    \item Simplified Put-Call-Parity formula
    \begin{equation*}
        P_{call} - P_{put} = P_{underlying} - P_{strike}
    \end{equation*}
    \item This is true otherwise arbitrage exists.
    \item Being long a call and short a put at the same strike is isomorphic
    to being long the underlying.
    \item Taking into account interest rates we get
    \begin{equation*}
        P_{forward} - P_{strike} = (P_{call} - P_{put})e^{rt}.
    \end{equation*}
\end{itemize}

\subsection{Combos}
We will talk about \textit{buying} these, remember the seller has the same
payoff diagram but reflected on $x$ axis.

\textbf{Call Spread}: 
Let $K_2 < K_1$, then buy call at $K_1$ and sell at $K_2$. Payoff diagram:

\textbf{Put Spread}: 
Let $K_1 < K_2$, then sell put at $K_1$ and buy at $K_2$. Payoff diagram:

\textbf{Box}: 
Let $K_1 > K_2$. Buy a call spread (buy $K_1$, sell $K_2$)
and sell a put spread (sell $K_1$, buy $K_2$). Payoff diagram:

\textbf{Call/Put Fly}: 
A call and a put fly are identical (in payoff and greeks).
Max. loss is the fly price.
Max. gain is $K_2 - K_1 - P_{fly}$.
Breakevens at $K_1 + P_{fly}$ and $K_3 - P_{fly}$.
Similar to betting on a volitility distribution.
Let $K_1 < K_2 < K_3$. A fly for $K_1/K_2/K_3$ is buying one contracts at $K_1$,
selling two at $K_2$ and buying one at $K_3$.

\textbf{Straddle}: A straddle for $K$ is buying a call and a put at the same
strike price $K$. Payoff diagram:

\textbf{Strangle}: Let $K_1 < K_2$. A strangle consists on buying a lower strike
put (at $K_1$) and a higher strike call (at $K_2$). Same directional/vol bet as
straddle but less premium. Payoff diagram:

\textbf{Ratio 1x2s}: Either a 1x2 call spread (buy one sell two) or a 1x2 put spread.

\textbf{Risk Reversal}: Let $K_1 < K_2$. Buy a put at $K_1$ and sell a call at
$K_2$. Done to protect underlying position.

\textbf{Reversal/Conversion}: Buy call, sell put at $K$ \textit{and} sell
underlying is \textit{reversing out of underlying}. Sell call, buy put at $K$ and buy underlying
is \textit{converting to underlying}.

\subsection{Option Limits/Boundaries}
Things that always hold (for sanity checks and spotting opportunities)
\begin{itemize}
    \item ITM call/put should never be worth less that instrinsic value.
    \item Option can't be worth less than zero.
    \item Call spread is never worth more than difference of strikes.
    \item Symmetrical fly is never worth less than zero.
    \item Same strike calendar (exp. into same ul) is never worth less than
    zero.
    \item Put-call parity equation always holds.
\end{itemize}

\section{Greeks}
\subsection{Delta}
\textbf{Delta}: The number of underlying contracts to establish a neutral hedge
under currecnt market conditions using the current theo value of the option.

\begin{itemize}
    \item For a call $\Delta \in [0, 1.00]$ but expressed in $[0,100]$ (drop the comma).
    e.g. 47 delta means for every 100 options we trade 47 uls as a hedge.
    \item For a put $\Delta \in [-1.00, 0]$.
    \item Common definitions for $\Delta$:
    \begin{itemize}
        \item Sensitivity of option price to change in price of underlying.
        \item Hedge ratio: the \# of ul contracts needed for a neutral hedge.
        \item The probability that an option will expire ITM (call). For a put
        it'd be $|\Delta|$ for expiring ITM. This def. is not rigurous but
        rather to build intuition.
        \item Calls have positive $\Delta$, when price of ul increases to does
        value of the call.
        \item Au contraire, puts have negative delta; price of ul increases
        value of put decreases.
        \item For small changes in ul price, we can use the previous delta value
        to estimate the new option price (finite diff. approx.)
        \item  For calls, If implied vol. or time to expiry up then delta
        approaches
         >0.50. For decr. then delta approaches 0. For puts, vice versa.
        \item Increase in time $\cong$ increase in implied vol. Similar for
        decrease in time and decrease in implied vol.
        \item For the same strike $K$, $\Delta_{call} + |\Delta_{put}| = 1$.
        \item \textbf{Covered Trade}: also tied trade is one where the hedge is
        already included with the option as a package. Both parties get rid of
        directionality of the bet.
    \end{itemize}
\end{itemize}

\subsection{Gamma}
\textbf{Gamma}: The rate of change of the $\Delta$ w.r.t. change of underlying
price.

\begin{equation}
    \Gamma = \frac{d\Delta}{dP_{ul}}.
\end{equation}

\begin{itemize}
    \item All options have positive gamma. Long positions get longer deltas and
    short get shorter deltas.
    \item Gamma peaks at ATM option (approx. normal distr) with higher peak when
    vol. is low.
    \item Gamma scalping and maintaining a delta neutral portfolio:
    \begin{figure}[h]
        \includegraphics[scale=0.4]{./3/gamma-scalp}
        \centering
    \end{figure}
    \item Note that delta changes (due to gamma) so we recalculate that first
    before calculating change in option price due to delta.
    \item Note that if we were short these options we'd lock in a loss.
    \item Why not always long gamma? Because of theta! (next chapter)
    \item \textbf{Cash Gamma}: amount by which cash delta changes if ul moves 1\%.
    \begin{equation*}
        Deltas \ per \ 1\% = \frac{Cash \ gamma}{Cash \ delta}, \ Cash \ delta = P_{future}*mult*1
    \end{equation*}
    \newpage
    \item Cool graph:

    \begin{figure}[h]
        \includegraphics[scale=0.4]{./3/gamma-surface}
        \centering
    \end{figure}
\end{itemize}

\subsection{Theta}
\textbf{Theta}: How much value an option loses daily. We call this loss of value
\textit{decay}.

\begin{itemize}
    \item A position is \textit{paying} (buy a call) or \textit{receiving} (buy
    a put) each day
    depending on theta.
    \item Whether the theta compensates for gamma determines whether gamma
    hedging works.
    \item Relationship of theta and gamma. (s.d. = stdev):
    \begin{equation*}
        \Theta_{fair} = \frac{\Gamma_{cash}*(s.d \% \ move)^2}{200}.
    \end{equation*}
    \newpage
    \item Cool graph:
    \begin{figure}[h]
        \includegraphics[scale=0.4]{./3/theta-surface}
        \centering
    \end{figure}
\end{itemize}

\subsection{Volatility}
\textbf{Volatility}: A measure of the deviation of an underlying's annual price
movement. The degree of variation of a trading price series over time as
measured by the standard deviation of logarithmic returns.

\begin{figure}[h]
    \includegraphics[scale=0.4]{./3/lognormal}
    \centering
\end{figure}

\begin{itemize}
    \item Different types of vol.s used:
    \begin{itemize}
        \item \textbf{Historical/Realised volatility}: a measure of how much the
        ul has moved in the past. Easy to do. Close-to-close prices used (issue
        when wild movement thru day to end at same price as start). Most widely
        used for historical datasets.
        \item \textbf{Implied volatility}: Expected/predicted future vol of ul.
        An \textit{input} into pricing model and used to derive option theos.
        Each individual can control option theos by changin implied vol in
        model.
        \item \textbf{Forward volatility}: expected avg. vol. between expiration
        date of two options with successive maturities.
    \end{itemize}
\end{itemize}

\subsubsection{Volatility as a Measure of Movement}
\begin{itemize}
    \item Vol. numbers used in options pricing are a measure of stdev of a
    product's future price move. 252 is the \# of trading days in a yr.
    \begin{equation*}
        expected \ 1 \ s.d. \ move \ over \ t \ days = \sigma \sqrt{\frac{t}{252}}.
    \end{equation*}
    \item Not the same as saying for any given day expected move is as above.
    The expected daily move formula is
    \begin{equation*}
        Expected \ Daily \ Move = 0.8\sigma \sqrt{\frac{t}{252}}.
    \end{equation*}
\end{itemize}

Here, the proof of the equation (yay math!)
\begin{proof}
    Let $X$ be the random variable representing the daily move. We let 
    \begin{equation*}
        X \sim N\left(0, \frac{\sigma^2}{256}\right)
    \end{equation*}
    where $\sigma$ is the volatility. Let the daily volatility be
    $\sigma_d = \sigma / 16$. We want to find $E[|X|]$, we write
    \begin{align*}
        E[|X|] &= \int_{-\infty}^\infty |x| \frac{1}{\sqrt{2\pi}\sigma_d}\exp\left(-\frac{x^2}{2\sigma_d^2}\right) \ dx, \\
        &= \sqrt[]{\frac{2}{\pi}}\int_0^\infty \frac{x}{\sigma_d}\exp\left(-\frac{x^2}{2\sigma_d^2}\right) \ dx, \\
        &= \sigma_d\sqrt{\frac{2}{\pi}}\left[\exp\left(-\frac{x^2}{2\sigma_d^2}\right)\right]^\infty_0, \\
        &= \frac{\sigma}{16}\frac{2}{\pi}
    \end{align*}
    For $\sigma = 16$ we get the expected daily move of 0.8\%.
\end{proof}

\subsubsection{Historical Volatility Calculation Example}
\begin{itemize}
    \item Assuming calculation from daily closing we calculate log returns (not
    absolute) and assume the mean of the returns is always zero. The formula is 
    \begin{equation*}
        vol = \sqrt{\frac{252}{n}\sum_{i=0}^n\left(\log\frac{S_i}{S_{i-1}}\right)^2}
    \end{equation*}
    \item TLDR: Add daily variances, take average, multiply by 252 to annualize.
    Then tada! we have out vol.
\end{itemize}

\subsection{Vega}
\textbf{Vega}: The option value price change per 1-point change in implied
volatility.
\begin{itemize}
    \item Most talked about greek, options traders are vega traders.
    \item Depends on time to expiry, distance from ATM, implied vol.
    \item Closer to ATM is higher vega.
    \item Options have positive vega. Buyer makes money when implied vol. goes
    up and lose if it goes down.
    \item The vega of a portfolio is simply $\nu = num. \ options * mult *
    option \ vega$
\end{itemize}

\subsection{Rho \& Boxes}
\textbf{Rho}: The sensitivity of option price to change in interest rates. 
The derivative of option value w.r.t. interest rates.
\begin{itemize}
    \item Call and put on same line can have different $\rho$.
    \item Expressed as change in option price for 1 point (1\%) change in
    interest rates.
    \item Usually managed on portfolios, not per option basis.
    \item individual investors don't manage it much, but MMs do. They express as
    PnL per 1\% change in interest rates.
\end{itemize}
\newpage
\subsubsection{Interest rate curves}
\begin{itemize}
    \item No single interest rate used, used interest rate curve as this one:
    \begin{figure}[h]
        \includegraphics[scale=0.4]{./3/ircurve}
        \centering
    \end{figure}
    \item Influenced by the Fed and the rates from the Clearing House.
    \item Shape can be changed by shifting inputs.
\end{itemize}

\subsubsection{Boxes}
\begin{itemize}
    \item Capturing interest rates thru options. E.g. for exp 4 mos away
    700-1200 box combo is (+700c-700p+1200p-1200c). No matter what the box is
    worth 500.00 at expiry (difference in strike prices). Model shows box is
    worth 499.20 because of cost in tying up capital for 4 mos to purchase asset
    today that will be worth 500.00 in 4 mos.
    \item Discount the final price using interest rate and simple Net Present
    Value equation. Represents the value in interest one could earn by putting
    capital in bonds that mature in 4 mos.
    \item The model gives the "fair value" of 499.20
\end{itemize}

\end{document}