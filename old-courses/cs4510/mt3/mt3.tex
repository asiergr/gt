\documentclass{article}

\usepackage[utf8]{inputenc}
\usepackage{enumitem}
\usepackage{latexsym}
\usepackage{amsfonts}
\usepackage{amsmath,amssymb,amsthm}
\usepackage{amsfonts}
\usepackage{parskip}
\usepackage{listings}
\usepackage{braket}

\usepackage{tikz}
\usetikzlibrary{arrows, automata, bending, positioning}

\newcommand{\Z}{\mathbb{Z}}
\renewcommand{\epsilon}{\varepsilon}

\newenvironment{question}[2]
{
    {\large \textbf{Question #1.}}\\
    #2\\\\
}{\newpage}

\title{Midterm 3}
\author{Asier Garcia Ruiz}

\begin{document}
\maketitle

\begin{question}
    {0}{Honor Code}
    I commit to uphold the ideals of honor and integrity by refusing to betray the
    trust bestowed upon me as a member of the Georgia Tech community.

    Signature: Asier Garcia Ruiz
\end{question}

\begin{question}
    {1a}
    {True or False: For any single tape Turing machine $M$, there is a
        single tape Turing machine $M'$ such that $L(M) = L(M')$ and for all inputs $x$, if
        $M$ halts on $x$, then $M'$ halts on $x$, with $x$ written on the tape when it is
        finished (and nothing else). Defend your answer.}

    True. We will describe how $M'$ works. Firstly, we would have $M'$ add a \# to the input tape at the first blank. Then, it would proceed to copy
    over the input $x$ on the tape again. The new tape would look like $x\# x$. Now, $M'$ will do the exact same thing than $M$, only it will act exclusively
    on the input to the right of \#. If it tried to overwrite the \#, then everything will get shifted enough so that it
    does not happen. When $M'$ is done processing the input on the right side, if it accepts, then it will erase from the \# rightward (including the \#).
    This will leave only the original input $x$ on the tape.
\end{question}

\begin{question}
    {1b}
    {True or False: the number of \emph{true} arithmetical statements
        involving positive integers, $+, \times, (, )$ and $=$ is countable, i.e. $``(17+31)\times2=96"$. Defend your answer.}

    True. We are given the set of elements that can compose a true arithmetical statement. That is, the numbers 0-9 and the operators
    $+, \times, (, ), =$. We will map the four latter to the letters $A, B, C, D, E$, and will map the numbers 0-9 to themselves.
    Now, we will map all arithmetic expressions to these elements. For example, the expression $1 + 1 = 2$ would be mapped to 1B1E2.
    We now note that this is nothing but a hexadecimal number. Therefore, we are able to map all these hexadecimal numbers to their base 10
    counterparts. This now implies that we are anle to map each true arithmetic expression on the positive integers to a natural number.
    Therefore, both groups are the same size and thus the number of statements is countable.
\end{question}

\begin{question}
    {1c}
    {Let $P$ be some decidable property of graphs. That is,
        $$L_P = \{\braket{G} ~|~ G\text{ has property } P\}$$
        is decidable. Prove that the language
        $$\{\braket{G} ~|~ G \text{ does not have property }P \text{ but } G \text{ contains a subgraph with property }P\}$$ is decidable.}

    We know that $L_P$ is decideable, therefore there is a decider $D$ that can decide whether a graph $G$ has a property $P$. We will construct a new
    decider $D'$ as follows. First, we run the decider $D$ on $\braket{G}$, and if it accepts, then we will reject. This is because we are told that $G$ does
    not have the property $P$. Now, we will enumerate every possible subgraph of $G$, we will call this set $G_e$. For each $G' \in G_e$, we will run
    $D$ on $\braket{G'}$. If any of the $\braket{G'}$ is accepted, we will accept. Otherwise, if they are all rejected, we will reject.

    Clearly, we can form a new decider $D'$ for $L_{p}'$, so this language is decideable.
\end{question}

\begin{question}
    {2}
    {Show that $L_1 = \{\langle M\rangle ~|~ |L(M)| \geq 1000\}$ is Turing-recognizable
        by constructing a recognizer for it. Explain why the machine you constructed is not
        a decider.}

    Let $M$ be a multitape Turing Machine with two tapes $T_1$ and $T_2$. We will generate strings, and check which ones are in $L(M)$. We will now have
    $T_1$ store all the strings generated, this is to ensure we do not test the same string twice. We will have $T_2$ store the number of strings which we
    found are in $L(M)$, we do this to check whether the number of strings in the language is greater than or equal to 1000. If this is true, then we
    accept, otherwise we reject. In order to prevent us getting stuck on the same computation of whether a string is in the language, we will process
    them as follows. Let $s_1, s_2, \dots, s_m$ be the strings we are processing. We will process $s_1$ for one step, then
    $s_1, s_2$ for two timesteps, then $s_1, s_2, s_3$ for three timesteps. We will do this successively and stop processing a string once its belonging to
    the language has been determined. We have therefore constructed a recogniser.

    The reason this is not a decider is because it can loop infinitely. Consider the case where we cannot find 1000 string from the same language because
    there are not that many, then the recogniser keeps looping between generating the string and checking if the string has already been generated.
\end{question}

\begin{question}
    {3a}
    {}

    First, we will simulate a regular Turing machine using a write-twice (i.e., $k \geq 2)$ Turing machine. This new TM will simulate a single step of the
    original machine by copying the entire tape over to a fresh portion of the tape to the right-hand side of the currently used portion. This copying
    will operate character by character, and it will mark each character as copied once it does so. We observe that this sequence will alter each square
    twice; it writes the character for the first time and then marks it as copied. We record the position of the original TM tape head.
    When we copy cells at or adjacent to the marked position, the tape content is updated according to the rules of the original Turing machine.
\end{question}

\begin{question}
    {3b}
    {}
    Now, in order to do this with a write once machine, we will do the same as before, except that each cell of the previous tape is represented by two cells
    on the new tape. The first cell contains the original TMs tape symbol and the second is the mark used in the copying procedure. Because the input
    is not presented to the machine with two cells per symbol, the first time the tape is copied over, the copying marks are put directly over the input symbols.
\end{question}

\begin{question}
    {4}
    {Prove that $L_2 = \{(\langle M_1\rangle, \langle M_2\rangle)\ |\
            L(M_1) \cup L(M_2) = \Sigma^*\}$ is undecidable.}

    To prove this, we will reduce $ALL_{CFG}$, to $L_2$. As usual, we start by assuming that $L_2$ is decidable and decided by $D$. From here, we will
    construct $D'$ that decided $ALL_{CFG}$. The input is $\braket{C}$ where $C$ is the CFG we are checking the language for.

    Now, we will create $D'$, a new CFG. First, we create a new Turing machine $M_1$ that is equivalent to the CFG $C$, and a new Turing Machine $M_2$ such
    that $L(M') = \emptyset$. This is how $D'$ will run: we will run $D$ on the input $(\braket{M_1}, \braket{M_2})$. We have that
    $L(M_1) \cup L(M_2) = \Sigma^*$ and $L(M_2) = \emptyset$, so we are simply running $D$ on $\braket{M_1}$ where $L(M_1) = \Sigma^*$. Now, if
    $D$ rejects this implies that $L(M_1) \neq \Sigma^*$, so $D'$ rejects. Otherwise, $L(M_1) = \Sigma^*$ and $D'$ accepts.

    Finally, we have that $D$ runs $M_1$ and either accepts or rejects it depending on whether $L(M_1) = \Sigma^*$. Since $D'$ is a decider
    for $ALL_{CFG}$, we have a contradiction. Therefore, $L_2$ is undecidable.
\end{question}
\end{document}