\documentclass{article}
\usepackage[utf8]{inputenc}
\usepackage{amsmath}
\usepackage{amssymb}
\usepackage{amsfonts}
\usepackage{amsthm}
\usepackage{parskip}
\usepackage{listings}
\usepackage{graphicx}


\title{HW1}
\author{Asier Garcia Ruiz}

\begin{document}
\maketitle

\section*{Question 1}

\begin{lstlisting}
#start_function
void main():
int-list: i, c[101], a[101], b[101], x, y, d[101],
         t0, t1, t2, t3, t4, t5, t6, t8
float-list: t7
    assign, i, 0
    while_loop:
    brgt, if_label_0, i, 100
    array_load t0, c, i
    brneq if_label_1, t0, 0
    array_load, t1, a, i
    array_load, t2, b, i
    add t3, t1, t2
    sub, t4, x, y
    add, t4, t4, 2
    mult, t5, t3, t4
    array_store, t5, c, i 
    goto end_cond
    
    if_label_1:
    sub, t7, x, y
    div, t7, t7, t0
    array_store, t8, d, i

    end_cond
    add, i, i, 1
    goto while_loop

if_label_0:
#end-function
\end{lstlisting}

The code starts and assigns the value of 0 to the variable $i$. Then
the while loop starts and we evaluate the condition $i \leq 100$. If
it is met we go to the end of the program (if\_label\_0), otherwise we continue.
Then we check $c[i] == 0$ and do the arithmetic corresponding to each
branch of the if statement. Finally, we increment $i$ and go to the
beginning of the for loop.

\section*{Question 2}

\includegraphics[scale=0.5]{graphs}

c)
\begin{center}
    \begin{tabular}{| l | l |}
        \hline
        Def                & Uses it reaches \\
        \hline
        1. callr, t0, geti & 9, 14           \\
        \hline
        2. assign, t1, 0   & 5, 7            \\
        \hline
        3. assign, t2, 0   & 12              \\
        \hline
        4. assign t3, 0    &                 \\
        \hline
        8. callr, t2, geti & 12              \\
        \hline
        9. sub, t0, t0, 2  & 14              \\
        \hline
        13. assign, t3, 0  &                 \\
        \hline
        15. assign, t3, 0  &                 \\
        \hline
    \end{tabular}
\end{center}

The end result after the dead code elimination using reaching definitions is

\begin{lstlisting}
    1. callr, t0, geti
    2. assign, t1, 0
    3. assign, t2, 0
    5. call, puti, t1
    6. loop0:
    7. brleq, pool0, t1, 0
    8. callr, t2, geti
    9. sub, t0, t0, 2
    10. goto, loop0
    11. pool0:
    12. call, puti, t2
    14. brleq, else1, t0, 0
    16. else1:
\end{lstlisting}

With this method we have eliminated lines 4, 13, 15 wich assigned a value to
t3 which was never used. The critical instructions form the start and basis of
the analysis, any variable used in a critical instruction is necessary for the
program, and thus anything that reaches those variables also is.

d) There is an additional dead instruction in the code, and that is line 14.
This is because this instruction is a break to nothing and can thus be
eliminated. The reason it is not eliminated is because it is marked as a
critical instruction.

\section*{Question 3}
a)
For Block0 we get the value numbers
\begin{align*}
    1. \  & x^2 = a^0 * b^1     \\
    2. \  & a^3 = z^3           \\
    3. \  & y^4 = a^3 * b^1     \\
    4. \  & z^6 = b^1 + c^5     \\
    5. \  & brgt, label\_0,z,20 \\
\end{align*}
There are no optimisations that can be done with LVN here.

For Block1 we get value numbers
\begin{align*}
    6.\  & a^2 = a^0 + b^1 \\
    7.\  & x^3 = b^1 + c^2 \\
    8.\  & goto, after\_if \\
\end{align*}

There are no optimisations that can be done by LVN here.

For Block2 we get value numbers
\begin{align*}
    9.\   & label\_0        \\
    10.\  & y^2 = b^0 + c^1 \\
\end{align*}
There are no optimisations that can be done by LVN here.

For Block3 we get value numbers
\begin{align*}
    11.\  & after\_if       \\
    12.\  & call, puti, x   \\
    13.\  & call, puti, y   \\
    14.\  & z^2 = a^0 * b^1 \\
    15.\  & b^4 = b^1 + c^3 \\
\end{align*}
There are no optimisations that can be done by LVN here.

b)
Using Extended Basic Blocks we can combine Block0, Block1, and Block2.
For this EBB we get value numbers
\begin{align*}
    (1)\   & x^2 = a^0 * b^1       \\
    (2)\   & a^3 = z^3             \\
    (3)\   & y^4 = a^3 * b^1       \\
    (4)\   & z^6 = b^1 + c^5       \\
    (5)\   & brgt, label\_0, z, 20 \\
    (6)\   & a^4 = a^3 * b^1       \\
    (7)\   & x^6 = b^1 + c^5       \\
    (8)\   & goto, after\_if       \\
    (9)\   & label\_0              \\
    (10)\  & y^6 = b^1 + c^5       \\
\end{align*}

Since BLOCK3 forms an EBB of its own, we have already concluded that there
are no possible optmisations in BLOCK3 in part (b).

After optimising we get
\begin{align*}
    (1)\   & x = a * b             \\
    (2)\   & a = z                 \\
    (3)\   & t1 = a * b            \\
    (4)\   & y = t1                \\
    (5)\   & t2 = b + c            \\
    (6)\   & z = t2                \\
    (7)\   & brgt, label\_0, z, 20 \\
    (8)\   & a = t1                \\
    (9)\   & x = t2                \\
    (10)\  & goto, after\_if       \\
    (11)\  & label_0               \\
    (12)\  & y = t2                \\
    (11)\  & after\_if             \\
    (12)\  & call, puti, x         \\
    (13)\  & call, puti, y         \\
    (14)\  & z = a * b             \\
    (15)\  & b = b + c             \\
\end{align*}

There are 3 more redundant expressions compared to part (a), these are in
lines 8, 9, 12 on the above code segement.
The one in line 8 is due to the reuse of the expression $a * b$ which
is already calculated in line 3 in the code segment above.
The one in lines 9, 12 is due to the reuse of the expression $b + c$ which
These opportunities appear
because in the EBB's there are more a lot of the variables in BLOCK1
and BLOCK2 have the same value number as variables in BLOCK0.

c) There is a missing optimisation in line (15). Since the value of
$b + c$ does not change from line (5) to line (15) because $b$ and $c$
themselves are not changed, we could replace this line
with $b = t2$. The result would look as such

\begin{align*}
    (1)\   & x = a * b             \\
    (2)\   & a = z                 \\
    (3)\   & t1 = a * b            \\
    (4)\   & y = t1                \\
    (5)\   & t2 = b + c            \\
    (6)\   & z = t2                \\
    (7)\   & brgt, label\_0, z, 20 \\
    (8)\   & a = t1                \\
    (9)\   & x = t2                \\
    (10)\  & goto, after\_if       \\
    (11)\  & label_0               \\
    (12)\  & y = t2                \\
    (11)\  & after\_if             \\
    (12)\  & call, puti, x         \\
    (13)\  & call, puti, y         \\
    (14)\  & z = a * b             \\
    (15)\  & b = t2                \\
\end{align*}



\end{document}