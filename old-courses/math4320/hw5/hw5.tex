\documentclass{article}
\usepackage[utf8]{inputenc}
\usepackage{amsmath}
\usepackage{amssymb}
\usepackage{amsfonts}
\usepackage{amsthm}
\usepackage{parskip}

\newcommand{\N}{\mathbb{N}}
\newcommand{\Z}{\mathbb{Z}}
\newcommand{\Q}{\mathbb{Q}}
\newcommand{\R}{\mathbb{R}}
\newcommand{\C}{\mathbb{C}}
\newcommand{\ra}{\xrightarrow{}}
\newcommand{\zbar}{\overline{z}}
\newcommand{\partiald}[2]{\frac{\partial #1}{\partial #2}}

\DeclareMathOperator*{\Log}{Log}
\DeclareMathOperator*{\Arg}{Arg}

\title{HW5}
\author{Asier Garcia Ruiz }
\date{August 2021}

\begin{document}
\maketitle

\section*{31-33}
\subsection*{1} % DONE
Show that
(a)\begin{align*}
    \Log(-ei) & = \ln(e) + i\Arg(-ei) \\
              & = 1 - i\frac{\pi}{2}
\end{align*}
(b) \begin{align*}
    \Log(1 - i) & = \ln(\sqrt{2}) + i\Arg(1 - i)       \\
                & = \frac{1}{2}\ln(2) - i\frac{\pi}{4}
\end{align*}

\subsection*{5} % DONE
(a)
\begin{proof}
    Let $e^{\frac{\pi}{2} + 2n\pi} = i$, then $i^{1/2} = e^{\frac{\pi/2 + 2n\pi}{2}}$.
    Thus $i^{1/2} = e^{\frac{\pi}{4} + n\pi}$. Therefore we have that the first two
    squared roots of $i$ are $e^{i\pi/4}$ and $e^{i5\pi/4}$.

    Now to calculate $\log(e^{i\pi/4})$ we can write
    \begin{align*}
        \log(e^{i\pi/4}) & = \ln(1) + i(\frac{1}{4}+2n)\pi                    \\
                         & = i(\frac{1}{4} + 2n)\pi \qquad (n = 0, \pm 1,...)
    \end{align*}
    and we also have that
    \begin{align*}
        \log(e^{i5\pi/4}) & = \ln(1) + i(\frac{5}{4}+2n)\pi                       \\
                          & = i(\frac{1}{4} + 1 + 2n)\pi \qquad (n= 0, \pm 1,...)
    \end{align*}
    Hence we can conclude that
    $$\log(i^{1/2}) = (n + \frac{1}{4})\pi i \qquad (n = 0, \pm 1,...)$$
\end{proof}
(b)
We can write
\begin{align*}
    \log(i) & = \log(e^{i\pi /2})                                         \\
            & = \ln(1) + i(\frac{1}{2} + 2n)\pi \qquad (n = 0, \pm 1,...) \\
            & = i(\frac{1}{2} + 2n)\pi \qquad (n= 0, \pm 1,...)           \\
            & = 2i(\frac{1}{4} + n)\pi \qquad (n = 0, \pm 1,...)          \\
            & = 2 \log(i^{1/2})
\end{align*}
Hence, we have finally that $\log(i^{1/2}) = \frac{1}{2}\log(i)$.
\subsection*{9} % DONE
\begin{proof}
    We can start by writing
    \begin{align*}
        \log(e^z) & = \ln(|e^xe^{iy}|) + i\arg(e^xe^{iy}) \\
                  & = x + i(y + 2\pi)
        \intertext{However, beacause $y \in \alpha < y < \alpha + 2\pi$.}
                  & = x + iy
    \end{align*}
\end{proof}
\section*{34}
\subsection*{1} % DONE
\begin{proof}
    We note that $-\pi < \Theta_i < \pi$ for $i=1,2$ and $r > 0$ throughout. We write
    \begin{align*}
        \Log(z_1z_2) & = \Log(r_1e^{i\Theta_1}r_2e^{i\Theta_2})                \\
                     & = \Log(r_1r_2e^{i(\Theta_1\Theta_2)})                   \\
                     & = \ln(r_1r_2) + i\Arg(z_1z_2)                           \\
        \intertext{Note that $-2\pi < \Theta_1 + \Theta_2 < 2\pi$. This
            means that $\Arg(z_1z_2) = \Theta_1 + \Theta_2 + 2N\pi$ where
            $N = 0, \pm 1$.}
                     & = \ln(r_1) + \ln(r_2) + i(\Theta_1 + \Theta_2 + 2N\pi)  \\
                     & = \ln(r_1) + i\Theta_1 + \ln(r_2) + i\Theta_2 + 2N\pi i \\
                     & = \Log(z_1) + \Log(z_2) + 2N\pi i
    \end{align*}
\end{proof}
\section*{35-36}
\subsection*{1} % DONE
(a) \begin{align*}
    (1 + i)^i & = e^{i\log(1+i)}                                                               \\
              & = \exp(i(\ln(\sqrt{2})+ i(\frac{1}{4} + 2n)\pi)) \qquad (n=0,\pm 1,...)        \\
              & = \exp(i\frac{\ln(2)}{2} + 2n\pi - \frac{pi}{4}) \qquad (n = 0, \pm 1,...)     \\
              & = \exp(-\frac{\pi}{2} + 2n\pi)\exp(i\frac{\ln(2)}{2}) \qquad (n= 0, \pm 1,...)
\end{align*}
(b) \begin{align*}
    \frac{1}{i^{2i}} & = i^{-2i}                                                            \\
                     & = \exp(-2i\log(i))                                                   \\
                     & = \exp(-2i(\ln(1) + i(\frac{1}{2} + 2n)\pi) \qquad (n= 0, \pm 1,...) \\
                     & = \exp(2(\frac{1}{2} + 2n)\pi) \qquad (n= 0, \pm 1,...)              \\
                     & = \exp((4n + 1)\pi) \qquad (n=0,\pm 1,...)
\end{align*}
\subsection*{3} %DONE
\begin{align*}
    (-1 + \sqrt{3}i)^{3/2} & = \exp(\frac{3}{2}\log(-1 + \sqrt{3}i))                                   \\
                           & = \exp(\frac{3}{2}(\ln(2) + i(\frac{2}{3} +2n)\pi)) \qquad (n=0,\pm1,...) \\
                           & = \exp(\frac{3}{2}\ln(2) + i(1 + 3n)\pi) \qquad (n=0,\pm 1,...)           \\
    \intertext{Clearly, $(1 + 3n)\pi$ only produces real numbers, hence}
                           & = \exp(\ln(2^{3/2}))                                                      \\
                           & = 2^{3/2}                                                                 \\
                           & = \pm 2\sqrt{2}
\end{align*}
\subsection*{8} %DONE
(a) \begin{proof}
    \begin{align*}
        z^{c_1}z^{c_2} & = \exp(c_1\log(z))\exp(c_2\log(z)) \\
                       & = \exp((c_1+c_2)\log(z))           \\
                       & = z^{c_1+c_2}
    \end{align*}
\end{proof}
(b) \begin{proof}
    \begin{align*}
        z^{c_1}/z^{c_2} & = \exp(c_1\log(z))/\exp(c_2\log(z)) \\
                        & = \exp((c_1-c_2)\log(z))            \\
                        & = z^{c1-c2}
    \end{align*}
\end{proof}
(c) \begin{proof}
    \begin{align*}
        z^{c^n} = (\exp(c\log(z)))^n \\
         & = \exp(nc\log(z))         \\
         & = z^{nc}
    \end{align*}
\end{proof}
\section*{37-38}
\subsection*{2} %DONE
\begin{proof}
    (a)
    \begin{align*}
        e^{iz_1}e^{iz_2} & = (\cos z_1 + i\sin z_2)(\cos z_2 + i\sin z_2)                                     \\
                         & = \cos z_1 \cos z_2 - \sin z_1 \sin z_2 + i(\cos z_1 \sin z_2 + \sin z_1 \cos z_2) \\
    \end{align*}
    Now we Show
    \begin{align*}
        e^{-iz_1}e^{-iz_2} & = (\cos (-z_1) + i\sin (-z_2))(\cos (-z_2) + i\sin (-z_2))                                                 \\
                           & = \cos (-z_1) \cos (-z_2) - \sin (-z_1) \sin (-z_2) + i(\cos (-z_1) \sin (-z_2) + \sin (-z_1) \cos (-z_2)) \\
                           & = \cos z_1 \cos z_2 - \sin z_1 \sin z_2 - i(\cos z_1 \sin z_2 + \sin z_1 \cos z_2)                         \\
    \end{align*}
\end{proof}
(b) \begin{proof}
    \begin{align*}
        sin(z_1 + z_2) & = \frac{1}{2i}\left[e^{iz_1}e^{iz_2} - e^{-iz_1}e^{-iz_2} \right]                               \\
                       & = \frac{1}{2i}[\cos z_1 \cos z_2 - \sin z_1 \sin z_2 + i(\cos z_1 \sin z_2 + \sin z_1 \cos z_2) \\
                       & \qquad -(\cos z_1 \cos z_2 - \sin z_1 \sin z_2 - i(\cos z_1 \sin z_2 + \sin z_1 \cos z_2)) ]    \\
                       & = \frac{1}{2i}+2i(\cos z_1 \sin z_2 + \sin z_1 \cos z_2)                                        \\
                       & = \sin z_1 \cos z_2 + \sin z_2 \cos z_1
    \end{align*}
\end{proof}
\subsection*{4} %DONE
(a) \begin{proof}
    \begin{gather*}
        \cos (z + (-z)) = \cos z \cos(-z) - \sin z \sin(-z) \\
        \cos 0 = \cos z \cos z + \sin z \sin z \\
        1 = cos ^2 z + \sin^2 z
    \end{gather*}
\end{proof}
(b) \begin{proof}
    We know that $\sin z$, $z^2$, $\cos z$ are all analytic in $\C$.
    Hence $f(z) = \sin^2 z + \cos^2 z -1$ is also analytic in $\C$.
    We also know that $f(z) \equiv 0$ for all $z \in \C$.
    Hence, $f(z) = \sin^2 z + \cos^2 z - 1 = 0$ and thus
    $\sin^2 z + \cos^2 z = 1$.
\end{proof}
\subsection*{9} %DONE
(a)
\begin{proof}
    We have that $|\sin z|^2 = \sin^2 x + \sinh^2 y$. We also know that $|\sin x| \leq 1$
    for $x \in \R$. Hence, $\sinh^2 y \leq |\sin z|^2$. Now since $sinh^2 y - cosh^2 = 1$
    we have that $\sinh^2 y \leq |\sin z|^2 \leq \cosh^2 y$.
    Then finally $|\sinh y| \leq |\sin z| \leq \cosh y$.
\end{proof}
(b)
\begin{proof}
    We have that $|cos z|^2 = \cos^2 x + \sinh^2 y$. Also, $|\cos x| \leq 1$.
    Hence, $\sinh^2 y \leq |cos z|^2$. Now since $sinh^2 y - cosh^2 = 1$ we
    have that $\sinh^2 y \leq |\cos z|^2 \leq \cosh^2 y$.
    Then finally $|\sinh y| \leq |\cos z| \leq \cosh y$.
\end{proof}
\section*{39}
\subsection*{2} %DONE
(a) \begin{proof}
    \begin{align*}
        \sinh 2z & = \frac{e^{2z} - e^{-2z}}{2}         \\
                 & = \frac{e^z-e^{-z}}{2}(e^z + e^{-z}) \\
                 & = 2\sinh z \cosh z
    \end{align*}
\end{proof}
(b)
\begin{proof}
    \begin{align*}
        \sinh 2z & = -i\sin (i2z)            \\
                 & = -i(2\sin (iz)\cos(iz))  \\
                 & = 2*-i*i\sinh z * \cosh z \\
                 & = 2\sinh z \cosh z
    \end{align*}
\end{proof}
\subsection*{7} %DONE
(a)
\begin{proof}
    \begin{align*}
        \sinh (z + \pi i) & = \sinh z \cosh(\pi i) + \cosh(z) \sinh (\pi i) \\
                          & = -\sinh z + 0 = -\sinh z
    \end{align*}
\end{proof}
(b) \begin{proof}
    \begin{align*}
        \cosh(z + \pi i) & = \cosh z \cosh(\pi i) + \sinh z \sinh (\pi i) \\
                         & = -\cosh z + 0 = -\cosh z
    \end{align*}
\end{proof}
(c)
\begin{proof}
    \begin{align*}
        \tanh(z + \pi i) & = \frac{\sinh(z + \pi i)}{\cosh(z + \pi i)} \\
                         & = \frac{-\sinh z}{-\cosh z}                 \\
                         & = \tanh z
    \end{align*}
\end{proof}
\subsection*{17} %TODO
\begin{proof}
    We first note that $\cosh z = \frac{e^z + e^{-z}}{2}$, hence we write
    \begin{gather*}
        \cosh z = \frac{e^z + e^{-z}}{2} = -2 \\
        e^z + e^{-z} = -4 \\
        \log(e^z + e^{-z}) = \log(-4) \\
        \log(e^x(e^{iy} + e^{-iy})) = \log(-4) \\
        \log(e^x(\cos y + i\sin y + \cos y - i\sin y)) = \ln(4) + (\pi + 2n\pi)i \\
    \end{gather*}
    Hence, $z = \pm \ln(2 + \sqrt{3} + (2n + 1)\pi i) \ (n = 0, \pm 1, \pm 2,...).$
\end{proof}
\section*{40}
\subsection*{4}%DONE
\begin{proof}
    We know that $\sin^{-1}z = w$ for $z,w \in \C$, so $z = \sin w$ and thus
    $\frac{dz}{dw} = \cos w$. We also have that
    $\cos w = (1-\sin^2 w)^{1/2} = (1-z^2)^{1/2}$. Combining the two
    we get $\frac{dz}{dw} = (1-z^2)^{1/2}$ and thus
    $$\frac{dw}{dz} = \frac{d}{dz}\sin^{-1} z = \frac{1}{(1-z^2)^{1/2}}$$
\end{proof}
\end{document}