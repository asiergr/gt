\documentclass{article}
\usepackage[utf8]{inputenc}
\usepackage{amsmath}
\usepackage{amssymb}
\usepackage{amsfonts}
\usepackage{amsthm}
\usepackage{parskip}

\usepackage{graphicx}

\newcommand{\N}{\mathbb{N}}
\newcommand{\Z}{\mathbb{Z}}
\newcommand{\Q}{\mathbb{Q}}
\newcommand{\R}{\mathbb{R}}
\newcommand{\C}{\mathbb{C}}
\newcommand{\zbar}{\overline{z}}
\newcommand{\partiald}[2]{\frac{\partial #1}{\partial #2}}

\DeclareMathOperator*{\Log}{Log}
\DeclareMathOperator*{\Arg}{Arg}
\DeclareMathOperator*{\sech}{sech}
\DeclareMathOperator*{\Res}{Res}

\title{HW 14\&15}
\author{Asier Garcia Ruiz }
\begin{document}
\maketitle

\section*{103-106}
\subsection*{2}
By considering the images of \textit{horizontal} line segments, verify that the
image of the rectangular region $a \leq x \leq b, c \leq y \leq d$ under the
transformation $w = \exp z$ is the region
$e^a \leq \rho \leq e^b$, $c\leq \phi \leq d$, as shown in Fig. 125 (Sec. 103).

\begin{proof}
    We consider $x = x + iy$, then
    \begin{align*}
        f(z) & = \exp(x + iy),   \\
             & = e^xe^{iy},      \\
             & = \rho e^{i\phi}.
    \end{align*}
    Now consider the horizontal line $x=a$, $c \leq y \leq d$. We can see that
    this gets mapped to $e^a e^{i\phi}$ where $c \leq \phi \leq d$. Similarly
    for the line $x=b$, $c \leq y \leq d$ we get
    $e^b e^{i\phi}$ where $c \leq \phi \leq d$. Any other line between these
    will get mapped in the same manner. Hence, the rectangular region
    gets mapped to $e^a \leq \rho \leq e^b$, $c \leq \phi \leq d$.
\end{proof}

\subsection*{7}
Verical half lines were used in the example in Sec. 104 to show that the
transformation $w = \sin z$ is a one to one mapping of the open region
$-\pi / 2 < x \pi / 2$, $y > 0$ onto the half plane $v > 0$.
Verify that result by using, instead, the \textit{horizontal} line segments
$y = c_2 (-\pi / 2 < x < \pi /2)$, where $c_2 > 0$.

\begin{proof}
    We know from the example that $\sin z = u + iv$ where
    \begin{equation*}
        u = \sin x \cosh y, \ v = \cos x \sinh y.
    \end{equation*}
    Now we consider the horizontal line $y = c_2$ where $(-\pi / 2 < x < \pi / 2)$.
    This line gets mapped to $u + iv$ where
    \begin{equation*}
        u = \sin x \cosh c_2, \ v = \cos x \sinh c_2.
    \end{equation*}
    This is the right hand side of the elipse
    \begin{equation*}
        \frac{u^2}{\cosh^2 c_2} + \frac{v^2}{\sinh^2 c_2} = 1,
    \end{equation*}
    with focii at points
    $\sqrt{\cosh^2 c_2 - \sinh^2 c_2} = \pm 1$.

    Thus, we see that as the point $(x, c_2), (-\pi/2 < x < \pi/2)$ moves right
    along the line, its image moves along the upper half of the elipse. Therefore,
    the line $y = c_2 (-\pi / 2 < x < \pi /2)$, where $c_2 > 0$ gets mapped to
    the upper half of the ellipse. Meaning that each one of these lines
    corresponds to the upper part of a different ellipse with the same focii.
    Therefore, there is a 1-to-1 mapping of the open region
    $-\pi/2 < x < \pi/2$, $y> 0$ to the half plane $v>0$.
\end{proof}

\section*{107-108}
\subsection*{4}
Modify the discussion in Example 1, Sec. 107, to show that when $w = z^2$, the
image of the closed traingular region formed by the lines $y = \pm x$ and $x = 1$
is the closed parabolic region bounded on the left by the segment $-2\leq v\leq 2$
of the $v$ axis and on the right by the portion of the parabola
$v^2 = -4(u-1)$. Verify the corresponding points on the two boundaries shown in
Fig. 137.

\begin{proof}
    We know that $w = u + iv$ where
    \begin{equation*}
        u = x^2 - y^2, \ v = 2xy.
    \end{equation*}
    We consider the lines $y = \pm x$ and get that $u = 0$ and $v = \pm 2x^2$.
    For the points of the line from $x=0$ to $x=1$ we get the line segment
    $-2 \leq v \leq 2$. We can see that $D$ at $z = 1 + i$ and
    $B$ at $z = 1 - i$ then clearly get mapped to $(0,2)$ and $(0,-2)$ respectively.

    Now we consider the vertical segment $x = 1$. We then have that
    $u = 1 - y^2, \ v= 2y$. Substituting we then get $u = 1 - \frac{v^2}{4}$.
    Finally rearranging again we get $v^2 = -4(u-1)$. Hence, this segment gets
    mapped to the parabola. We can easily see that $C$ at $z = 1$ gets mapped
    to $(1, 0)$.

    Hence, we have verified all the points in Fig. 137 and shown that the
    traingular region gets mapped to the closed parabolic region as required.
\end{proof}

\subsection*{7}
According to Example 2, Sec. 102, the linear fractional transformation
\begin{equation*}
    Z = \frac{z-1}{z+1}
\end{equation*}
maps the $x$ axis onto the $X$ axis and the half planes $y > 0$ and $y<0$ onto the
half planes $Y>0$ and $Y< 0$, respectively. Show that, in particular, it maps the
segment $-1 \leq x \leq 1$ of the $x$ axis onto the segment $X\leq 0$ of the $X$
axis. Then show that when the principal branch of the square root is used, the
composite function
\begin{equation*}
    w = Z^{1/2} = \left(\frac{z-1}{z+1}\right)^{1/2}
\end{equation*}
maps the $z$ plane, except for the segment $-1 \leq x \leq 1$ of the $x$ axis,
onto the right half plane $u > 0$.

\begin{proof}
    We consider the segment $-1\leq x \leq 1$. We consider first the real part
    by letting $z = x$, then $Z = \frac{x-1}{x+1}$. Now if $Z + X+iY$ we have
    \begin{gather*}
        X + iY = \frac{x - 1}{x + 1},\\
        X = \frac{x-1}{x+1}, Y = 0.
    \end{gather*}
    Now when $-1\leq x \leq 1$ we have $x -1 \leq 0$ and $x + 1 \geq 0$.
    Therefore, $\frac{x-1}{x+1} \leq 0$. So $X \leq 0$ if $-1\leq x \leq 0$.
    This is mapping on the segment.

    We let $Z= re^{i\Theta}$ with $r > 0, -\pi < \Theta < \pi$. Now since the
    segment $-1\leq x \leq 1$ is mapped to $X \leq 0$, the $Z$ we chose does
    not include this segment $X \leq 0$ of the axis.

    Now we will consider $Z^{1/2}$. We can write
    \begin{align*}
        Z^{1/2} & = \exp\left(\frac{1}{2}\log Z\right),            \\
                & = \exp\left(\frac{1}{2}(\ln r + i\Theta)\right), \\
                & = r^{1/2} \exp\left(i\frac{\Theta}{2}\right).
    \end{align*}
    Now if $w = \rho e^{i\phi}$ we have
    \begin{equation*}
        \rho e^{i\phi} = r^{1/2} e^{i\frac{\Theta}{2}}.
    \end{equation*}
    Thus we have that $-\frac{\pi}{2} < \phi < \frac{\pi}{2}$. That is, the image
    lies in the third and fourth quadrants of the $w$-plane, i.e the right half
    plane $u > 0$.

\end{proof}

\section*{112-114}
\subsection*{2}
What angle of rotation is produced by the transformation $w=1/z$ at the point

(a) $z_0 = 1$;

\begin{proof}
    Considering the function $w = f(z) = \frac{1}{z}$ we know the angle of
    rotation is given by $\arg f'(z) = \arg \left(-\frac{1}{z^2}\right)$. At
    $z_0$ we have $f(z_0) = -1$, and $\arg -1 = \pi$. Hence the angle of rotation
    is $\pi$.
\end{proof}

(b) $z_0 = i$

\begin{proof}
    Considering the function $w = f(z) = \frac{1}{z}$ we know the angle of
    rotation is given by $\arg f'(z) = \arg \left(-\frac{1}{z^2}\right)$. At
    $z_0$ we have $f(z_0) = 1$, and $\arg 1 = 0$. Hence the angle of rotation
    is $0$.
\end{proof}


\subsection*{6}
Find the local inverse of te transformation $w  = z^2$ at the point

(a) $z_0 = 2$

\begin{proof}
    We start by noting that $w = 2^2 = 4 = 4e^{i2n\pi}$. Now if $w = f(z) = z^2$
    then $z = g(w) = w^{1/2} = \rho^{1/2} e^{i\phi/2}$. However, points like
    $w = 4$ can go to $z = -2$. Hence we restrict the domain to end with
    $g(w) = w^{1/2} = \rho^{1/2} e^{i\phi/2}, -\pi < \phi < \pi$.
\end{proof}

(b) $z_0 = -2$

\begin{proof}
    We start by noting that $w = (-2)^2 = 4 = 4e^{i2n\pi}$. Now if
    $w = f(z) = z^2$ then $z = g(w) = \rho^{1/2} e^{i\phi/2}$. However,
    points like $w = 4$ can go to $z = 2$. Hence we restrict the domain to end with
    $g(w) = w^{1/2} = \rho^{1/2} e^{i\phi/2}, \pi < \phi < 3\pi$.
\end{proof}

(c) $z_0 = -i$
\begin{proof}
    We start by noting that $w = (-i)^2 = -1 = e^{i(3\pi/2 + 2n\pi)}$. Now if
    $w = f(z) = z^2$ then $z = g(w) = \rho^{1/2} e^{i\phi/2}$.
    We restrict the domain to end with
    $g(w) = w^{1/2} = \rho^{1/2} e^{i\phi/2}, 2\pi < \phi < 4\pi$ to have a
    single branch.
\end{proof}

\section*{93-94}
\subsection*{1}
Let $C$ denote the unit circle $|z|=1$, described in the positive sense.
Use the theorem in Sec. 93 to determine the value of $\Delta_C\arg f(z)$ when

(a) $f(z) = z^2$;
\begin{proof}
    We see that $f(z)$ has one zero of multiplicity 2 at $z = 0$, and no poles
    in $C$. Hence using the theorem
    \begin{equation*}
        \Delta_C\arg f(z) = 2\pi(2) = 4\pi.
    \end{equation*}
\end{proof}

(b) $f(z) = \frac{1}{z^2}$;
\begin{proof}
    We see that $f(z)$ has no zeros on $C$, and it has a pole of multiplicity 2
    at $z = 0$.
    Hence
    \begin{equation*}
        \Delta_C \arg f(z) = 2\pi (0-2) = -4\pi.
    \end{equation*}
\end{proof}

(c) $f(z) = (2z-1)^7 / z^3$;
\begin{proof}
    We see that $f(z)$ has a zero of multiplicity 7 at $z=1/2$ and a pole of multiplicity
    3 at $z=0$. Hence
    \begin{equation*}
        \Delta_C \arg f(z) = 2\pi (7-3) = 8\pi
    \end{equation*}
\end{proof}

\subsection*{3}
Using the notation in Sec.93, suppose that $\Gamma$ does not enclose the origin
$w=0$ and that there is a ray from that point which does not intersect $\Gamma$.
By observing that the absolute value of $\delta_C f(z)$ must be less than $2\pi$
when a point $z$ makes one cycle around C and recalling that $\delta_C f(z)$
is an integral multiple of $2\pi$, point out why the winding number of $\Gamma$
with respect to the origin $w=0$ must be zero.

\begin{proof}
    Becuase $\Gamma$ does not enclose the origin we have that
    \begin{equation*}
        0 \leq \Delta_C \arg f(z) < 2\pi.
    \end{equation*}
    By the argument principle,
    \begin{equation*}
        \Delta_C \arg f(z) = Z- P = 2n\pi.
    \end{equation*}
    Because $0 \leq \Delta_C \arg f(z) < 2\pi$. the only possibility is
    \begin{equation*}
        \Delta_C \arg f(z) = 0.
    \end{equation*}
    Which implies, $2n\pi =0$ and thus $n= 0$. That is, the winding number is zero.
\end{proof}

\subsection*{8}
Determine the number of roots, counting multiplicities, of the equation
\begin{equation*}
    2z^5 - 6z^2 + z + 1 = 0
\end{equation*}
in the annulus $1 \leq |z| < 2$.

\begin{proof}
    We first let $f(z) = 2z^5$ and $g(z) = -6z^2 + z + 1$ and consider the curve
    $C_1: |z| = 2$. We have that
    \begin{equation*}
        |f(z)| = 2|z^5| = 2|z|^5 = 2(2^5) = 64
    \end{equation*}
    and
    \begin{equation*}
        |g(z)| = |-6z^2 + z + 1| \leq 6|z|^2 + |z| + 1 = 27.
    \end{equation*}
    So $|f(z)| > |g(z)|$ on $C_1$. Also, $f(z)$ has one zero of multiplicity
    5 in $C_1$. So by Rouche's theorem
    $f(z) + g(z) + 2z^5 - 6z^2 + z + 1$ has 5 zeros in $C_1$.

    Now consider $C_2: |z| = 1$. We let $f(z) = -6z^2$ and $g(z) = 2z^5 + z + 1$.
    We find
    \begin{equation*}
        |f(z)| = |-6z^2| = 6,
    \end{equation*}
    and
    \begin{equation*}
        |g(z)| = |2z^5 + z + 1| \leq 2|z|^5 + |z| + 1 = 2 + 1 +1 = 4.
    \end{equation*}
    Hence, $|f(z)| > |g(z)|$ on $C_2$. Now $f(z)$ has one zero of multiplicity
    3 in $C_1$. So by Rouche's theorem
    $f(z) + g(z) + 2z^5 - 6z^2 + z + 1$ has 3 zeros in $C_2$.

    Finally, we get that
    \begin{equation*}
        |f(z)| = 2|z^5| = 2|z|^5 = 2(2^5) = 64
    \end{equation*}
    has 3 zeros in the annular domain $1 \leq |z| < 2$

\end{proof}

\end{document}