\documentclass{article}
\usepackage{parskip}

\title{Project 1 Wrapper}
\author{Asier Garcia Ruiz }
\date{August 2021}

\begin{document}
\maketitle
\section*{Question 1}
In the Romania example in class we considered the length of roads to be the edge cost between vertices (cities),
and the optimality of the route was respective to the shortest distance a car would actually have to travel by following roads.
How might we account for speed limits? How might we account for traffic conditions if we know that traffic is flowing slower
than the speed limit?

\textit{Answer:}
There are two possible ways to account for these issues: by including them within the edge cost or the heuristic function.
Something constant like the speed limits would probably be better implemented in the edge weight.
A measure like $\frac{distance}{speed\ limit}$ could easily account for this factor, with
more complex realtions (e.g exponential, log) being used do weight things differently.

Something like traffic would probably be better accounted for in the heuristic function, since it can fluctuate more
and it is counterintuitive to have edge weights change. We could increase the heuristic function value for
roads with more traffic, for example.

\section*{Question 2}
Cities are designed with a few major roads and a lot of smaller roads.  Sometimes the shortest route between two
places is through a neighborhood. Suppose there is a neighborhood in between two very popular destinations where a
lot of children live and could be playing in the streets, and A* has been routing cars through  the  neighborhood.
As  more  people  use  GPS-based  routing  services, the neighborhood has started seeing an increase in cars cutting
through dangerously fast.  Suppose we wanted to discourage A* from routing cars through
the  neighborhood.   What  would  happen  if  we  artificially  adjusted  the  speed limit on roads in the neighborhood
versus if we artificially increased the heuristic  values  of  intersections  in  the  neighborhood?
Would  either  guarantee  cars never cut through the neighborhood? Would either keep people who live in
the neighborhood from generating routes to and from their homes?

\textit{Answer:}
Artifically increasing the speed limit and increasing the heuristic would have the same effect
since A* minimises $g(x)+h(x)$. Neither could guarantee 100\% that no cars will go there,
since maybe traffic is so bad in other roads that the algorithm decides to go through the
neightbourhood, as an example. However, it would minimise the amount of cars going through.

The algorithm would not keep people from getting to/from their house since it will not stop
until it finds the destination. It may, however, take them through paths that do not seem
intuitive at first. For example, the algorithm may take you through roads that have
higher speed limits or less traffic simply because those are the values used. In reality
this could result in inefficient or strange paths.

\section*{Question 3}
There is currently a big societal concern regarding artificial intelligence and automation affecting jobs.
How do routing systems impact jobs? Is their impact mainly positive or mainly negative?

\textit{Answer:}
The impact of routing systems on end-user experience is definitely positive. Apps like Google Maps
allow for much easier and user-friendly navigation through unknown places. In terms of negative
effects, obviously map makers have been affected by this. Another aspect of routing algorithms
is self-driving cars, which make jobs like truck/taxi drivers superflous. However, the systems
are yet not advanced enough to fully substitute these workers, but rather make their life easier.
Overall, I would say the influence is positive.

\section*{Question 4}
Reliance on artificial intelligence systems can change human behavior in unanticipated ways.
Describe one way in which a routing system can have an unde-sirable impact on human behavior.

\textit{Answer:}
Humans becoming dependant on routing systems causes them to maybe make irrational decisions
just because the algorithm dictates so. Algorithms can be wrong sometimes, and if one is too
dependant on the output, it can cause issues like perhaps driving too fast through a
residential neighborhood where children routinely play in the street.

\end{document}