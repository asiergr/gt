\documentclass{article}
\usepackage[utf8]{inputenc}
\usepackage{amsmath}
\usepackage{amssymb}
\usepackage{amsfonts}
\usepackage{amsthm}
\usepackage{parskip}

\newcommand{\N}{\mathbb{N}}
\newcommand{\Z}{\mathbb{Z}}
\newcommand{\Q}{\mathbb{Q}}
\newcommand{\R}{\mathbb{R}}
\newcommand{\C}{\mathbb{C}}
%\DeclareMathOperator{\mod}{mod}


\title{HW2}
\author{Asier Garcia Ruiz}

\begin{document}
\maketitle

Lauritzen Chapter 1: @1, 2, 3, @6 i-v, @8, @11
% Done
\section*{1}
Prove that if a subset $S \subseteq \Z$ has a first element then the latter has
to be unique.

\begin{proof}
    We begin by assuming that both $x$ and $x'$ are first elements of $S$.
    This means that $x \leq s, \ \forall s \in S$ and $x' \leq s \ \forall s \in S$.
    However, this implies that $x \leq x'$ and $x' \leq x$, therefore $x = x'$.
\end{proof}

\section*{2}
Let $x, d \in \Z$, where $d > 0$. Prove that $M \cap \N \neq \emptyset$, where
$M = \{x-qd | q \in \Z\}$.

\begin{proof}
    Consider the case when $x \geq 0$. Then there exists $q < x$ such that $0 < qd < x$
    and thus $x-qd \in \N$. Now consider the case when $x < 0$, then there exists
    $0 > q > x$ such that $qd < x$ and thus $x - qd > 0$. In both cases we see that
    $x -qd \in M$ and $x-qd\in\N$. Therefore, $M\cap \N \neq \emptyset$.
\end{proof}

\section*{3}
Let $a,b,N \in \Z$, where $N>0$. Prove that $[ab] = [[a][b]]$, where $[x]$ denotes
the remainder of $x$ after division by $N$.

\begin{proof}
    From Proposition 1.3.2 we have that $a \equiv [a]_N \mod N$. Now since
    we have that $a \equiv [a]_N \mod N$ and $b \equiv [b]_N \mod N$ from
    Proposition 1.3.4 we get
    \begin{equation*}
        ab \equiv [a]_N[b]_N \mod N.
    \end{equation*}
    Now, since they are equivalent $\mod N$ we have that
    \begin{equation*}
        [ab] = [[a][b]]
    \end{equation*}
    as needed.
\end{proof}

\section*{6}
Let $a$ be a number written (in base 10) as
\[a_0 * 10^0 + a_1*10^1 + a_2*10^2 + \dots + a_n*10^n\]
where $0 \leq a_i \leq 10$.

(i) Prove that 2 divides $a$ if and only if 2 divides $a_0$.
\begin{proof}
    We start by noting that $10 \equiv 0 \mod 2$ and thus $10^n \equiv 0 \mod 2$.
    Hence
    \begin{align*}
        a & = a_0 * 10^0 + a_1*10^1 + a_2*10^2 + \dots + a_n*10^n, \\
          & \equiv a_0 + a_1*0 + a_2*0 + \dots + a_n*0 \mod 2,     \\
          & \equiv a_0 \mod 2.
    \end{align*}
    Therefore, $2|a \iff 2|a_0$.
\end{proof}

(ii) Prove that 4 divides $a$ if and only if 4 divides $a0 + 2a1$.
\begin{proof}
    We start by noting that $10 \equiv 2 \mod 4$, and $10^n \equiv 0 \mod 4, n > 1$.
    Hence
    \begin{align*}
        a & = a_0 * 10^0 + a_1*10^1 + a_2*10^2 + \dots + a_n*10^n, \\
          & \equiv a_0 + 2a_1 + a_2*0 + \dots + a_n*0 \mod 4,      \\
          & \equiv a_0 + 2a_1.
    \end{align*}
    Therefore, $4 | a \iff 4 | a_0 + 2a_1$.
\end{proof}

(iii) Prove that 8 divides $a$ if and only if 8 divides $a0 + 2a1 + 4a2$.
\begin{proof}
    We start by noting that $10 \equiv 2 \mod 8$, $10^2 \equiv 4 \mod 8$,
    and $10^n \equiv 0 \mod 8, n > 2$. Hence,
    \begin{align*}
        a & = a_0 * 10^0 + a_1*10^1 + a_2*10^2 + \dots + a_n*10^n,   \\
          & \equiv a_0 + 2a_1 + 4a_2 + a_3*0 + \dots + a_n*0 \mod 8, \\
          & \equiv a_0 + 2a_1 + 4a_2 \mod 8.
    \end{align*}
    Therefore, $8 | a \iff 8 | a_0 + 2a_1 + 4a_2$.
\end{proof}

(iv)Prove that 5 divides $a$ if and only if 5 divides $a_0$.
\begin{proof}
    We start by noting that $10^n \equiv 0 \mod 5$,
    Hence,
    \begin{align*}
        a & = a_0 * 10^0 + a_1*10^1 + a_2*10^2 + \dots + a_n*10^n, \\
          & \equiv a_0 + a_1*0 + a_2*0 + \dots + a_n*0 \mod 5,     \\
          & \equiv a_0 \mod 5.
    \end{align*}
    Therefore, $5 | a \iff 5 | a_0$.
\end{proof}

(v)Prove that 9 divides $a$ if and only if 9 divides the sum
$a_0 +a_1 +\dots +a_n$ of its digits.
\begin{proof}
    We start by noting that $10^n \equiv 1 \mod 9$,
    Hence,
    \begin{align*}
        a & = a_0 * 10^0 + a_1*10^1 + a_2*10^2 + \dots + a_n*10^n, \\
          & \equiv a_0 + a_1 + a_2 + \dots + a_n \mod 9,           \\
    \end{align*}
    Therefore, $9 | a \iff 9| \sum_{i=0}^n a_i$.
\end{proof}

\section*{8}
Prove that $3 | 4^n - 1$ where $n \in \N$.

\begin{proof}
    We note that
    \begin{align*}
        4^n - 1 & = \underbrace{4*4*4\dots}_{\text{n times}} - 1,             \\
        \intertext{Now taking mod 3 and by Proposition 1.3.2}
                & \equiv \underbrace{1*1*1\dots}_{\text{n times}} - 1 \mod 3, \\
                & \equiv 0 \mod 3.
    \end{align*}
    This means that $3 | 4^n -1 - 0$, and thus $3 | 4^n -1$.
\end{proof}

\section*{11}
Let $x,y,z \in \Z$. Prove the following statements.

(i) $x \equiv x \ \mod d$
\begin{proof}
    We can write $x = qc + r$ where $x \mod d = r$. Clearly if $x = x$ then by
    the uniqueness property of Definition 1.2.1 we have that $x \equiv x \mod d$.
\end{proof}

(ii) If $x \equiv y \mod d$ then $y \equiv x \mod d$.
\begin{proof}
    If $x \equiv y \mod d$ by Proposition 1.3.2 (ii) we know that $[x]_d = [y]_d$.
    And thus since equality is reflexive we have that
    $y \equiv x \mod d$.
\end{proof}

(iii) If $x \equiv y \mod d$ and $y \equiv d \mod d$ then $x \equiv z \mod d$.
\begin{proof}
    If $x \equiv y \mod d$ and $y \equiv d \mod d$ then by Proposition 1.3.2 we
    know that $[x]_d = [y]_d$, and $[y]_d = [z]_d$. By the transitivity of
    equality we have that $[x]_d = [z]_d$ and thus
    $x \equiv z \mod d$.
\end{proof}

\end{document}

