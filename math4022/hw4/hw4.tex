\documentclass{article}
\usepackage[utf8]{inputenc}
\usepackage{amsmath}
\usepackage{amssymb}
\usepackage{amsfonts}
\usepackage{amsthm}
\usepackage{parskip}
\usepackage{bm}
\usepackage{framed}
\usepackage{tikz}
\usepackage{wasysym}
\usetikzlibrary{shapes,snakes}
\usetikzlibrary{arrows.meta}
\usetikzlibrary{decorations.pathreplacing}
\usetikzlibrary{decorations.markings}

\newcommand{\contradiction}{\Rightarrow\!\Leftarrow}
\newcommand{\N}{\mathbb{N}}
\newcommand{\Z}{\mathbb{Z}}
\newcommand{\Q}{\mathbb{Q}}
\newcommand{\R}{\mathbb{R}}
\newcommand{\C}{\mathbb{C}}
\newcommand{\defeq}{:=}
\newcommand{\set}[1]{\{#1\}}
\renewcommand{\epsilon}{\varepsilon}

\DeclareMathOperator*{\dist}{dist}
\DeclareMathOperator*{\val}{val}

\newenvironment{hwproof}[2]
{
    \textbf{Problem #1.}\\
    #2
    \begin{proof}
}{
    \end{proof}
    \newpage
}

\title{Homework 1}
\author{Asier Garcia Ruiz}

\begin{document}
\maketitle

I worked on this homework with and Ryan Hamblet.

\begin{hwproof}
    {1}
    {The goal of this problem is to show that the Ford--Fulkerson algorithm may run forever and not even get close to the maximum feasible
        flow value if some of the capacities are irrational.

        \medskip

        Consider the network $N$ shown in Fig.1. Here $\phi = (\sqrt{5}-1)/2 = 0.618\ldots$ is the positive solution to the equation
        $\phi^2 = 1 - \phi$ (also known as the \textbf{Golden Ratio}.) Note that all of the capacities in this network are integers, except for
        the capacity of the connection from $a$ to $b$.
    }

    (a) What is the maximum value of a feasible flow in $N$?

    By the cut max-flow min-cut theorem, we can easily see by inspection that the maximum feasible flow in $N$ is 201.

    Suppose we are trying to use the Ford--Fulkerson algorithm to find a maximum flow in $N$. We start with the zero flow $f_0$, and then build
    the next flow $f_1$ as follows:

    \begin{leftbar}
        Use the augmenting path $(s,c,b,t)$ to obtain a flow $f_1$.
    \end{leftbar}
    (b) Compute the flow $f_1$. What is $\val(f_1)$?

    We observe that $\val(f_1) = 1$.

    Next, we apply the following sequence of four augmenting paths:

    \begin{leftbar}
        Use the augmenting path $(s,a,b,c,d,t)$ to obtain a flow $f_2$.

        Use the augmenting path $(s,c,b,a,t)$ to obtain a flow $f_3$.

        Use the augmenting path $(s,a,b,c,d,t)$ to obtain a flow $f_4$.

        Use the augmenting path $(s,d,c,b,t)$ to obtain a flow $f_5$.
    \end{leftbar}

    We then continue to repeatedly use the same four augmenting paths. That is, once we have computed $f_{4k+1}$ for some $k$, we do the following:

    \begin{leftbar}
        Use the augmenting path $(s,a,b,c,d,t)$ to obtain a flow $f_{4k+2}$.

        Use the augmenting path $(s,c,b,a,t)$ to obtain a flow $f_{4k+3}$.

        Use the augmenting path $(s,a,b,c,d,t)$ to obtain a flow $f_{4k+4}$.

        Use the augmenting path $(s,d,c,b,t)$ to obtain a flow $f_{4k+5}$.
    \end{leftbar}

    (c) Show that for every $k \geq 0$, the following equalities hold:
    \[
        f_{4k+1}(a,b) \,=\, \phi - \phi^{2k+1}, \qquad f_{4k+1}(c, b) \,=\, 1, \qquad \text{and} \qquad f_{4k+1}(c, d) \,=\,  1 - \phi^{2k}.
    \]

    \begin{proof}
        We will prove this using induction. We start by noting that the edges with capacity of 100 are "large enough", that is, they will never consititute the
        minimum excess in the augmenting path. At $k = 0$ we conider $f_1$, which is the flow we get after augmenting on $(s,c,b,t)$. We get that
        \[
            f_{1}(a,b) = 0 = \phi - \phi^{2*0 +1}, \qquad f_1(c,b) = 1, \qquad f_1(c,d) = 1 = 1 - \phi^{2*0}
        \]
        Now, we assume $4k + 1$ and prove $4k + 5$. Assuming $4k + 1$ we have the graph.
        After augmenting on $p_1 = (s,a,b,c,d,t)$ we get that $\epsilon(p_1) = \phi^{2k + 1}$. Augmenting on this path we now have that
        \[
            f_{4k+2}(a,b) \,=\, \phi, \qquad f_{4k+2}(c, b) = 1 - \phi^{2k + 1}, \qquad \text{and} \qquad f_{4k+2}(c, d) =  1 - \phi^{2k + 2}.
        \]

        Now, we augment on $p_2 = (s,c,b,a,t)$ we get $\epsilon = \phi^{2k + 1}$. Augementing on this path we get
        \[
            f_{4k+3}(a,b) \,=\, \phi - \phi^{2k + 1}, \qquad f_{4k+3}(c, b) = 1, \qquad \text{and} \qquad f_{4k+3}(c, d) =  1 - \phi^{2k + 2}.
        \]

        Now we augment on $p_3 = (s,a,b,c,d,t)$ and we get that $\epsilon(p_3) = \phi^{2k + 2}$. Augmenting on this path we get
        \[
            f_{4k+4}(a,b) \,=\, \phi - \phi^{2k + 1}, \qquad f_{4k+4}(c, b) = 1, \qquad \text{and} \qquad f_{4k+4}(c, d) =  1 - \phi^{2k + 2}.
        \]

        Now we augment on $p_4 = (s,d,c,b,t)$ and we get that $\epsilon(p_4) = \phi^{2k + 2}$. Augmenting on this path we get
        \[
            f_{4k+5}(a,b) \,=\, \phi - \phi^{2k + 3}, \qquad f_{4k+5}(c, b) = 1, \qquad \text{and} \qquad f_{4k+5}(c, d) =  1 - \phi^{2k + 2}.
        \]

        Now, we can clearly see that $4k + 1$ implies $4k + 5 = 4(k + 1) + 1$ as needed. Hence, we are done.
    \end{proof}

    (d)
    \begin{proof}
        Again, we will prove this using induction. We see that for $k = 1$ we have $\val(f_1) = 3 = 1 + 2\phi + 2\phi^2$. Now, we assume that the flow at
        $4k + 1$ is $1 + 2\phi + 2\phi^2 + \dots 2\phi^{2k}$. Now, from the previous exercise we know how much flow we added using the previous agumenting paths.
        The flow we added is exactly $2\phi^{2k + 1} + 2\phi^{2k + 2}$. Since the FF algorithm only adds flow at each step we now have that the flow at
        $4k + 5 = 4(k + 1) + 1$ is exactly $ 1 + 2\phi + 2\phi^2 + \dots 2\phi^{2k} + 2\phi^{2k + 1} + 2\phi^{2k + 2}$ as needed.
    \end{proof}

    (e)
    \begin{proof}
        We can represent the flow at each $k$ as a geometric series $-1 + \sum_0^n 2\phi^n$. Since $\phi < 1$ we know this series converges to the value
        $-1 + 2/(1 - \phi) \approx 4.236 < 5$.
    \end{proof}
\end{hwproof}

\begin{hwproof}
    {2}
    {Given two graphs $G$ and $H$, their \textbf{union} is the graph $G \cup H$ with
        \[
            V(G \cup H) \,=\, V(G) \cup V(H) \qquad \text{and} \qquad E(G \cup H) \,=\, E(G) \cup E(H).
        \]
        Show that if $G$ and $H$ are $k$-connected graphs such that $|V(G) \cap V(H)| \geq k$, then the graph $G \cup H$ is also $k$-connected.}

    Let $S \subseteq V(G \cup V)$, we will show that if $|S| < k$ then $G \cup H$ is still connected. When we remove vertices from $V(G) \triangle V(H)$ the union is
    still connected as $G, H$ are $k$-connected. Now, if we remove vertices from $V(G\cap H)$ we know that since $|S| < k$ and $G, H$ are $k$-connected that this
    will not disconnect the graph. This is because there will always be at least one vertex in the intersecting set that had $k$ disjoint paths to every other
    vertex in the graph, which cannot be made disjoint by removing less that $k$ edges (Menger's Thm.).
\end{hwproof}

\begin{hwproof}
    {3}
    {In this problem you shall prove the vertex version of Menger's theorem, namely:

        \textbf{Theorem.} \textsl{If $G$ is a finite simple graph and $x$, $y \in V(G)$ are distinct non-adjacent vertices, then $\kappa(x,y) = \lambda(x,y)$.}

        To prove this theorem, we set up a network $N$ as follows. For each vertex $v \in V(G) \setminus \set{x,y}$, we create a pair of new vertices $v_{\mathrm{in}}$ and $v_{\mathrm{out}}$. The vertex set of $N$ is
        \[
            V \,\defeq\, \set{x,y} \cup \set{v_{\mathrm{in}}, v_{\mathrm{out}} \,:\, v \in V(G) \setminus \set{x,y}}.
        \]
        Make $x$ the source and $y$ the sink. As for the capacity function, we define:


        \begin{itemize}
            \item $c(v_{\mathrm{in}}, v_{\mathrm{out}}) \defeq 1$ for all $v \in V(G) \setminus \set{x,y}$;

            \item $c(v_{\mathrm{out}}, u_{\mathrm{in}}) \defeq \text{``}\infty\text{''}$ for all adjacent $u$, $v \in V(G) \setminus \set{x,y}$;

            \item $c(x, v_{\mathrm{in}}) \defeq \text{``}\infty\text{''}$ for all $v \in N_G(x)$; and

            \item $c(v_{\mathrm{out}}, y) \defeq \text{``}\infty\text{''}$ for all $v \in N_G(y)$.
        \end{itemize}

        (You should explain what $\text{``}\infty\text{''}$ means here.) All other capacities are zero.

        Use the network $N$ to prove the theorem.}

    First, we note that in this network "$\infty$" is simply $\kappa(x,y)$. Now, we note that since the capacity is integral, there exists a maximum flow of
    integral value.

    By construction of $N$ we have that in a maximum integral flow $f(v_{in}, v_{out}) = 1$ or 0 for all $v \in V(G)\backslash\set{x,y}$. This
    implies that a flow from $x$ to $y$ with a flow of value $k$ means there are $k$ vertex disjoint paths from $x$ to $y$. This is because if any two paths
    used the same vertex, then either the flow is greater than 1 (a contradiction) or is not integral. Therefore,if there is a flow of value $k$ in this network,
    then we have $k$ internally vertex disjoint paths in $G$.

    Now we must show that $\kappa(x,y) = k$ implies this property. Consider any $XY$ cut in $G$, if this cut has some $v_{out} \in X$ then clearly this cut has
    capacity greater than $k$, since $v_{out}$ alone has capacity at least $k$ coming out of it. So, consider an $XY$ cut where all the elements in $X$ are
    $v_{in}$, then, all the elements of $Y$ are the $v_{out}$. We observe that the capacity is also $k$, since otherwise we could remove corresponding vertices in
    $G$ and separate $x$ and $y$ with fewer than $k$ vertices, a contradiction.

    Therefore, every $XY$ cut in $N$ has capacity at least $\kappa(x,y)$. This implies there is an integral flow of this value, meaning there are at least $k$
    internally vertex disjoint paths from $x$ to $y$. This now implies that $\lambda(x,y) \geq \kappa(x,y)$.

    Now, it is relatively obvious that $\kappa(x,y) \geq \lambda(x,y)$ since otherwise if there were $\lambda(x,y)$ vertex disjoint paths from $x$ to $y$, we would have
    to remove at least one vertex from each to disconnect them.

    Therefore, we have that $\kappa(x,y) = \lambda(x,y)$ as required.

\end{hwproof}

\begin{hwproof}
    {4}
    {Let $G$ be a $k$-connected finite simple graph, where $k \geq 2$. Prove that if $S \subseteq V(G)$ is a set of size $k$, then there is a cycle $C$ in $G$ such that $S \subseteq V(C)$.

        \textbf{Hint.} Show that if $C$ is a cycle such that $S \not \subseteq V(C)$, then there is another cycle $C'$ with $|S \cap V(C')| > |S \cap V(C)|$.}

    Let $C$ be a cycle containing as many of the $s \in S$ as possible. Now, assume, for the sake of contradiction, that only $l$ of the vertices in $S$ are actually
    in $C$ and call these vertices $v_1, v_2, \dots, v_l$, ($l < k$). Then, there exists some $u \in S$ such that $u \not \in C$. By Menger's Theorem we know
    there are at least $k$ independent paths from $u$ to $C$. However, these paths must meet in between each of the vertices $v_1, v_2,\dots ,v_l$ with no paths
    meeting in the same portion of the cycle $v_iCu$, or else there exists a larger cycle containing $u$. However, if this is not the case, there are already
    $k$ elements in $C$ (since we essentially partitioned the cycle into $k$ segments, each containing at least one vertex from $S$), a contradiction.

    Therefore, this cycle exists as needed.
\end{hwproof}

\begin{hwproof}
    {5}
    {Let $G$ be a finite graph. Given two distinct vertices $x$, $y \in V(G)$, the notation $x \,\overset{k}{\aquarius}\, y$ means that $G$ contains $k$ edge-disjoint $xy$-paths.

        \smallskip

        \noindent Let $x$, $y$, $z \in V(G)$ be three distinct vertices. Show that if $x \,\overset{k}{\aquarius}\, y$ and $y \,\overset{k}{\aquarius}\, z$, then $x \,\overset{k}{\aquarius}\, z$.}

    Assume, for the sake of contradiction, that this is not true. That is, there exist $n < k$ disjoint paths between $x$ and $z$. Then, we can use $\leq k - 1$
    edges to disconnect them. Call this set of edges $E$. We claim that there is at least one $xy$ path and one $yz$ path that contain no edges form $E$.

    To prove this claim, we assume that this is not true. Since these paths are composed of distinct edges, this would imply that $|E| \leq k$, which is a contradiction.
    Therefore, our claim is true.

    Now, consider the auxiliary graph $G - E$. Since $|E| \leq k - 1$ there is one $xy$ path and one $yz$ path remaining. Therefore, there is one more $xz$ path,
    a contradiction.
\end{hwproof}

\end{document}