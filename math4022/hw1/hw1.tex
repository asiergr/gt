\documentclass{article}
\usepackage[utf8]{inputenc}
\usepackage{amsmath}
\usepackage{amssymb}
\usepackage{amsfonts}
\usepackage{amsthm}
\usepackage{parskip}
\usepackage{bm}

\newcommand{\contradiction}{\Rightarrow\!\Leftarrow}
\newcommand{\N}{\mathbb{N}}
\newcommand{\Z}{\mathbb{Z}}
\newcommand{\Q}{\mathbb{Q}}
\newcommand{\R}{\mathbb{R}}
\newcommand{\C}{\mathbb{C}}

\DeclareMathOperator*{\dist}{dist}

\newenvironment{hwproof}[2]
{
    \textbf{Problem #1.}\\
    #2
    \begin{proof}
}{
    \end{proof}
}

\title{Homework 1}
\author{Asier Garcia Ruiz}

\begin{document}
\maketitle

I worked on this homework with Armaan Lala and Ryan Hamblet.

\begin{hwproof}{1}{
        Use the definition of a connected graph to prove that a nonempty graph $G$
        is connected if and only if there does not exist a nontrivial partition of $V(G)$
        with no edges between the parts.\\
        (Recall that a nontrivial partition of $V(G)$ is a pair $(A, B)$ of nonempty subsets
        of $V(G)$ such that $A \cap B = \emptyset$ and $A \cup B = V(G)$.)
    }
    ($\Rightarrow$) We assume that $G$ is connected. We want to show that there does not exist
    a nontrivial partition of $V(G)$ with no edges between the parts. Since $G$ is connected,
    we have that $\forall u \in V(G)$ there exists a path from $u$ to any other $v \in V(G)$.

    Now, for the sake of contradiction, assume that there exists at least one nontrivial partition of
    $V(G)$ with no edges between any two parts. Call any two of these disconnected parts $C_1$
    and $C_2$. Now let $u \in C_1$ and $v \in C_2$. Clearly, there cannot be a path from
    $u$ to $v$, a contradiction $\contradiction$.

    ($\Leftarrow$) We assume there does not exist a nontrivial partition of $V(G)$ with no edges
    between the parts. Then, clearly, we can construct a path from any vertex $u$ to any other
    vertex $v$. Thus, the graph is connected (since there is exactly one connected component).
\end{hwproof}

\begin{hwproof}{2.a}{
        Let $G$ be a an $n$-vertex graph, where $n \geq 1$.\\
        (a) Prove that if $G$ has no cycles, then $|E(G)| \leq n - 1$.
    }
    We will prove this by minimal coutnerexample. Consider a graph $G$ that has no cycles such
    that $|E(G)| > n - 1$. The minimal counterexample is a graph $G$ with no cycles and
    $|E(G)| = n$.

    Now, consider a path $P = v_0v_1\dots v_k$ of maximal length in $G$. We will show that
    $\deg(v_0) = 1$. Clearly, since $v_0v_1$ is an edge we have that $\deg(v_0)$. (Note:
    there is a degenrate case where the longest path is a signle vertex, but this implies there
    are no edges).
    Consider a vertex $u \neq v_1$ that is adjacent to $v_0$. If $u$ is not in $P$
    then $uP$ is a longer path than $P$, a contradiction. If $u \in P$ then there is a cycle
    $uv_0v_1\dots v_iu$, a contradiction. Therefore, $\deg(v_0) = 1$

    Now consider the graph $G' = G - v_0$. Then, $G'$ is a cycle free graph with $n - 1$
    vertices and $n - 1$ edges. However, we assumed $n$ was the smallest counterexample
    $\contradiction$.

\end{hwproof}

\begin{hwproof}{2.b}{
        Let $G$ be a an $n$-vertex graph, where $n \geq 1$.\\
        (b) Prove that if $G$ is a tree, then $|E(G)| = n - 1$
    }
    We have $G$ is a tree, and thus has no cycles and is connected. Now assume, for the sake
    of contradiction, that $|E(G)| \neq n - 1$. We proved in 2.a that if $G$ has no cycles then
    $|E(G)| \leq n - 1$. Combine this with our initial assumption and we have that
    $|E(G)| < n - 1$.

    However, it is impossible to connect $n$ vertices with $< n - 1$ edges.
    We can prove this easily by induction on the number of edges in a connected component $C$.
    Consider the base case where we have $V(C) = {v_0}$. We have $n = 1$ vertices and $n - 1 = 0$
    edges.
    Now, consider a connected component with $k$
    vertices and $k - 1$ edges, to add a vertex into this connected component, we must also add an edge.
    Therefore we will have $k + 1$ vertices and $k$ edges. Hence, we need at least $n - 1$
    edges to connect $n$ vertices.

    So if $|E(G)| < n - 1$, then $G$ is not connected $\contradiction$

    Thus, if $G$ is a tree, then $|E(G)| = n -1$.
\end{hwproof}

\begin{hwproof}{2.c}{
        Let $G$ be a an $n$-vertex graph, where $n \geq 1$.\\
        Show that $G$ is a tree if and only if for all $u,v \in V(G)$, $G$ contains
        exactly one $uv$-path.
    }
    ($\Rightarrow$)
    We know that $G$ is a tree, and thus has no cycles and exactly $n - 1$ edges.

    Now, for the sake of contradiction, assume there is a $G$ with $u,v \in V(G)$ such that
    there exists more that one $uv$-path in $G$. Call these paths $P_1, P_2$.
    Let $P_1 := u = u_1u_2...u_n=v$ and $P_2 := u = v_1v_2...v_n = v$. Now, since
    $P_1 \neq P_2$ there is some $i, j \in [1,n]$ such that $u_i \neq v_i$ and
    $u_j \neq v_j$ (i.e. the paths "split" at some point). Thus, we can
    create a cycle $u_{i-1}u_iu_{i+1}...u_jv_jv_{j-1}...v_iv_{i-1} = u_{i - 1}\contradiction$.

    ($\Leftarrow$)
    We know that $G$ contains exactly on $uv$-path for all $u,v \in V(G)$. We want to show
    that $G$ is a tree. Assume, for the sake of contradiction, that $G$ is not a tree.
    Then, we could (for example) have $G$ be a 3-cycle, which clearly has more than
    one $uv$-path for all $u,v \in V(G) \contradiction$.

    Therefore we have that $G$ is a tree if and only if for all $u,v \in V(G)$, $G$ contains
    exactly one $uv$-path.
\end{hwproof}

\begin{hwproof}{3}{
        Let $G$ be an $n$-vertex connected simple graph. Run the Bredth-First Search
        on $G$ with root $r \in V(G)$ and let $T$ be the resulting spanning subtree.\\
        Show that for all $v \in V(G)$, $\dist_T(r,v) = \dist_G(r,v)$.
    }
    We will show that $\dist_T(r,v)$ is always the shortest path (i.e. $\dist_G(r,v)$) by
    induction.

    Consider the base case with $v \neq r$ adjacent to $r$. Then at the start of the
    algorithm all the neighbors of $r$ are added to the spanning tree $T$. Meaning that
    $v$ is added to $T$ as a neighbor of $r$ and $\dist_T(r,v) = 1 = \dist_G(r,v)$.

    Now consider $v_k$ such that $\dist_G(r, v_k) = \dist_T(r,v_k)$. We assume strongly that
    for any predecessor $v_{k - i}, 0 < i < k$ that $\dist_T(r, v_{k - i}) = \dist_G(r, v_{k-1})$.
    Now, take some $v_{k + 1} \not \in T$ and adjacent to $v_k$.

    Then, the BFS algorithm will add all $v_{k + 1}$ to $T$, meaning
    that $\dist_T(r, v_{k + 1}) = \dist_T(r, v_k) + 1 = \dist_G(r, v_k) + 1$.
    We know there is not a shorter path because otherwise $v_{k + 1}$ would have been added to $T$
    earlier (our strongly inductive assumption). Therefore, The $rv_{k + 1}$-path is the shortest
    path in $G$.
\end{hwproof}

\begin{hwproof}{4}
    {
        Let $G$ be a $n$-vertex connected simple graph. Run the Depth-First Search
        on $G$ with root $r \in V(G)$ and let $T$ be the resulting spanning subtree.
        By Problem 2(c), we know that $T$ contains exactly one $rv$-path for each vertex
        $v \in V(G)$. Let us denote this path by $P_v$.

        Show that if two vertices $u, v$ are adjacent in $G$, then either $u \in V(P_v)$
        or $v \in V(P_u)$.
    }
    Assume, without loss of generality, that $u$ is added into $T$ before $v$. We also note that when
    vertices are added to the tree they are given a timestamp. Let $i,j$ be the timestamps for
    $u, v$ respectively. Clearly, if $v$ is added to $T$ as a descendant of $u$, or after
    one of $u$'s descendants, we are done since $u$ is in the $rv$-path.

    We will prove that there is no other way that $v$ can be added to $T$. Assume, for the sake
    of contradiction, that $v$ is added otherwise. This implies that $v$ was added after a vertex
    with timestamp lesser than $i$. This implies then that $u,v$ are not adjacent in $G$ $\contradiction$.
\end{hwproof}

\begin{hwproof}
    {5.a}
    {
        Let $G$ be a nonempty finite simple graph. We use
        $\delta(G) := \min_{v \in V(G)}\deg_G(v)$ to denote the \textbf{minimum degree}
        of $G$.

        Let $P$ be a path in $G$ of maximum length.\\
        (a) Show that $|V(P)| \geq \delta(G) + 1$
    }
    Let $v_0\dots v_k$ be a longest path in $G$. Then, all the neighbors of $v_n$ lie on
    this path (otherwise the path would not be a longest one). Therefore,
    $n \geq \deg(v_n) \geq \delta(G)$. So we have that $|E(G)| = n \geq \delta(G)$. Adding one on
    both sides $n + 1 \geq \delta(G) + 1$ which is the same as $|V(G)| \geq \delta(G) + 1$
    since $P$ is a tree.
\end{hwproof}

\begin{hwproof}
    {5.b}
    {
        Let $G$ be a nonempty finite simple graph. We use
        $\delta(G) := \min_{v \in V(G)}\deg_G(v)$ to denote the \textbf{minimum degree}
        of $G$.

        Let $P$ be a path in $G$ of maximum length.\\
        (b) Show that if $\delta(G) \geq 2$, then $G$ contains a cycle of at least
        $\delta(G) + 1$.
    }
    Let $v_0\dots v_n$ be a longest path in $G$. From 5a we know that
    $|E(P)| = n \geq \deg(v_n) \geq \delta(G)$. Now, consider some
    $v_i \in P$ such that $i < n$ is minimal with $v_iv_n \in E(G)$. This implies we have
    some cycle $C := v_iv_{i + 1}\dots v_nv_i$ that is of length at least $\delta(G) + 1$.
\end{hwproof}

\begin{hwproof}
    {6}
    {
        Let $G$ be a nonempty finite simple graph. For $v \in V(G)$ and $p \in \N$, a
        \textbf{flower} with a center $v$ and $p$ petals is a collection of $p$ cycles
        in $G$ any two of which have precisely one vertex in common, namely $v$.
        (Draw a picture of what this looks like to se why we call it a "flower".)

        Suppose that $G$ is $d$-\textbf{regular}, meaning that every vertex in $G$
        has degree $d$. Prove that $G$ contains a flower with $\lfloor d/2 \rfloor$
        petals.

        Hint: Consider a longest path in $G$
    }
    Consider a longest $uv$-path $P = v_0v_1\dots v_k$ in $G$. Then, we have that all
    all of the neighbors of $v_0$ lie inside the path. We also know that $\deg(v_0) = d$. Now,
    pair the neighbors of $v_0$ in $P$ (i.e. $(v_i, v_j), (v_k, v_l), \dots$)
    such that $i < j < k < l$.
    Note that we have $\lfloor d / 2 \rfloor$ pairs. Now, create cycles\\
    $(v_0v_i\dots v_jv_0, v_0v_k\dots v_lv_0, \dots)$. We now have a flower centered at $v_0$ with
    $\lfloor d/2 \rfloor$ petals.
\end{hwproof}


\end{document}