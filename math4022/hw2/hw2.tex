\documentclass{article}
\usepackage[utf8]{inputenc}
\usepackage{amsmath}
\usepackage{amssymb}
\usepackage{amsfonts}
\usepackage{amsthm}
\usepackage{parskip}
\usepackage{bm}

\usepackage{tikz}

\usetikzlibrary{graphs,graphs.standard}

\makeatletter
\pgfmathdeclarefunction{alpha}{1}{%
  \pgfmathint@{#1}%
  \edef\pgfmathresult{\pgffor@alpha{\pgfmathresult}}%
}


\newcommand{\contradiction}{\Rightarrow\!\Leftarrow}
\newcommand{\N}{\mathbb{N}}
\newcommand{\Z}{\mathbb{Z}}
\newcommand{\Q}{\mathbb{Q}}
\newcommand{\R}{\mathbb{R}}
\newcommand{\C}{\mathbb{C}}
\newcommand{\set}[1]{\{#1\}}

\DeclareMathOperator*{\Aut}{Aut}
\DeclareMathOperator*{\aut}{aut}
\DeclareMathOperator*{\emb}{emb}

\newenvironment{hwproof}[2]
{
    \textbf{Problem #1.}\\
    #2
    \begin{proof}
}{
    \end{proof}
    \newpage
}

\title{Homework 2}
\author{Asier Garcia Ruiz}

\begin{document}
\maketitle

I worked on this homework with Armaan Lala and Ryan Hamblet.

\begin{hwproof}
    {1.a}
    {
        The \textbf{maximum degree} of a finite graph $G$ is the quantity
        $\Delta(G) := \max_{v\in V(G)}\deg_G(v)$. A \textbf{leaf} in a tree $T$
        is a vertex of degree 1.

        (a) Prove that every finite tree $T$ has at least $\Delta(T)$ leaves.
    }
    We will prove this by induction on $\Delta(G)$. The base case is trivial,
    since a tree $T$ with only one vertex has $\Delta(T) = 0$ and $0 \geq 0$ leaves.

    Now, assume we have a tree $T$ with $\Delta(T) = k$ and $n \geq k$ leaves. Now, consider
    another tree $T'$ with $n' \geq 1$ leaves. Without loss of generality,
    suppose that $\Delta(T) \geq \Delta(T')$. We draw an edge from $v \in V(T)$ with
    maximum degree to any vertex in $T'$. Now, $\Delta(T) = k + 1$. The number of
    leaves on $T$ is now $n + n' \geq k + 1$. Note that if we connect $T$ to a leaf
    in $T'$ the number of leaves added is still $\geq 1$. This is easy to see if
    $n' \geq 2$ (we are "removing" a leaf but adding $n' - 1 \geq 1$ of them).
    If $n' = 1$ then this node will remain a leaf after being connected to $T$.

    Therefore, every finite tree $T$ has at least $\Delta(T)$ leaves.
\end{hwproof}

\begin{hwproof}
    {1.b}
    {
        The \textbf{maximum degree} of a finite graph $G$ is the quantity
        $\Delta(G) := \max_{v\in V(G)}\deg_G(v)$. A \textbf{leaf} in a tree $T$
        is a vertex of degree 1.

        (b) Conclude that every finite tree on at least 2 vertices has at least
        2 leaves.
    }
    Suppose we have a tree $T$ with at least two vertices, for it to be a tree
    these have to be connected. Thus, $\Delta(T) \geq 1$. Now, if
    $\Delta(T) \geq 2$ then by the previous exercise we know there are at least
    two leaves. If $\Delta(T) = 1$ then we have some $uv$-path, which has two leaves
    (the vertices $u$ and $v$).
\end{hwproof}

\begin{hwproof}
    {2a}
    {
        Let $G$ be a finite simple bipartite graph with a bipartition $(A, B)$.

        (a) Suppose that:
        \begin{itemize}
            \item every vertex in $G$ has at least one neighbor; and
            \item for all $a \in A$ and $b \in B$, if $ab \in E(G)$, then
                  $\deg_G(a) \geq \deg_G(b)$.
        \end{itemize}
        Show that $|A| \leq |B|$.\\
        Hint: Assign a weight $w$ to each edge $ab$ so that for all $a \in A, w(a) = 1$.
    }
    Let $e = ab \in E(G)$ and $w: E(G) \to \R$ such that $w(ab) = \frac{1}{\deg_G(a)}, \forall a \in A$. We know that since $G$
    is bipartite
    \begin{equation}
        \sum_{a \in A} w(a) = \sum_{b \in B} w(b).
    \end{equation}
    The left side is $|A|$ by construction. Now, since $\deg_G(a) \geq \deg_G(b)$
    for all $a \in A, b \in B$, then $w(b) \leq 1, \forall b \in B$. Therefore,
    for (1) to hold, it must be true that $|A| \leq |B|$.
\end{hwproof}

\begin{hwproof}
    {2b}
    {
        Let $G$ be a finite simple bipartite graph with a bipartition $(A, B)$.

        Suppose that:
        \begin{itemize}
            \item every vertex in $G$ is adjacent to at least one but not all the vertices in the
                  opposite side of the bipartition; and
            \item  for all $a \in A$ and $b \in B$, if $ab \not \in E(G)$, then
                  $\deg_G(a) \geq \deg_G(b)$.
        \end{itemize}
        Show that $|A| \leq |B|$.
    }
    Let $e = ab \in E(G)$ and $w: E(G) \to \R$ be defined by
    $w(a) = 1$ for all $a \in V(G)$. then $w(a) = \deg_G(v)$ for any $v \in V(G)$.
    Since $G$ is bipartitie, we also have that
    \begin{equation}
        \sum_{a \in A} w(a) = \sum_{a \in A} \deg(a) = \sum_{b \in B} w(b) = \sum_{b \in B} \deg(b).
    \end{equation}
    We know that for every $b \in B$ there exists an $a \in A$ such that
    $\deg_G(a) > \deg_G(b)$. Therefore, $|A| < |B|$, otherwise the sums would not
    be equal.


\end{hwproof}

\begin{hwproof}
    {3}
    {
        Suppose that $P$ is a finite set of points in the plane of size $|P| \geq 2$. For each
        pair of points in $P$, there is exactly one straight line containing both of them,
        and we let $L$ be the set of all such straight lines. Fig. 1 shows an example with
        5 points and 8 lines.

        Show that $|L| \geq |P|$ unless all the points in $P$ lie on a single straight line.
    }
    Let $A$ be the set of points in the plane and $B$ the set of lines. We draw an
    edge $e = ab, a \in A, b \in B$ if and only if $b$ passes through $a$.

    Now let $G$ be a bipartite graph with $A, B$ as the bipartition. By construction,
    all vertices have at least one neighbor (each line passes through at least two
    points, and we consider all pairs). Let $a\in A, b \in B$ such that
    $\deg_G(a) = k$ be non-adjacent, and let $n_a$ be the number of points lying
    on the line as $a$. Then, $k = |P| - n_a$. Now, since $b$ is not adjacent
    to $a$ (or any point colinear to $a$), then $\deg_G(b) = |P| - 1 - n_a$.

    Therefore, we have that $\deg_G(b) \leq |P| - 1 - n_a < \deg_G(a)$. Hence,
    by problem 2b, $|P| \leq |L|$.
\end{hwproof}

\begin{hwproof}
    {4a}
    {
        Let $k, n$ be positive integers such that $n \geq 2k$. The \textbf{Kneser graph}
        $KG_{k,n}$ is the simple graph whose vertices are the $k$-elements subsets of
        the set $\set{1,\dots,n}$ in which there is an edge between two sets $a,b$
        if and only if $a\cap b = \emptyset$

        (a) How many vertices and edges does $KG_{k,n}$ have?
    }
    There are $\binom{n}{k}$ vertices since there are these many ways of picking $k$ elements
    from a set of size $n$. Now, to find the number of edges, first we find the degree of each
    vertex. That is, we need to find the number of $k$-elements subjects disjoint from a given
    subset $A$. This is the same as the number of $k$-elements subsets in
    $\set{1\dots n} - A$ which has size $n - k$. Therefore, each vertex has degree
    $\binom{n - k}{k}$. Therefore, the number of edges is
    $\frac{1}{2}\binom{n}{k}\binom{n - k}{k}$.
\end{hwproof}

\begin{hwproof}
    {4b}
    {
        Let $k, n$ be positive integers such that $n \geq 2k$. The \textbf{Kneser graph}
        $KG_{k,n}$ is the simple graph whose vertices are the $k$-elements subsets of
        the set $\set{1,\dots,n}$ in which there is an edge between two sets $a,b$
        if and only if $a\cap b = \emptyset$

        (b) Show that $KG_{2,5}$ is isomorphic to the Petersen graph
    }
    First, we will use the labelled version of the Petersen Graph $P$ below:

    \begin{center}
        \begin{tikzpicture}[every node/.style={draw,circle,very thick}]
            \graph [clockwise,math nodes] {
            subgraph C_n [V={a,b,c,d,e},name=A, radius=2cm];
            subgraph I_n [V={f,g,h,i,j},name=B, radius=1cm];
            \foreach \x [evaluate={%
                        \i=alpha(\x);
                        \j=alpha(mod(\x+1,5)+6);
                        \k=alpha(\x+5);}] in {1,...,5}{
                    A \i -- B \k;
                    B \j -- B \k;
                }
            };
        \end{tikzpicture}
    \end{center}

    Now, consider the graph $KG_{2,5}$. This graph has vertices corresponding to
    $\set{1,2}, \set{1,3}, \set{1,4}, \set{1, 5}, \set{2, 3}, \set{2, 4}, \set{2,5}$,
    $\set{3,4}, \set{3,5}, \set{4,5}$.

    Consider the function $f: V(KG_{2,5}) \to V(P)$ such that
    \begin{align*}
        f(\set{1, 2}) & = i \\
        f(\set{1, 3}) & = b \\
        f(\set{1, 4}) & = e \\
        f(\set{1, 5}) & = h \\
        f(\set{2, 3}) & = j \\
        f(\set{2, 4}) & = c \\
        f(\set{2, 5}) & = a \\
        f(\set{3, 4}) & = f \\
        f(\set{3, 5}) & = d \\
        f(\set{4, 5}) & = g \\
    \end{align*}
    This function is clearly bijective and is such that $uv \in E(KG_{2,5}) \iff f(u)f(v)\in E(P)$.
\end{hwproof}

\begin{hwproof}
    {4c}
    {
        Let $k, n$ be positive integers such that $n \geq 2k$. The \textbf{Kneser graph}
        $KG_{k,n}$ is the simple graph whose vertices are the $k$-elements subsets of
        the set $\set{1,\dots,n}$ in which there is an edge between two sets $a,b$
        if and only if $a\cap b = \emptyset$

        (c) Prove that for every pair of vertices $a,b$ of $KG_{k,n}$ there is an automorphism
        $f \in \Aut(KG_{k,n})$ such that $f(a) = b$. (Graphs that have this property are
        called \textbf{vertex-transitive}).
    }
    For brevity, we will denote the general $KG_{k,n}$ as $K$.
    Let $v \in V(K)$ be represented by a $k$-subset in the set $\set{1\dots n}$
    such that $v = \set{n_1, n_2, \dots, n_k}$.
    We the consider a permutation $\pi: \N \to \N$ on $n$ elements. We will define
    a new function $\tau: V(K) \to V(K)$ such that
    \[\tau(v) = \set{\pi(n_1), \pi(n_2),\dots, \pi(n_k)}.\]
    The function $\tau$ is clearly a bijection since $\pi$ is a bijection.

    Now, since $\tau$ is a bijection we have that $\tau(v) = \tau(u)$ if and only
    $v = u$. Now consider two adjacent vertices $u,v \in V(K)$. Then by construction
    of $K$, $u,v$ have no elements in common. Therefore, $\tau(u), \tau(v)$ also
    have no elements in common, and are thus still adjacent. So we have that
    $\tau$ induces an automorphism on $K$ such that $f(u) = v$ for any
    $u, v \in V(G)$.
\end{hwproof}

\begin{hwproof}
    {5a}
    {
        The $n$-dimensional Boolean cube is the simple graph $Q_n$ whose vertices are the
        sequences of 0s and 1s of length $n$, and where there is an edge between two sequences
        $x_1x_2\dots x_n$ and $y_1y_2\dots y_n$ if and only if they differ in precisely
        one coordinate, i.e., if there is a unique index $i \in \set{1,\dots n}$ such that
        $x_i \neq yi$. Fig. 3 shows the graphs $Q_0, Q_1, Q_2$, and $Q_3$.

        (a) How many edges and vertices does $Q_n$ have?
    }
    For each $i \in \set{1\dots n}$ we have two options (1 or 0). Therefore there are
    $2^n$ vertices. We will prove by induction this graph has $2^{n-1}n$ edges.

    The base case is simple, $Q_0$ has $n=0$ and $2^{-1}*0 = 0$ edges. Now consider $Q_k$ and
    assume it has $2^{k - 1}k$ edges. We observe that $Q_{k + 1}$ can be represented as
    two copies of $Q_k$, one with $x_{k + 1} = 0$ and another with $x_{k + 1} = 1$.
    There are $2^n$ edges between these copies (since we connect each vertex with
    $x_{k + 1} = 0$ to it's "copy" with $x_{k + 1} = 1$). Therefore, we have that
    the total number of edges is $2*2^{k-1}k + 2^k = 2^kk + 2^k = 2^k{k + 1}$.

    Therefore, the $n$-dimensional Boolean cube has $2^n$ vertices and $2^{n-1}n$ edges.
\end{hwproof}

\begin{hwproof}
    {5b}
    {
        The $n$-dimensional Boolean cube is the simple graph $Q_n$ whose vertices are the
        sequences of 0s and 1s of length $n$, and where there is an edge between two sequences
        $x_1x_2\dots x_n$ and $y_1y_2\dots y_n$ if and only if they differ in precisely
        one coordinate, i.e., if there is a unique index $i \in \set{1,\dots n}$ such that
        $x_i \neq yi$. Fig. 3 shows the graphs $Q_0, Q_1, Q_2$, and $Q_3$.

        (b) Compute $\aut(Q_n)$
    }
    We pick a vertex $v \in V(Q_n)$. Given that the graph is $2^n$-regular, we see that
    in an automorphism we can place this vertex in any of the $2^n$ points.
    Now, consider this same vertex $v$ with $\deg(v) = n$. We want to place the vertices
    of this edge. There are $n!$ ways of doing this. Since
    we do this for each vertex, we have that $\aut(Q_n) = 2^nn!$

    Consider two automorphism on $Q_n$, $f_1, f_2$, that leave a vertex $v \in V(G)$
    and it's neighbors unchanged. Consider a neighbor of a neighbor of $v$. This
    edge has two elements different from $v$ (since every vertex is a binary string).
    Pick one of the elements that are different and "flip" it, this given us
    a neighbor of $v$. If we "flip" the other element we get a different
    neighbor of $v$. Generally, a vertex $n$ edges away from $v$ must have
    $n$ neighbors with their position unchanged which lock the position of
    the vertex. These edges must remain the same after $f_1^{-1}\circ f_2$.
    Therefore, $f_1^{-1}\circ f_2$ is the identity map and $f_1 = f_2$.
    Thus, labelling a vertex and its neighbors gives rise to a unique
    automorphism. Hence, the number of automorphism of $Q_n$ is $2^nn!$.
\end{hwproof}

\begin{hwproof}
    {6}
    {
        Let $G$ and $H$ be finite simple graphs such that $\hom(F, G) = \hom(F,H)$ for every
        connected finite simple graph $F$. Show that $G \cong H$.
    }
    Let $X$ be a finite graph (may not be
    connected). If $X$ is connected, then we are done (as proven in class).
    Otherwise, we can split up $X$ into it's connected components
    $C_1, C_2, \dots C_k$. We have that
    \[\hom(C_i, G) = \hom(C_i, H), \ \forall i \in [1,k].\]

    Now we also note that
    \[\hom(C_i, G) = \emb(C_i, G) = c_i \ \forall i \in [1,k].\]
    Since all the $C_i$s are independent, we can embed them individually without
    interfering with each other.

    Therefore,
    \[\hom(X, G) = \prod_{i = 1}^k c_i = \hom(X, H).\]
    Thus, for any finite simple graph $X$ the number of embeddings into $G$ and
    $H$ are the same. As shown in class, this implies that $G \cong H$.
\end{hwproof}

\end{document}