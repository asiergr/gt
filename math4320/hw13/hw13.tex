\documentclass{article}
\usepackage[utf8]{inputenc}
\usepackage{amsmath}
\usepackage{amssymb}
\usepackage{amsfonts}
\usepackage{amsthm}
\usepackage{parskip}

\usepackage{graphicx}

\newcommand{\N}{\mathbb{N}}
\newcommand{\Z}{\mathbb{Z}}
\newcommand{\Q}{\mathbb{Q}}
\newcommand{\R}{\mathbb{R}}
\newcommand{\C}{\mathbb{C}}
\newcommand{\zbar}{\overline{z}}
\newcommand{\partiald}[2]{\frac{\partial #1}{\partial #2}}

\DeclareMathOperator*{\Log}{Log}
\DeclareMathOperator*{\Arg}{Arg}
\DeclareMathOperator*{\sech}{sech}
\DeclareMathOperator*{\Res}{Res}

\title{HW 13}
\author{Asier Garcia Ruiz }
\date{August 2021}
\begin{document}
\maketitle

\section*{96}
\subsection*{1} % CORRECT
State why the transformation $w=iz$ is a rotation in the $z$ plane through the
angle $\pi/2$. Then find the image of the infinite strip $0<x<1$.

\begin{proof}
    We can see that for any $z = e^{i\theta}$ we get
    \begin{equation*}
        w = iz = e^{i\pi/2}e^{i\theta} = e^{i(\theta + \pi/2)}
    \end{equation*}
    which is clearly a rotation through the angle $\pi/2$.

    Now we consider the infinite strip $0<x<1$. We can see that in the transformation
    $w = u + iv$ we get $w = iz = -y + ix$. Hence, $u = -y$ and $v = x$.
    Hence, clearly the strip $0<x<1$ gets mapped to $0<v<1$.
\end{proof}

\subsection*{3} %CORRECT 
Find a linear transformation that maps the strip $x>0$, $0<y<2$ onto the strip
$-1<u<1$, $v>0$, as shown in Fig. 117.

\begin{proof}
    We want to achieve a rotation of $\pi/2$ about the origin and then a
    translation of 1 to the right. Hence, our transformation is simply
    $w = iz + 1$.
\end{proof}

\section*{97-98}
\subsection*{4} %CORRECT
Find the image fo the infinite strip $0<y<1/(2c)$ under the transformation
$w = 1/z$. Sketch the strip and its image.

\begin{proof}
    First, we graph the strip

    \includegraphics[scale=0.7]{q4prev}

    Note that the height of the strip is simply $\frac{1}{2c}$.

    We know that in this transformation $y = \frac{-v}{u^2 + v^2}$. Thus,
    the line $y = 0$ gets mapped to $v = 0$. Furthermore, the region
    $y > 0$ gets mapped to
    \begin{gather*}
        \frac{-v}{u^2 + v^2} > 0,\\
        -v > 0,\\
        v < 0.
    \end{gather*}
    Now, the line $y = \frac{1}{2c}$ gets mapped to
    \begin{gather*}
        \frac{-v}{u^2+v^2} = \frac{1}{2c},\\
        -2cv = u^2 + v^2,\\
        u^2 + v^2 + 2vc = 0,\\
        u^2 + (v^2 + 2vc + c^2) - c^2 = 0,\\
        u^2 + (v+c)^2 = c^2.
    \end{gather*}
    Hence, the region $y < \frac{1}{2c}$ gets mapped to
    \begin{gather*}
        \frac{-v}{u^2 + v^2} < \frac{1}{2c},\\
        u^2 + v^2 > -2cv,\\
        u^2 + v^2 + 2cv > 0,\\
        u^2 + (v+c)^2 > c^2.
    \end{gather*}
    Finally, the region $0<y<\frac{1}{2c}$ gets mapped to
    $u^2 + (v+c)^2 > c^2, v < 0$. A graph for this region is

    \includegraphics[scale=0.65]{q4post}

    Note that the radius of the circle is $c$.
\end{proof}

\subsection*{9} %CORRECT
Find the image of the semi-infinite strip $x > 0, 0<y<1$ when $w=\frac{i }{z}$.
Sketch the strip and its image.

\begin{proof}
    We start with the graph of the semi-infinite strip

    \includegraphics[scale=0.5]{q9prev}

    We can get the transformation as a composition of $Z = \frac{1}{z}$ and
    $w = iZ$. Now, we will find how the transformation acts on individual components
    \begin{align*}
        u + iv & = w = \frac{i }{z} = \frac{i }{x + iy},       \\
               & = \frac{y + ix}{(x + iy)(x-iy)},              \\
               & = \frac{y}{x^2 + y^2} + \frac{ix}{x^2 + y^2},
    \end{align*}
    thus
    \begin{equation*}
        u = \frac{y}{x^2+ y^2}, \ v = \frac{x}{x^2 + y^2},
    \end{equation*}
    taking inverses,
    \begin{equation*}
        x = \frac{v }{u^2 + v^2}, \ y = \frac{u }{u^2 + v^2}.
    \end{equation*}

    Then the region $x > 0$ gets mapped to
    \begin{gather*}
        \frac{v }{u^2 + v^2} > 0,\\
        v > 0
    \end{gather*}
    and the region $y < 1$ gets mapped to
    \begin{gather*}
        \frac{u }{u^2 + v^2} < 1,\\
        u < u^2 + v^2,\\
        0 < u^2 + v^2 - u,\\
        0 < -\frac{1}{4} + \frac{1}{4} - u + u^2 + v^2,\\
        \frac{1}{4} < (u - \frac{1}{2})^2 + v^2,
    \end{gather*}
    and the region $y > 0$ gets mapped to
    \begin{gather*}
        \frac{u }{u^2 + v^2} > 0,\\
        u > 0.
    \end{gather*}
    Finally, we get that
    \begin{equation*}
        v > 0, \ u > 0, \ \frac{1}{4} < (u - \frac{1}{2})^2 + v^2.
    \end{equation*}
    The graph for this follows

    \includegraphics[scale=0.4]{q9post}

\end{proof}

\section*{99-100}
\subsection*{4} % CORRECT
Find the bilinear transformation that maps distinct points $z_1, z_2, z_3$ onto the
points $w_1 = 0, w_2 = i, w_3=\infty$.

\begin{proof}
    To solve this we will use the implicit form of the linear fractional
    transformation.
    \begin{equation*}
        \frac{(w-w_1)(w_2-w_3)}{(w-w_3)(w_2-w_1)} = \frac{(z-z_1)(z_2-z_3)}{(z-z_3)(z_2-z_1)}.
    \end{equation*}
    Since one of the points is at infinity we can rewrite the left hand side as
    \begin{equation*}
        \lim_{w_3\to0} \frac{(w-w_1)(w_2-\frac{1}{w_3})}{(w-\frac{1}{w_3})(w_2-w_1)}\frac{w_3}{w_3}
        =\lim_{w_3\to0} \frac{(w-w_1)(w_2w_3 - 1)}{(ww_3 - 1)(w_2-w_1)}
        = \frac{w-w_1}{w_2-w_1}
    \end{equation*}

    Now we can rewrite the first equation
    \begin{align*}
        \frac{w-w_1}{w_2-w_1} & =  \frac{(z-z_1)(z_2-z_3)}{(z-z_3)(z_2-z_1)}, \\
        \intertext{and let $w_1 = 0, w_2 = 1, w_3=\infty$.}
        w                     & = \frac{(z-z_1)(z_2-z_3)}{(z-z_3)(z_2-z_1)},  \\
    \end{align*}
\end{proof}

\subsection*{7} % CORRECT
Find the fixed points (see Exercise 6) of the transformation

(a) $w = \frac{z-1}{z+1}$;
\begin{proof}
    To find the fixed points we simply write
    \begin{gather*}
        f(z_0) = z_0, \\
        \frac{z_0 -1}{z_0+1} = z_0,\\
        z_0 -1 =  z_0^2 + z_0,\\
        z_0^2 = -1,\\
        z_0 = \pm i.
    \end{gather*}
\end{proof}

(b) $w= \frac{6z-9}{z}$.
\begin{proof}
    To find the fixed points we simply write
    \begin{gather*}
        f(z_0) = z_0, \\
        \frac{6z_0-9}{z_0} = z_0, \\
        6z_0 - 9 = z_0^2,\\
        z_0^2 -6z_0 + 9 = 0,\\
        (z_0 -3)^2 = 0,\\
        z_0 = 3.
    \end{gather*}
\end{proof}

\section*{101-102}
\subsection*{3}
(a) By finding the inverse of the transformation
\begin{equation*}
    w = \frac{i-z}{i+z}
\end{equation*}
and appealing to Fig. 13, Appendix 2, whose verification was completed in Exercise 1,
show that the transformation
\begin{equation*}
    w = i\frac{1-z}{1+z}
\end{equation*}
maps the disk $|z| \leq 1$ onto the half plane $\Im(w) \geq 0$

\begin{proof}
    We start by finding the inverse of
    \begin{equation*}
        w = \frac{i-z}{i+z}.
    \end{equation*}
    We write
    \begin{gather*}
        z = \frac{i-w}{i + w},\\
        iz + wz = i - w, \\
        wz + w = i - iz, \\
        w(z + 1) = \frac{i - iz}{z + 1} = i\frac{1-z}{1+z}.
    \end{gather*}
    Because Fig.13 in Appendix 2 has been verified, we then readily have that this
    inverse maps the disk $|z|\leq 1$ to the half plane $\Im(w) \geq 0$.
\end{proof}

(b) Show that the linear fractional transformation
\begin{equation*}
    s = \frac{z-2}{z}
\end{equation*}
can be written
\begin{equation*}
    Z = z - 1, \ W = i\frac{1-Z}{1+Z}, \ w = iW.
\end{equation*}

\begin{proof}
    We start by substituting in $Z$
    \begin{align*}
        W & = i\frac{1-(z-1)}{1 + z-1}, \\
          & = i\frac{2 + z}{z}.
    \end{align*}
    Now we substitute $W$ in
    \begin{align*}
        w & = iW,                             \\
          & = i\left[i\frac{2 - z}{z}\right], \\
          & = \frac{z - 2}{z}.
    \end{align*}
    Hence proven.
\end{proof}

\subsection*{6}
Show that if $\Im(z_0) < 0$, transformation (1), Sec. 102, maps the lower half plane
$\Im(z) \leq 0$ onto the unit disk $|w| \leq 1$.

\begin{proof}
    We have the transformation
    \begin{equation*}
        w = e^{i\alpha}\left(\frac{z-z_0}{z-\overline{z_0}}\right),
    \end{equation*}
    taking norms
    \begin{equation*}
        |w| = \left|e^{i\alpha}\left(\frac{z-z_0}{z-\overline{z_0}}\right)\right|
        =  \frac{|z-z_0|}{|z-\overline{z_0}|}
    \end{equation*}
    and we know that $\Im(z) < 0$. Now, thinking geometrically we know that if
    $z$ is under the real axis, then both it and $z_0$
    are under the real axis. Clearly then $|z - z_0| < |z - \overline{z_0}|$, meaning
    that $|w| < 1$. Now, if $z$ is on the real axis, then $|z - z_0| = |z - \overline{z_0}|$
    and thus $|w| = 1$. Hence, this transformation maps the lower half plane
    $\Im(z) \leq 0$ onto the unit disk $|w| \leq 1$.
\end{proof}

\end{document}