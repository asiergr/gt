\documentclass{article}
\usepackage[utf8]{inputenc}
\usepackage{amsmath}
\usepackage{amssymb}
\usepackage{amsfonts}
\usepackage{amsthm}
\usepackage{parskip}

\newcommand{\N}{\mathbb{N}}
\newcommand{\Z}{\mathbb{Z}}
\newcommand{\Q}{\mathbb{Q}}
\newcommand{\R}{\mathbb{R}}
\newcommand{\C}{\mathbb{C}}
\newcommand{\zbar}{\overline{z}}
\newcommand{\partiald}[2]{\frac{\partial #1}{\partial #2}}

\DeclareMathOperator*{\Log}{Log}
\DeclareMathOperator*{\Arg}{Arg}
\DeclareMathOperator*{\sech}{sech}
\DeclareMathOperator*{\Res}{Res}

\title{HW 11\&12}
\author{Asier Garcia Ruiz }
\date{August 2021}
\begin{document}
\maketitle


\section*{74-77}
\subsection*{2}
Use Cauchy's Residue theorem (Sec.76) to evaluate the integral of each of these
functions around the circle $|z|=3$ in the positive sense:

(a) $\frac{\exp(-z)}{z^2}$
\begin{proof}
    We have that $f(z) = \frac{e^{-z}}{z^2}$ is not analytic at $z = 0$ in the disk.
    Hence, we find the Laurent expansion about this point. This is
    \begin{align*}
        \frac{e^{-z}}{z^2} & = \frac{1}{z^2}\sum_{n=0}^\infty \frac{(-1)^n}{n!}z^n,       \\
                           & = \frac{1}{z^2}\left(1 - z + \frac{1}{2}z^2 - \dots \right), \\
                           & = (\frac{1}{z^2} - \frac{1}{z} + \frac{1}{2} - \dots).
    \end{align*}
    Clearly $\Res_{z=0} f(z) = -1$. Thus,
    \[\int_C f(z) \ dz = -2\pi i\]
\end{proof}

(b) $f(z) = \frac{\exp(-z)}{(z-1)^2}$
\begin{proof}
    We have that $f(z)$ is not analytic only at $z = 1$ in the disk. Hence,
    we find the Laurent expansion about this point. We write
    \begin{align*}
        \frac{e^{-z}}{(z-1)^2} & = \frac{1}{(z-1)^2}
        (\frac{1}{e} - \frac{1}{e}(z-1) + \frac{1}{2e}(z-1)^2 - \frac{1}{6e}(z-1)^3 + \dots),                \\
                               & = \frac{1}{e(z-1)^2} - \frac{1}{e(z-1)} + \frac{1}{2e} - \frac{1}{6e(z-1)}. \\
    \end{align*}
    Clearly, $\Res_{z=1} f(z) = -\frac{1}{e}$. Thus,
    \[\int_C f(z) \ dz = -\frac{2\pi i}{e}\]
\end{proof}

(c) $f(z) = z^2 e^{1/z}$
\begin{proof}
    We have that $f(z)$ is not analytic only at $z=0$ in the disk. The Laurent
    expansion about this point is
    \begin{align*}
        f(z) & = z^2\sum_{n=0}^\infty \frac{1}{k!z^k},                           \\
             & = z^2(1 + \frac{1}{z} + \frac{1}{2z^2} + \frac{1}{6z^3} + \dots), \\
             & = z^2 + z + \frac{1}{2} + \frac{1}{6z} + \dots.
    \end{align*}
    Clearly $\Res_{z=0} f(z) = \frac{1}{6}$. Thus,
    \[\int_C f(z) \ dz = \frac{\pi i}{3}\]
\end{proof}

(d) $f(z) = \frac{z + 1}{z^2 - 2z}$
\begin{proof}
    We have that $f(z)$ is not analytic at $z=0,2$. First, we find the Laurent
    expansion about $z =0$. We write
    \begin{align*}
        \frac{z+1}{z(z-2)} & = \frac{z+1}{z}\frac{1}{-2+z},                  \\
                           & = \frac{z+1}{-2z} \frac{1}{1-z/2},              \\
                           & = (-\frac{1}{2} - \frac{1}{2z})(1 - \frac{z}{2}
        + \frac{z^2}{3} - \dots ).                                           \\
    \end{align*}
    We find that $\Res_{z=0} f(z) = -\frac{1}{2}$.

    Now to find the Laurent expression about $z =2$ we write
    \begin{align*}
        \frac{z+1}{z(z-2)} & =\frac{(z-2) + 3}{z-2} * \frac{1}{2 + (z-2)}, \\
                           & = \frac{1}{2}(1 + \frac{3}{z-2})
        \left[1 - \frac{z-2}{2} + \frac{(z-2)^2}{4} - \dots\right].        \\
    \end{align*}
    We find that $\Res_{z=2} f(z) = \frac{3}{2}$.

    Hence, finally we have that
    \[\int_C f(z) \ dz = 2\pi i(-\frac{1}{2} + \frac{3}{2}) = 2\pi i.\]
\end{proof}

\subsection*{4}
Use theorem in Sec. 77, involving a single residue, to evaluate the integral
of each of these functions around the circle $|z|=2$ in the positive sense.

(a)$f(z) = \frac{z^5}{1-z^3}$
\begin{proof}
    To compute the integral first we will find
    ${\Res_{z=0}[(\frac{1}{z^2})f(\frac{1}{z})]}$. We can easily compute the
    Laurent expansion
    \begin{align*}
        \frac{1}{z^2}f(\frac{1}{z}) & = \frac{1}{z^2}\frac{1}{z^7 - z^4},      \\                                         \\
                                    & = -\frac{1}{z^4}\frac{1}{1-z^3},         \\
                                    & = -\frac{1}{z^4}(1 + z^3 + z^6 + \dots). \\
    \end{align*}
    We find that the residue is $B = -1$, thus
    \[\int_C f(z) \ dz = -2\pi i\]
\end{proof}

(b) $f(z) = \frac{1}{1 + z^2}$.
\begin{proof}
    To compute the integral first we will find
    ${\Res_{z=0}[(\frac{1}{z^2})f(\frac{1}{z})]}$. We can easily compute the
    Laurent expansion
    \begin{align*}
        \frac{1}{z^2}\frac{1}{1 + (1/z)^2} & = \frac{1}{z^2}\frac{1}{1 - (-z^{-2})},                     \\
                                           & = \frac{1}{z^2}(1 - \frac{1}{z^2} + \frac{1}{z^4} - \dots). \\
    \end{align*}
    We find that the residue is $B = 0 $, thus
    \[\int_C f(z) \ dz = 0\]
\end{proof}

(c) $f(z) = \frac{1}{z}$
\begin{proof}
    To compute the integral first we will find
    ${\Res_{z=0}[(\frac{1}{z^2})f(\frac{1}{z})]}$. We can easily compute the
    Laurent expansion
    \begin{align*}
        \frac{1}{z^2}\frac{1}{z^{-1}} & = \frac{1}{z}. \\
    \end{align*}
    Evidently $B = 1$, thus
    \[\int_C f(z) \ dz = 1.\]
\end{proof}


\section*{78-79}
\subsection*{1}
In each case, write the principal part of the function at its isolated singular point
and determine whether that point is a removable singular point, an essential
singular point, or a pole:

(a) $f(z) = ze^{-z}$
\begin{proof}
    We see that $f(z)$ is not analytic at $z = 0$. We find the Laurent expansion
    at this point to
    \begin{align*}
        ze^{-z} & = z(1 - z + \frac{z^2}{2} - \frac{z^3}{6} + \dots).
    \end{align*}
    Clearly, this is an essential singularity
\end{proof}

(b) $f(z) = \frac{z^2}{1 + z}$
\begin{proof}
    We see that $f(z)$ is not analytic at $z = -1$. We find the Laurent expansion
    at this point to
    \begin{align*}
        f(z) & = z + \frac{1}{z + 1} - 1
    \end{align*}
    Thus, this is a pole of order $1$.
\end{proof}

(c) $f(z) = \frac{\sin z}{z}$
\begin{proof}
    We see that $f(z)$ is not analytic at $z = 0$. We find the Laurent expansion
    at this point to be function itself. Hence, this is a pole of order $1$.
\end{proof}

(d) $f(z) = \frac{\cos z}{z}$
\begin{proof}
    We see that $f(z)$ is not analytic at $z = 0$. We find the Laurent expansion
    at this point to be function itself. Hence, this is a pole of order $1$.
\end{proof}

(e) $f(z) = \frac{1}{(2-z)^3}$
\begin{proof}
    We see that $f(z)$ is not analytic when $z = 2$. We find the Laurent expansion
    at this point to be $f(z)$ itself. Hence, this is a pole of order $1$.
\end{proof}

\subsection*{4}
Write a function
\[f(z) = \frac{8a^3z^2}{(z^2 + a^2)^3} \quad (a > 0)\]
as
\[f(z) = \frac{\phi(z)}{(z-ai)^3} \ \text{where} \ \phi(z) = \frac{8a^3z^2}{(z+ai)^3}.\]

Point out why $\phi(z)$ has a Taylor series representation about $z = ai$, and then
use it to show that the principal part of $f$ at that point is
\[\frac{\phi''(ai)/2}{z - ai} + \frac{\phi'(ai)}{(z-ai)^2} + \frac{\phi(ai)}{(z-ai)^3}
    = \frac{i/2}{z - ai} - \frac{a/2}{(z-ai)^2} - \frac{a^2i}{(z-ai)^3}.\]

\begin{proof}
    First we can easily rearrage $f(z)$ as
    \begin{align*}
        f(z) & = \frac{8a^3z^2}{(z^2 + a^2)^3},        \\
             & = \frac{8a^3z^2}{((z - ai)(z + ai))^3}, \\
             & = \frac{8a^3z^2}{(z-ai)^3(z+ai)^3},     \\
             & = \frac{\phi(z)}{(z - ai)^3}.
    \end{align*}
    where $\phi(z) = \frac{8a^3z^2}{(z+ai)^3}$.

    Now, $\phi(z)$ has a Taylor series at $z = ai$ because it is analytic within
    a neighborhood of $z = ai$. Thus we can write the Taylor expansion of $f(z)$
    about $z = ai$ as
    \begin{equation*}
        f(z) = \frac{1}{(z-ai)^3}\left[\phi(ai) + \phi'(ai)(z-ai)
            + \frac{1}{2}\phi''(ai)(z-ai)^3 + \dots\right].
    \end{equation*}
    By simple differentiation
    \begin{equation*}
        \phi'(z) = \frac{16a^4iz-8a^3z^2}{(z+ai)^4},
    \end{equation*}
    and
    \begin{equation*}
        \phi''(z) = \frac{16a^3(z^2-4aiz-a^2)}{(z+ai)^3}.
    \end{equation*}
    Thus
    \begin{equation*}
        \phi(ai) = -a^2i, \phi'(z) = -\frac{a}{2}, \phi''(z) = -i.
    \end{equation*}
    Thus, finally
    \begin{equation*}
        f(z) = \frac{1}{(z-ai)^3}\left[-a^2i - \frac{a}{2}(z-ai) - \frac{i}{2}(z-ai)^2\right],
    \end{equation*}
    giving that the principal part of $f(z)$ is
    \[\frac{i/2}{z - ai} - \frac{a/2}{(z-ai)^2} - \frac{a^2i}{(z-ai)^3}.\]
\end{proof}

\section*{80-81}
\subsection*{1}
In each case, show that any singular point of the function is a pole.
Determine the order $m$ of each pole, and find the corresponding residue $B$.

(a) $f(z) = \frac{z + 1}{z^2 + 9}$
\begin{proof}
    Clearly $f(z)$ has two singular points at $z = \pm 3i$. Since we can write
    \[\frac{z + 1}{(z-3i)(z+3i)},\]
    we can write
    \begin{equation*}
        \phi(z) = \frac{z + 1}{z \pm 3i},
    \end{equation*}
    and
    \begin{equation*}
        f(z) = \frac{\phi(z)}{z \mp 3i},
    \end{equation*}
    we have two poles of order $1$. Hence, the residues are
    \[\phi(\pm 3i) = \frac{3 \pm i}{6}.\]
\end{proof}

(b) $f(z) = \frac{z^2 + 2}{z - 1}$
\begin{proof}
    Clearly $f(z)$ has a singular point at $z = 1$. We let $\phi(z) = z^2 + 2$
    and thus $B = \phi(1) = 3$.
\end{proof}

(c) $f(z) = \frac{z^3}{(2z + 1)^3} = \frac{1}{2^3}\frac{z^3}{(z + 1/2)^3}$
\begin{proof}
    Clearly we have a pole of order $3$ at $z = -\frac{1}{2}$.
    Let $\phi(z) = \frac{z^3}{8}$
    and thus
    \[B = \frac{\phi^{(2)}(-\frac{1}{2})}{2!} = \frac{-3}{16}\]
\end{proof}

(d) $f(z) = \frac{e^z}{z^2 + \pi^2}$
\begin{proof}
    By rewritting $f(z) = \frac{e^z}{(z - \pi i)(z + \pi i)}$ we see we have
    two poles of order $1$ at $z = \pm \pi i$.
    We let $\phi(z) = \frac{e^z}{z \pm \pi i}$ and thus we get that the residuals
    are $B = \phi(z) = \pm \frac{i}{2\pi}$.
\end{proof}

\subsection*{4}
Find the value of the integral
\[\int_C \frac{3z^3 + 2}{(z-1)(z^2 + 9)} \ dz,\]
taken counterclockwise around the circle

(a) $|z - 2| = 2$
\begin{proof}
    In this circle the integrand is not analytic at $z = 1$. Hence,
    we have a pole of order $1$. We can let
    \[\phi(z) = \frac{3z^2+2}{(z^2 + 9)}.\]
    Thus, the residue is $B = \phi(1) = \frac{5}{10}$. Finally we get that
    \[\int_C \frac{3z^3 + 2}{(z-1)(z^2 + 9)} \ dz = 2\pi i \frac{1}{2} = \pi i.\]
\end{proof}

(b) $|z| = 4$
\begin{proof}
    In this circle the integrand is not analytic at $z = 1, \pm 3i$.
    The computation for the residue at $z =1$ is the same as in (a).
    To find the residue at $z = \pm 3i$ we let
    \[\phi(z) = \frac{3z^2 + 2}{(z-1)(z\pm 3i)}.\]
    Hence, we have two poles of order 1. We can find the residues
    $B = \phi(\pm 3i) = \frac{5}{4} \pm \frac{5i}{12}$.

    Finally we get that
    \[\int_C \frac{3z^3 + 2}{(z-1)(z^2 + 9)} \ dz =
        2\pi i \left(\frac{1}{2} + \frac{5}{2}\right) = 6\pi i.\]
\end{proof}

\section*{82-83}
\subsection*{3}
Show that

(a) $\Res_{z=\pi i/2} \frac{\sinh z}{z^2\cosh z} = -\frac{4}{\pi^2}$
\begin{proof}
    We let $\frac{\sinh z}{z^2 \cosh z} = \frac{p(z)}{q(z)}$. Then
    we have that both $p(z),g(z)$ are analytic at ${z_0 = \pi i/2}$. Additionally,
    $p(z_0) = i$, $g(z_0) = 0$, and
    \[g'(z_0) = 2z\cosh z + z^2\sinh(z) \big |_{z_0} = i\frac{-\pi^2}{4}.\]
    Hence,
    \begin{equation*}
        \Res_{z = \pi i/2}\frac{p(z)}{q(z)} = \frac{p(z_0)}{q'(z_0)}
        = \frac{i}{i\frac{-\pi^2}{4}} = -\frac{4}{\pi^2}
    \end{equation*}
\end{proof}

(b) $\Res_{z=\pi i} \frac{e^{zt}}{\sinh z}
    + \Res_{z=-\pi i} \frac{e^{zt}}{\sinh z} = -2\cos(\pi t)$
\begin{proof}
    We let $\frac{e^{zt}}{\sinh z} = \frac{p(z)}{q(z)}$. Then we have that
    both $p(z), g(z)$ are analytic at $z_0 = \pm \pi i$. Furthermore,
    $p(z_0) = e^{\pm \pi i} \neq 0$, $g(z_0) = 0$, and
    \[g'(z_0) = \cosh z \big |_{z_0} = -1\].

    Hence,
    \begin{align*}
        \Res_{z=\pi i} \frac{e^{zt}}{\sinh z}
        + \Res_{z=-\pi i} \frac{e^{zt}}{\sinh z}
         & = \frac{e^{\pi i t}}{-1} + \frac{e^{-\pi i t}}{-1}, \\
         & = -2\cos(\pi t).
    \end{align*}
\end{proof}

\subsection*{7}
Show that
\[\int_C \frac{dz}{(z^2 - 1)^2 + 3} = \frac{\pi}{2\sqrt{2}},\]
where $C$ is the positively oriented boundary of the rectangle whose
sides lie along the lines $x = \pm 2, y = 0$ and $y = 1$.

\begin{proof}
    We let \[f(z) = \frac{1}{(z^2-1)^2 + 3} = \frac{p(z)}{q(z)}\]
    where $p(z) = 1$ and $q(z) = (z^2-1)^2 + 3$. We have by the
    suggestion that the zeros of $q(z)$ are the square roots of
    $1 \pm \sqrt{3}i$ hence
    \[z = \pm\left(\frac{i + \sqrt{3}}{\sqrt{2}}\right),\]
    and
    \[z = \pm\left(\frac{i - \sqrt{3}}{\sqrt{2}}\right).\]

    Of these roots $z_0 = \frac{i + \sqrt{3}}{\sqrt{2}}$ and
    $-\overline{z_0} = \frac{i - \sqrt{3}}{\sqrt{2}}$ are within
    the boundary rectangle.

    Now, we can calculate
    \begin{equation*}
        q'(z) = 2(z^2 - 1)(2z).
    \end{equation*}
    Hence we can evaluate
    \begin{align*}
        q'(z_0) & = 2(\left(\frac{\sqrt{3} + i}{\sqrt{2}}\right)^2 -1)*2\left(\frac{\sqrt{3}+i}{\sqrt{2}}\right), \\
                & = 4\sqrt{3}i\left(\frac{\sqrt{3} + i}{\sqrt{2}}\right) \neq 0,
    \end{align*}
    and
    \begin{align*}
        q'(-\overline{z_0}) & = 2(\left(\frac{-\sqrt{3} + i}{\sqrt{2}}\right)^2 -1)*2\left(\frac{-\sqrt{3}+i}{\sqrt{2}}\right), \\
                            & = -4\sqrt{3}i\left(\frac{i - \sqrt{3}}{\sqrt{2}}\right) \neq 0.
    \end{align*}

    Now by Theorem 2,
    \begin{equation*}
        \Res_{z=z_0} \frac{p(z)}{q(z)} = \frac{p(z_0)}{q'(z_0)}
        = \frac{\sqrt{2}}{4\sqrt{3}i(\sqrt{3} + i)}
    \end{equation*}
    and
    \begin{equation*}
        \Res_{z=-\overline{z_0}} \frac{p(z)}{q(z)} = \frac{p(-\overline{z_0})}{q'(-\overline{z_0})}
        = \frac{-\sqrt{2}}{4\sqrt{3}i(-\sqrt{3} + i)}
    \end{equation*}

    Finally by the Cauchy Residue Theorem we have that
    \begin{align*}
        \int_C f(z) \ dz & = 2\pi i \sum_{k=0}^1 \Res_{z=z_k} f(z),                                                                      \\
                         & = 2\pi i\left[\frac{\sqrt{2}}{4\sqrt{3}i(\sqrt{3} + i)} + \frac{-\sqrt{2}}{4\sqrt{3}i(-\sqrt{3} + i)}\right], \\
                         & = 2\pi i\frac{1}{2\sqrt{2}\sqrt{3}i}\left[\frac{-2\sqrt{3}}{(\sqrt{3} + i)(-\sqrt{3} + i)}\right],            \\
                         & = 2\pi i\frac{-1}{-4\sqrt{2}i},                                                                               \\
                         & = 2\pi i \frac{1}{4\sqrt{2}i},                                                                                \\
                         & = \frac{\pi}{2\sqrt{2}}.
    \end{align*}
\end{proof}


\section*{85-86}
\subsection*{2}
Use residues to derive the integration formula of
\[\int_0^\infty \frac{dx}{(x^2 + 1)^2} = \frac{\pi}{4}.\]

\begin{proof}
    We start by extending the function to the complex plane. Let
    ${f(z) = \frac{1}{(z^2 + 1)^2}}$.
    We observe that $(z^2 + 1)^2 = ((z + i)(z-i))^2 = (z+i)^2(z-i)^2$. Thus,
    the function has two poles of order 2 at $z = \pm i$. Now we will find
    define the contour $C_R$ as a semicircle on the upper half plane with
    radius $R$, and $C$ as the sum of this contour and the line joining
    where it touches the real axis.

    Hence, we have that
    \[\int_C f(z) \ dz = \int_{-R}^R f(x) \ dx + \int_{C_R} f(z) \ dz.\]

    We also have that
    \[\int_C f(z) \ dz  = 2\pi i \Res_{z=i} f(z) = 2\pi i\frac{-i}{4} = \frac{\pi}{2}.\]

    Now we will show that $\int_{C_R} f(z) \ dz$ tends to 0 as $R\rightarrow\infty$.
    We know that $|z^2 + 1| \geq ||z^2| - |1|| = R^2 - 1$. So on any point on $C_R$
    we have that
    \begin{equation*}
        |f(z)| = \frac{1}{(z^2 + 1)} \leq \frac{1}{(R^2 - 1)^2}.
    \end{equation*}
    This means that
    \begin{equation*}
        \left| \int_{C_R} f(z) \ dz \right| \leq \pi R\frac{1}{(R^2 -1)^2}.
    \end{equation*}
    Evidently, this tends to $0$ as $R\rightarrow\infty$.
    Hence, we have that
    \begin{equation*}
        \int_{-R}^R f(x) \ dx = \frac{\pi}{2}.
    \end{equation*}
    as $R\rightarrow\infty$.
    and since the function is even we can write
    \begin{equation*}
        \int_0^\infty f(x) \ dx = \frac{\pi}{4}.
    \end{equation*}
\end{proof}

\subsection*{6}
Use residues to derive the integration formula of
\[\int_0^\infty \frac{x^2 dx}{(x^2 + 9)(x^2 + 4)^2} = \frac{\pi}{200}.\]

\begin{proof}
    We extend the function to the complex plane and let
    \[f(z) = \frac{z^2 dx}{(z^2 + 9)(z^2 + 4)^2}\]. We let $C_R$ be the
    semicircle with radius $R > 3$ centered at the origin, we let $C$ be the
    contour of $C_R$ and the line that joins the two places where it touches
    the real axis.

    Hence, we have that
    \[\int_C f(z) \ dz = \int_{-R}^R f(x) \ dx + \int_{C_R} f(z) \ dz.\]

    Then we can find
    \begin{align*}
        \int_C f(z) \ dz & = 2\pi i\left[\Res_{z=2i} f(z) + \Res_{z=3i} f(z)\right], \\
                         & =2\pi i\left[\frac{-13i}{200} + \frac{3i}{50}\right],     \\
                         & = 2\pi i \frac{-i}{200},                                  \\
                         & = \frac{\pi}{100}.
    \end{align*}
    Now, to show that $\int_{C_R} f(z) \ dz$ goes to zero we consider any $z$
    on $C_R$. We know that $|z^2 + 9| \geq ||z|^2 - 9| = R^2 - 9$ and
    $|z^2 + 4| \geq ||z|^2 - 4| = R^2 - 4$. Hence
    \[|\int_{C_R} f(z) \ dz \leq \frac{\pi R^3}{(R^2 - 9)(R^2 -4)},\]
    which clearly goes to zero as $R$ goes to infinity. Thus, we have that
    \[\int_{-R}^R f (x) \ dx = \int_C f(z) \ dz = \frac{\pi}{100}\]
    considering that the function is even, we let $R$ go to infinity to get
    \[\int_0^\infty f(x) \ dx = \frac{\pi}{200}.\]
\end{proof}

\subsection*{9}
We start by extending the function to the complex plane. We let
$f(z) = \frac{dz}{1 + z^3}$. The contour $C$ will be defined as
in the figure, with $C_1$ being the line on the real axis, $C_R$ with $R>1$,
being the curved line, and $C_2$ being the remaining line. Since we have
that $C = C_1 + C_R + C_2$ then clearly
\begin{equation}
    \int_C f(z) \ dz = \int_{C_1} f(x) \ dx + \int_{C_R} f(z) \ dz + \int_{C_2} f(z) \ dz.
\end{equation}

Clearly the singularities of $f(z)$ are the unit roots of $z^3$ which
can be easily found to be $z = -1, e^{i\pi/3}, e^{-i\pi/3}$. Clearly
only $z_0 = e^{i\pi/3}$ lies within the contour. By the Residue Theorem
we have that
\[\int_C f(z) \ dz = 2\pi i\Res_{z = z_0} f(z)
    = \left(-\frac{1}{6}(1 + \sqrt{3}i)\right) = \frac{\pi}{3}(\sqrt{3} - i). \]

Now we will consider the contour integral over $C_R$. We write
\begin{align*}
    |\int_{C_R} \frac{dz}{1 + z^3}| & \leq \int_{C_R} \left|\frac{1}{1 + z^3}\right||dz|,   \\
                                    & \leq \int_{C_R} \left[\frac{1}{|z|^3 - 1}|dz|\right], \\
    = \int_0^{2\pi/3} \frac{R}{R^3 - 1} \ d\theta.
\end{align*}
This clearly goes to zero as $R$ goes to infinity.

Now we consider the integral over $C_2$. Considering the substitution
$z = xe^{\frac{2\pi i}{3}}$, then $dz = e^{\frac{2\pi i}{3}}dx$ we can then write.
\begin{align*}
    \int_{C_2} \frac{dz}{1 + z^3} & = \int_R^0 \frac{\exp(2\pi i/3)}{1 + x^3e^{2\pi i}}, \\
                                  & = -e^{\frac{2\pi i}{3}} \int_0^R \frac{dx}{1 + x^3}
\end{align*}

Now from (1) and letting $R$ go to infinity we have
\begin{gather*}
    \frac{\pi}{3}(\sqrt{3} - i) = \int_0^\infty \frac{dx}{1 + x^3} + 0 - -e^{\frac{2\pi i}{3}} \int_0^R \frac{dx}{1 + x^3},\\
    \frac{\pi}{3}(\sqrt{3} - i) = \left(1 - e^{2\pi i/3}\right)\int_0^\infty \frac{dx}{1 + x^3}, \\
    \frac{\pi}{3}(\sqrt{3} - i) = \left[1 + \frac{1}{2} - \frac{\sqrt{3}}{2}i\right]\int_0^\infty \frac{dx}{1 + x^3}, \\
    \int_0^\infty \frac{dx}{1 + x^3} = \frac{2\pi}{3\sqrt{3}}.
\end{gather*}

\section*{87-88}
\subsection*{2}
\begin{proof}
    We start by considering the function $f(z) = \frac{1}{z^2 + 1}$. Clearly
    this has singularities at $z = i, -i$. We now construct a contour $C = C_R + C_1$
    where $C_R$ is a positively oriented semicircle with radius $R > 1$ and
    $C_1$ is the line along the real axis joining the semicircle. We will now
    consider the function $f(z)e^{iaz}$. By the Residue Thm. we have
    \begin{equation*}
        \int_C f(z)e^{iaz} = 2\pi i \Res_{z=i} f(z)e^{iaz} = 2\pi i \frac{e^{-a}}{2i} = \pi e^{-a}.
    \end{equation*}

    Now consider the line integral along $C_R$. We can write
    \begin{align*}
        \left|\int_{C_R} f(z) e^{iaz} \ dz \right| & \leq \int_{C_R}\left| f(z) e^{iaz} \right| \ dz,
                                                   & \leq \frac{\pi Re^{-ay}}{R^2 - 1},               \\
                                                   & \leq \frac{\pi R}{R^2 - 1},
    \end{align*}
    Which clearly goes to zero as $R$ goes to infinity.

    Now since
    \begin{equation*}
        \int_C f(z)e^{iaz} = \int_{C_R} f(z)e^{iaz} + \int_{C_1} f(z)e^{iaz}.
    \end{equation*}
    We have that as $R$ goes to infinity
    \begin{equation*}
        \int_{-\infty}^\infty \frac{\cos ax}{x^2 + 1} = \pi e^{-a}.
    \end{equation*}
    Since the function is even
    \begin{equation*}
        \int_0^\infty \frac{\cos ax}{x^2 + 1} = \frac{\pi e^{-a}}{2}.
    \end{equation*}
\end{proof}

\subsection*{10}
We will consider
\begin{equation*}
    \int_C \frac{(z+1)e^{iz}}{z^2 + 4z + 5}dz = \int_C f(z) \ dz,
\end{equation*}
where $C = C_R + C_1$ is made up of the semicircle $C_R$ of radius $R>2$ and the
line on the real axis $C_1$ connecting the ends of the semicircle. We can see that
$f(z)$ has singularities at $z = -2 \pm i$ and only $z_0 = -2 + i$ lies inside $C$.
By the residue theorem
\begin{equation*}
    \int_C f(z) \ dz = 2\pi i\Res_{z=z_0} f(z) = 2\pi i \frac{(i -1)e^{i(-2+i)}}{2i}
    = \frac{\pi}{e}(\sin 2 - \cos 2).
\end{equation*}

Now consider the integral along $C_R$, we write
\begin{align*}
    \left|\int_C \frac{(z+1)e^{iz}}{z^2 + 4z + 5}dz\right| & \leq
    \int_C \frac{(|z|+1)e^{iz}}{(|z| - \sqrt{5})^2}dz,                                           \\
                                                           & \leq \frac{R + 1}{(R - \sqrt{5})^2}
\end{align*}
which clearly goes to zero as $R$ goes to infinity.

Finally, since
\begin{equation*}
    \int_C f(z) = \int_{C_R} f(z) + \int_{C_1} f(z).
\end{equation*}
we get that
\begin{equation*}
    \int_{-\infty}^\infty \frac{(x+1)\cos x}{x^2 + 4x + 5}dx = \frac{\pi}{e}(\sin 2 - \cos 2).
\end{equation*}

\section*{89-91}
\subsection*{3}
This homework was way too long.

\section*{92}
\subsection*{1}
\subsection*{4}

\end{document}