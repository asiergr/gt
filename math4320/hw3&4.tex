\documentclass{article}
\usepackage[utf8]{inputenc}
\usepackage{amsmath}
\usepackage{amssymb}
\usepackage{amsfonts}
\usepackage{amsthm}

\newcommand{\N}{\mathbb{N}}
\newcommand{\Z}{\mathbb{Z}}
\newcommand{\Q}{\mathbb{Q}}
\newcommand{\R}{\mathbb{R}}
\newcommand{\C}{\mathbb{C}}
\newcommand{\ra}{\xrightarrow{}}

\title{HW1}
\author{Asier Garcia Ruiz }
\date{August 2021}

\begin{document}
    \maketitle

    \section*{21-24}
    \subsection*{1}
    Use the theorem in Sec. 21 to show that $f'(z)$ does not exist at any point if\\
    (a) $f'(z) = \overline{z} = x - iy$
    \begin{proof}
        We have that $u(x,y) = x$ and $v(x,y) = -y$. Hence we get partial derivatives
        $u_x = 1$, $v_x = 0$, $u_y = 0$, $v_y = -1$. Since $u_x \neq v_y$ anywhere. Then $f'(z)$ does not
        exist at any point.
    \end{proof}
    (b) $f(z) = z - \overline{z} = x - iy - (x - iy) = -2iy$
    \begin{proof}
        We have $u(x,y) = 0$ and $v(x,y) = -2y$. Hence partial derivatives $u_x = 0$ and $v_y = -2$.
        Since $v_x \neq v_y$, $f'(z)$ does not exist anywhere.
    \end{proof}
    (c) $f(z) = 2x = ixy^2$
    \begin{proof}
        We have that $u(x,y) = 2x$ and $v(x,y) = xy^2$. With partial derivatives $u_x = 2$, $v_x = y^2$,
        $u_y = 0$, and $v_y = 2xy$. Since $v_x \neq -u_y$ anywhere except $y = 0$ and $u_x \neq v_y$
        anywhere except $xy = 1$, then $f'(z)$ does not exist anywhere.
    \end{proof}
    (d) $f(z) = e^x e^{-iy} = e^x(\cos y + i\sin(-y)) = e^x(\cos y - i\sin y)$
    \begin{proof}
       We have that $u(x,y) = e^x\cos y$ and $v(x,y) = -e^x\sin y$. Hence we have partial derivatives
       $u_x = e^x\cos y$, $v_x = -e^x\sin y$, $u_y = e^x\sin y$, and $v_y = e^x\cos y$. Since $v_x \neq -u_y$
       ??? 
    \end{proof}
    \subsection*{8}
    Recall (Sec.6) that if $z=x+iy$,then $x = \frac{z + \overline{x}}{2}$ and 
    $y = \frac{z - \overline{z}}{2i}$ By formally applying the chain rule in calculus to a function F(x, y) of 
    two real variables, derive the expression
    \section*{25-26}
    \subsection*{2}
    With the aid of the theorem in Sec. 21, show that each of these functions is nowhere analytic:\\
    (a) $f(z)=xy+iy$
    \begin{proof}
        We have that $u(x,y)=xy$ and $v(x,y)=y$. Hence there are partial derivatives $u_x=y$, $u_y=x$
        , $v_x=0$, $v_y=1$. Since $u_x \neq v_y$ except $y=1$, and $v_x \neq -u_y$ unless $x=0$. Hence the
        only point where $f'(z)$ exists is $(0, 1)$, which is not an open set. Hence $f$ is nowhere analytic.
    \end{proof}
    (b) $f(z)=2xy+ i(x^2+y^2)$
    \begin{proof}
        We have that $u(x,y)=2xy$ and $v(x,y)=x^2+y^2$. We then have partial derivatives $u_x = 2y$,
        $v_x=2x$, $u_y=2x$, and $v_y=2y$. Now since $v_x \neq -u_y$ anywhere then $f'(z)$ does not exists
        anywere. Hence $f$ is analytic nowhere.
    \end{proof}
    (c) $f(z)=e^y e^{ix}=e^y(\cos x + i\sin x)$
    \begin{proof}
        We have that $u(x,y)=e^y\cos x$ and $v(x,y) = e^y\sin x$. Hence we get partial derivatives
        $u_x=-e^y\sin x$, $v_x=e^y\cos x$, $u_y=e^y \cos x$, $v_y=e^y\sin x$. ???
    \end{proof}
    \subsection*{6}
    Use results in Sec. 24 to verify that the function $$g(z) = \ln r = i\theta \qquad (r > 0, 0 < \theta < 2\pi)$$
    is analytic in the indicated domain of definition, with derivative $g'(z) = \frac{1}{z}$
    Then show that the composite function $G(z) = g(z^2 + 1)$ is analytic in the quadrant $x > 0, y > 0$, with derivative
    $$G'(z)=\frac{2z}{z^2 + 1}$$ Suggestion: Observe that $Im(z2 + 1) > 0$ when $x > 0$, $y > 0$.
    \section*{27}
    \subsection*{1}
    \section*{30}
    \subsection*{1}
    \subsection*{6}
    \subsection*{13}
\end{document}