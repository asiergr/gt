\documentclass{article}
\usepackage[utf8]{inputenc}
\usepackage{amsmath}
\usepackage{amssymb}
\usepackage{amsfonts}
\usepackage{amsthm}
\usepackage{parskip}
\usepackage{bm}

\usepackage{permute}

\newcommand{\N}{\mathbb{N}}
\newcommand{\Z}{\mathbb{Z}}
\newcommand{\Q}{\mathbb{Q}}
\newcommand{\R}{\mathbb{R}}
\newcommand{\C}{\mathbb{C}}
\newcommand{\gen}[1]{\left\langle #1 \right\rangle}
\newcommand{\ZmodnZ}[1]{\Z / #1 \Z}
\DeclareMathOperator*{\sgn}{sgn}
\DeclareMathOperator*{\ord}{ord}

\newenvironment{hwproof}[1]
{
    #1
    \begin{proof}
}{
    \end{proof}
}

% Lauritzen
% Chapter 2: 52, 53, 55
% Chapter 3: 3, 4, 5, 8 
\title{HW9}
\author{Asier Garcia Ruiz}

\begin{document}
\maketitle

I worked on this homework with Daniel Yankin and Armaan Lala.
\section*{2}
\subsection*{52}
% Cauchy's thm
\begin{hwproof}
    {
        Let $G$ be a finite group and $p$ a prime number. Prove that $G$ contains
        an element of order $p$ if $p$ divides $|G|$. (Hint: reduce to the
        situation where $G$ is cyclic and of order $p^r$).
    }
    Consider a prime $p$ which divides the order of a group $G$. By the First Sylow
    Theorem we know that $G$ has a subgroup $H$ of order $p$. Now, by Proposition
    2.7.2 this group is isomorphic to the cyclic group $\ZmodnZ{p}$. Therefore,
    the generator of $H$ is an element of order $p$.
\end{hwproof}

\subsection*{53}
\begin{hwproof}
    {
        Prove that a group of order 15 is cyclic.
    }
    Clearly we see that $15=3*5$. Now, by the First Sylow Theorem we know
    that there are Sylow p-subgroups of orders 4 and 5. The Third Sylow Theorem
    tells us that
    \begin{gather*}
        |Syl_3(G)| \in \{1, 5\},
        |Syl_5(G)| \in \{1, 3\}
    \end{gather*}
    and
    \begin{gather*}
        |Syl_3(G)| \equiv 1 \mod 3,
        |Syl_5(G)| \equiv 1 \mod 5.
    \end{gather*}
    Let $P = Syl_3(G)$ and $Q = Syl_5(G)|$.
    We can conclude that $|P| = 1$ and $|Q| = 1$. Since there is
    only one subgroup for each $p$, then by the Second Sylow Theorem they are
    normal subgroups of $G$. Therefore, the product $PQ$ is a subgroup of
    $G$ containing $P$ and $Q$ (Lemma 2.3.6). This implies that
    $PQ = G$ since $|P|$ and $|Q|$ divide 15. Since $P \cap Q$ is a proper
    subgroup of $Q$, ir follows by Thm. 2.2.8 that $P\cap Q = \{e\}$.
    Now, by Lemma 2.8.1 we have an isomorphism from $PQ$ to $G$.
    Therefore, $G \cong \ZmodnZ{5} \times \ZmodnZ{3}$. By the Chinese Remainder
    Thm we finally get that $G \cong \ZmodnZ{15}$. Therefore, $G$ is a
    cyclic group.

\end{hwproof}

\subsection*{55}
\begin{hwproof}
    {
        Compute the number of elements of order 5 in a group of order 20.
    }
    Since $20 = 2^2*5$ we have Sylow p-subgroups of orders 4 and 5. By the
    Third Sylow Theorem we have
    \begin{gather*}
        |Syl_5(G)| \in \{1, 2, 4\},\\
        |Syl_5(G))| \equiv 1 \mod 5.\\
    \end{gather*}
    Therefore, $|Syl_5(G)| =1$. This implies that there are 4 elements of order
    5 in this unique subgroup of order 5.
\end{hwproof}

\section*{3}
\subsection*{3}
\begin{hwproof}
    {
        Let $R$ be a ring. Prove that $0\cdot x = 0$ and $-x = (-1)\cdot x$
        for every $x \in R$.
    }
    Since 0 is the neutral element in the abelian group $(R, +)$ we have that
    \begin{gather*}
        0 + 0 = 0,
        \intertext{Multiply by $x$ on both sides,}
        x(0 + 0) = x\cdot 0,\\
        x\cdot 0 + x\cdot 0 = x\cdot 0,
        \intertext{subtract $x \cdot 0$ on both sides, since every element has
            an additive inverse,}
        x \cdot 0 = 0
    \end{gather*}
    as required.

    Now consider again $x \in R$ and write
    \begin{gather*}
        x - x = 0,\\
        -1(x - x) = 0,\\
        -1\cdot x + (-1)(-x) = 0,\\
        -1 \cdot x = (-1)(-1)(-x),\\
        -1 \cdot x = -x,
    \end{gather*}
    as required.

\end{hwproof}

\subsection*{4}
\begin{hwproof}
    {
        Prove that an ideal $I$ in a ring $R$ is the whole ring if and only
        if $1 \in I$.
    }
    $(\Rightarrow)$
    Consider an ideal $I$ such that $I = R$. Then, $|I| = |R|$ and it must be
    true that $1 \in I$.

    $(\Leftarrow)$.
    Consider an ideal $I$ such that $1 \in I$. Since it is an ideal we have that
    $\lambda x \in I$ for all $\lambda \in R$ and $x \in I$. Since $1 \in I$ then it is true
    that $\lambda \cdot 1 = \lambda \in I$ for all $\lambda \in R$. Therefore,
    all elements of $R$ are contained in $I$. Since $I$ is a subgroup of $R$
    we get that $I = R$.
\end{hwproof}

\subsection*{5}
\begin{hwproof}
    {
        Let $R$ be a ring and $r_1,\dots, r_n \in R$. Prove that the subset
        $\gen{r_1,\dots,r_n} = \{\lambda_1r_1 + \dots + \lambda_nr_n :
            \lambda_1,\dots,\lambda_n \in R\}$ is an ideal in $R$.
    }
    Consider a subset $I = \gen{r_1,\dots, r_n}$, we want to show this subset is
    ideal in $R$. That is, we will show that $\lambda r \in I$ for all
    $\lambda \in R$ and $r \in I$. We can see that
    \begin{align*}
        \lambda r & = \lambda(\lambda_1r_1 + \dots + \lambda_nr_n)      &
        \lambda_1, \dots, \lambda_n \in R.
        \intertext{Now, since multiplication is distributive this yields}
        \lambda r & = \lambda\lambda_1r_1 + \dots + \lambda\lambda_nr_n &
        \lambda_1, \dots, \lambda_n \in R.
    \end{align*}
    Given the closure of the multiplication operation we know
    $\lambda \lambda_i \in R$ for $i = 1,\dots,n$. Therefore, the resulting
    product of $\lambda r \in \gen{r_1,\dots,r_n} = I$ as required.
\end{hwproof}

\subsection*{8}
\begin{hwproof}
    {
        Let $I$ and $J$ be ideals in the ring $R$.\\
        (i) Prove that
        \begin{equation*}
            I\cap J
        \end{equation*}
        is an ideal in $R$.
    }
    Consider $x \in I \cap J$. Since $I$ and $J$ are ideal we have that
    $\lambda x \in I, J$ for all $\lambda \in R$. Therefore,
    $\lambda x \in I \cap J \ \forall \lambda \in R$ and $I \cap J$ is ideal
    as required.
\end{hwproof}

\begin{hwproof}
    {
        (ii) Prove that
        \begin{equation*}
            I + J = \{a + b : a \in I, b \in J\}
        \end{equation*}
        is an ideal in $R$.
    }
    Consider $x \in I + J$, then $x = i + j$ for some $i \in I, j \in J$ where
    $I, J$ are ideal.
    Now, using the distributive property we can write
    \[\lambda x = \lambda(i + j) = \lambda i + \lambda j \in I + J\]
    Since $\lambda i \in I$ and $\lambda j \in J$ for all $\lambda \in R$.
\end{hwproof}

\begin{hwproof}
    {
        (iii) Prove that
        \begin{equation*}
            IJ = \left\{\sum_{i=1}^n a_ib_i : n \geq 1, a_i \in I, b_i \in J\right\}
        \end{equation*}
        is ideal in $R$.
    }
    Consider $x \in IJ$, then we can write $x$ as
    \begin{equation*}
        x = \sum_{i=1}^n a_ib_i : n \geq 1, a_i \in I, b_i \in J.
    \end{equation*}
    We note that since $I, J$ are ideal in $R$ then $\lambda a_i \in I$ and
    $\lambda b_i \in J$ for all $\lambda \in R$. Therefore multiiplying $\lambda x$
    we get
    \begin{equation*}
        \lambda x = \lambda \sum_{i=1}^n a_ib_i =
        \sum_{i=1}^n \lambda a_ib_i =
        \sum_{i=1}^n a_i\lambda b_i   : n \geq 1, a_i \in I, b_i \in J.
    \end{equation*}
    In all of these scenarios we have that $\lambda x \in IJ$ as required.
\end{hwproof}

\begin{hwproof}
    {
        (iv) Prove that $IJ \subseteq I \cap J$. Give an example where
        $IJ \subsetneq I \cap J$.
    }
    Since $I$ is ideal $IJ \subseteq IR \subseteq I$, and since $J$ is ideal
    $IJ \subseteq RJ \subseteq J$. Therefore $IJ \subseteq I \cap J$.

    Unsure of example
    %https://math.stackexchange.com/questions/2489208/products-of-ideals-being-a-subset-of-their-intersection

\end{hwproof}

\begin{hwproof}
    {
        (v) Is $\{ab : a \in I, b \in J\}$ an ideal in $R$?
    }
    Consider $(x,y) \subseteq \R[x,y]$. Now we see that
    $x^2, y^2 \in \{ab : a \in I, b \in J\}$ but
    $x^2 + y^2 \notin \{ab : a \in I, b \in J\}$.

    Therefore, $\{ab : a \in I, b \in J\}$ is not ideal in $R$.

\end{hwproof}

\end{document}