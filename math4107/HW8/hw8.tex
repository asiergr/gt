\documentclass{article}
\usepackage[utf8]{inputenc}
\usepackage{amsmath}
\usepackage{amssymb}
\usepackage{amsfonts}
\usepackage{amsthm}
\usepackage{parskip}
\usepackage{bm}

\usepackage{permute}

\newcommand{\N}{\mathbb{N}}
\newcommand{\Z}{\mathbb{Z}}
\newcommand{\Q}{\mathbb{Q}}
\newcommand{\R}{\mathbb{R}}
\newcommand{\C}{\mathbb{C}}
\newcommand{\gen}[1]{\left\langle #1 \right\rangle}
\DeclareMathOperator*{\sgn}{sgn}
\DeclareMathOperator*{\ord}{ord}

\newenvironment{hwproof}[1]
{
    #1
    \begin{proof}
}{
    \end{proof}
}

% 
\title{HW8}
\author{Asier Garcia Ruiz}

\begin{document}
\maketitle

I worked on this homework with Daniel Yankin and Armaan Lala.
\section*{Chapter 2}
\subsection*{Ex. 49}
Consider the permutations $\sigma_1 = (1)(2)(345), \sigma_2 = (3)(4)(152)$ and
$\tau = (13)(245)$ in $S_5$.

\begin{hwproof}
    {
        \textbf{(i)} What is the minimal number of simple transpositions needed in
        writing $\tau$ as a product of simple transpositions?
    }
    We can rewrite $\tau$ in two-line notation as
    \begin{equation*}
        \tau = \begin{pmatrix}
            1 & 2 & 3 & 4 & 5 \\
            3 & 4 & 1 & 5 & 2
        \end{pmatrix}.
    \end{equation*}

    We can see that $I_\tau = \{(1,3), (1,5) (2,3), (2,5), (4,5))\}$
    thus $|I_\tau | = 5$ and it takes 5 simple transpositions to write $\tau$ as
    as product of simple transpositions.
\end{hwproof}

\begin{hwproof}
    {
        \textbf{(ii)} Show that $\tau \not \in A_5$ and that
        \begin{equation*}
            \tau \sigma_1 \tau^{-1} = \sigma_2.
        \end{equation*}
    }
    Since $|I_\tau | = 5$ then $\sgn(\tau) = (-1)^5 = -1$. Thus, $\tau$
    is odd and $\tau \not \in A_5$.

    We have that
    \begin{equation*}
        \tau^{-1} = (13)(254)
    \end{equation*}
    \begin{align*}
        (13)(245)(345)(13)(254) & = \pmt{(13)(245)(345)}(13)(254), \\
                                & = \pmt{(135)(24)(13)}(254),      \\
                                & = \pmt{(15)(24)(254)},           \\
                                & = \sigma_2.
    \end{align*}
\end{hwproof}

\begin{hwproof}
    {
        \textbf{(iii)} Show that $\sigma_1, \sigma_2 \in A_5, \tau_1 = (34)\tau \in A_5$ and
        $\tau_1 \sigma_1 \tau^{-1} = \sigma_2$.
    }
    We have that $I_{\sigma_1} = \{(3,5),(4,5)\}$ and
    $I_{\sigma_2} = \{(1,2)(1,3),(1,4),(1,5), (3,5),(4,5)\}$. Thus,
    $| I_{\sigma_1} | = 2$ and $| I_{\sigma_2} | = 6$ so $\sgn(\sigma_1) = 1$
    and $\sgn(\sigma_2) = 1$. Therfore $\sigma_1, \sigma_2 \in A_5$.

    Since $\sgn$ is a homomorphism and $\sgn(34) = -1$ then
    \begin{equation*}
        \sgn(\tau_1) = \sgn((34)\tau) = \sgn(34)\sgn(\tau) = 1.
    \end{equation*}

    We have that
    \begin{equation*}
        \tau_1^{-1} = \pmt{(34)(13)(245)(13254)}.
    \end{equation*}
    Thus,
    \begin{align*}
        (34)(13)(245)(345)(13254) & = \pmt{(34)(13)(245)}(345)(13254), \\
                                  & = \pmt{(14523)(345)}(13254),       \\
                                  & = \pmt{(14235)(13254)},            \\
                                  & = \sigma_2.
    \end{align*}
\end{hwproof}

\begin{hwproof}
    {
        \textbf{(iv)} Now we know that $\sigma_1, \sigma_2$ are conjugate via
        permutation $\tau_1$ in $A_5$. Show that a permutation cycle type
        (a) $1 \leq 1 \leq 1 \leq 1 \leq 1$,
        (b) $1 \leq 2 \leq 2$, (c) $1 \leq 1 \leq 3$ or (d) 5 is even. We know that
        permutations of the same cycle type are conjugate via a permutation
        in $S_5$. Show that two permutations with the same cycle type, (a), (b)
        or (c), are conjugate via a permutation in $A_5$.
    }
    Clearly we see that a permutation cycle type (a) is simply the identity, and
    thus even.

    We know, generally, that the order of a permuation is the least common
    multiple of the the cycle type orders. Therefore we have that the cycle types
    of (b), (c) and (d) are even.

    By Lemma 2.9.8 we can see that two permutations with the same cycle are
    conjugate bia a permuation in $A_5$.


\end{hwproof}

\begin{hwproof}
    {
        Give an example of two 5-cycles that cannot be conjugate via a
        permutation in $A_5$.
    }
    Two 5-cycles that cannot be conjugate via a permutations in $A_5$ are
    $(12345), (13452)$.
\end{hwproof}


\subsection*{Ex. 50}
\begin{hwproof}
    {
        Let $G$ be a group. Prove that the center $Z(G)$ of the group is an abelian
        normal subgroup of $G$.
    }
    We will first prove that $Z(G)$ is a subgroup. We have that
    \begin{equation*}
        Z(G) = \{z \in G : zg = gz, \forall g \in G\}.
    \end{equation*}

    \textbf{(Identity)} We have that $eg = ge \ \forall g \in G$ so $e \in Z(G)$.

    \textbf{(Inverse)} Let $z \in Z(G)$, we can write
    \begin{gather*}
        zg = gz,\\
        z^{-1}zgz^{-1} = z^{-1}gzz^{-1}, \\
        gz^{-1} = z^{-1}g.
    \end{gather*}
    Thus, $z^{-1} \in Z(G)$.

    \textbf{(Closure)} Consider $z_1, z_2 \in Z(G)$. Then fixing a $g\in G$
    \begin{align*}
        z_1 z_2 & = gz_1 g^{-1}gz_2 g^{-1}, \\
                & = gz_1z_2 g^{-1}.         \\
    \end{align*}
    Therefore, $z_1z_2 \in Z(G)$.

    We have shown that $Z(G)$ is a subgroup, now we will show it is abelian.
    This is easy since $Z(G)$ is simply the set of elements in $G$ that
    commute with every element in $G$. Since any $z\in Z(G)$ is also an
    element of $G$, then they must commute. Therefore, $Z(G)$ is abelian.

    Furthermore, we know any abelian subgroup is normal. Therefore we have that
    $Z(G)$ is a normal subgroup of $G$ as required.

\end{hwproof}

\section*{Out of textbook}
\subsection*{1.}
\begin{hwproof}
    {
        Prove that $S_4$ can be generated by the two 4-cycles $(1234)$ and $(1243)$,
        i.e., that $S_4 = \gen{(1234)(1243)}$.
    }
    We know all 4-cycles are odd with order 3. Thus,
    \begin{equation*}
        |\gen{(1 2 3 4)}| = |\gen{(1 2 4 3)}| = 4.
    \end{equation*}
    Let $\gen{\gen{(1234)} \cup \gen{(1243)}}$
    By simple computation we can further find that
    \begin{equation*}
        |H| \geq 7.
    \end{equation*}

    We know that $|S_4| = 24$ and thus by Lagrange's Thm. any subgroup must
    divide 24. Now we observe that
    \begin{equation*}
        (1234)(1243) = (132)
    \end{equation*}
    and
    \begin{equation*}
        (1243)(1234) = (142).
    \end{equation*}
    Both of these are 3-cycles not included in $\gen{(1234)}$ or
    $\gen{(1243)}$. Therefore, we have that $H\geq 9$. Furthermore, we know that
    $A_4$ is the only subgroup of order 12 of $S_4$. Clearly $H\neq A_4$
    (since it contains odd elements). Therefore, $|H| = 24$ and $H = S_4$.
\end{hwproof}

\subsection*{2.}
\begin{hwproof}
    {
        Prove or disprove : $S_4$ can be generated by two 3-cycles.
    }
    We know all 3-cycles are even.
    Consider two 3-cycles $\sigma, \tau$. Then any permutation generated by these
    will be of the form
    \begin{equation*}
        \sigma^n\tau^m \ \text{or} \ \tau^m\sigma^n,
    \end{equation*}
    where $m,n \in \Z$. Now, we know that $\sgn$ is a homomorphism so WLOG
    \begin{align*}
        \sgn(\sigma^n\tau^m) & = \sgn(\sigma^n)\sgn(\tau^m),                                                                                         \\
                             & = \underbrace{\sgn(\sigma)\dots\sgn(\sigma)}_{n \text{times}}\underbrace{\sgn(\tau)\dots\sgn(\tau)}_{m \text{times}}, \\
                             & = 1^n*1^m = 1.
    \end{align*}
    Therefore, any permutation generated by two 3-cycles will be even, and thus
    two 3-cycles cannot generate $S_4$ (since it contains odd permutations).

\end{hwproof}

\subsection*{3.}
\begin{hwproof}
    {
        Consider the group action of $G=S_4$ on $M_4 = \{1,2,3,4\}$.
        \textbf{(a)} Find $G_4$, the stabilizer of 4. What familiar group is this
        isomorphic to?
    }
    We can find
    \begin{equation*}
        G_4 = \{g \in G : gX = X\} = \{(n_1 n_2 n_3) : n_1, n_2, n_3 \neq 4\}.
    \end{equation*}
    We can see by inspection that this group is isomorphic to $S_3$.
\end{hwproof}

\begin{hwproof}
    {
        \textbf{(b)}. Find $G*4$, the orbit of 4.
    }
    We find that
    \begin{equation*}
        G*4 = \{\sigma 4 : g \in G\} = \{1,2,3,4\}.
    \end{equation*}
\end{hwproof}

\begin{hwproof}
    {
        \textbf{(c)} Write out the bijection $\tilde{f}: G / G_4 \to G*4$ given
        by Proposition 2.10.5 (3) (the orbit-stabilizer lemma).
    }
    The orbit-stabilizer theorem tells us there is an isomorphism
    \begin{equation*}
        \hat{f}(gG_4) = g*4.
    \end{equation*}
    We can explicitly write the isomorphism as
    \begin{gather*}
        \hat{f}((1 4)G_4) = 1,\\
        \hat{f}((2 4)G_4) = 2,\\
        \hat{f}((3 4)G_4) = 3,\\
        \hat{f}(idG_4) = 4.
    \end{gather*}

\end{hwproof}

\end{document}