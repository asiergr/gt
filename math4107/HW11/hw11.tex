\documentclass{article}
\usepackage[utf8]{inputenc}
\usepackage{amsmath}
\usepackage{amssymb}
\usepackage{amsfonts}
\usepackage{amsthm}
\usepackage{parskip}
\usepackage{bm}

\usepackage{permute}

\newcommand{\N}{\mathbb{N}}
\newcommand{\Z}{\mathbb{Z}}
\newcommand{\Q}{\mathbb{Q}}
\newcommand{\R}{\mathbb{R}}
\newcommand{\C}{\mathbb{C}}
\newcommand{\gen}[1]{\left\langle #1 \right\rangle}
\newcommand{\ZmodnZ}[1]{\Z / #1 \Z}
\DeclareMathOperator*{\sgn}{sgn}
\DeclareMathOperator*{\ord}{ord}
\DeclareMathOperator*{\rchar}{char}

\newenvironment{hwproof}[1]
{
    #1
    \begin{proof}
}{
    \end{proof}
}

% Lauritzen Chapter 3: 24, 25, 26, 27, 31
\title{HW11}
\author{Asier Garcia Ruiz}

\begin{document}
\maketitle
\section*{Chapter 3}
\subsection*{24}
\begin{hwproof}
    {
        What is the fraction field of a field?
    }
    The fraction field of a field is the field itself.
\end{hwproof}

\subsection*{25}
\begin{hwproof}
    {
        Prove that every ideal in the quotient ring $R/I$ of a principal ideal
        domain $R$ is principal. Give an example of a ring which is not a
        domain but for which every ideal is a principal ideal.
    }
    We know $R$ is a principal ideal domain. Therefore, for any ideal $I$ in $R$
    there exists $d \in R$ such that $I = <d>$. Now, let $I$ be an ideal of $R$, 
    and let $H$ be an ideal in $R/I$. 
    
    Consider the set $J = \{a \in R : [a] \in H\}$. Because $H$ is an ideal
    we have that $[0] \in H$, so $0 \in J$. Now consider $a,b \in J$, then
    $[a], [b] \in H$. Since $H$ is an ideal we must have that
    $[a]+[b] = [a+b] \in J$, so $a + b \in J$. Finally, Consider $a \in J$,
    then $[a] \in H$. Then, since $a - a = 0 \in I$, we have that
    $[a] = [-a]$, so $[-a] \in H$ and thus $-a \in J$. Therefore,
    $J$ is a subgroup of $R$.

    Now, we will show that $H = \gen{[j]}$ for some $j \in R$ such that 
    $J = \gen{j}$ (since $R$ is a PID). Clearly, $[j] \subseteq H$.
    Now, if $[a] \in H$, then $a \in J$, so $a = rj$ for some $r \in R$.
    Therefore, $[a] = [r][j]$, so $[a]\in \gen{[j]}$.
    Therefore, $H$ is principal and $R/I$ is a PID.
\end{hwproof}

\subsection*{26}
\begin{hwproof}
    {
        What are the units in $\ZmodnZ{8}$? Give an example of a ring $R$ with
        and element $x \neq 0,1$ such that $x^2 = x$. Is $R$ a domain? Suppose
        that every $x \in R$ satisfies $x^2 = x$. Show that $\rchar R = 2$.
    }
    We start by finding that the ideals in $\ZmodnZ{8}$ are
    \begin{equation*}
        \{[1], [3], [5], [7]\}.
    \end{equation*}

    A ring $R$ with an element $x \neq 0,1$ such that $x^2 = x$ is the ring
    $\ZmodnZ{2} \times \ZmodnZ{2}$ since $([0],[1])^2 = ([0],[1])$ and in this
    ring $0 = ([0],[0])$ and $1 = ([1],[1])$.

    Now suppose that every $x\in R$ satisfies the property $x^2 = x$. Then, we
    have that
    \begin{align*}
        1 + 1 & = (1 + 1)^2,             \\
              & = 1^2 + 1^2 + 1^2 + 1^2, \\
              & = 1 + 1 + 1 + 1,
    \end{align*}
    and thus $1 + 1 = 0$ and $\rchar R = 2$.
\end{hwproof}

\subsection*{27}
\begin{hwproof}
    {
        Let $f(z) = \bar{z}$ denote the conjugation map for a complex number
        $z \in \C$. Prove that $f$ is a ring homomorphism $\Z[i] \to \Z[i]$
        and that $f(\pi)$ is a prime element if $\pi \in \Z[i]$ is a prime
        element.
    }
    For $f$ to be a ring homomorphism we must show three things

    \textbf{Group homomorphism.}
    To show that $f$ is a ring homomorphism we must first show it is a group
    homomorphism. Consider $x, y \in \Z[i]$ then we have that
    \begin{equation*}
        f(x + y) = \overline{x + y} = \overline{x} + \overline{y} = f(x) + f(y).
    \end{equation*}

    \textbf{Property 2}
    We must now show that $f(xy) = f(x)f(y), \ \forall x,y \in \Z[i]$. So
    we consider $x, y \in \Z[i]$ and we can see that
    \begin{equation*}
        f(xy) = \overline{xy} = \bar{x}\bar{y}.
    \end{equation*}

    \textbf{Multiplicative identities match up.}
    We can clearly see that $f(1) = \overline{1} = 1$.

    Therefore, this is ring homomorphism as required.
    
    Now consider a prime $\pi$ in $\Z[i]$, then $\pi$ is also irreducible.
    Assume that $\pi = ab$, then $f(\pi) = f(a)f(b)$. This means that
    $f(a) = p$ or $f(b) = p$. So $\pi$ is irreducible.
    We also know that in $\Z[i]$ every irreducible is prime (since it is a
    Euclidean field). Therefore $p$ is prime.

\end{hwproof}

\subsection*{31}
\begin{hwproof}
    {
        Is $\Z[\sqrt{-3}] = \{x + y\sqrt{-3}: x,y \in \Z\}$ a euclidean ring?
    }
    Assume, for the sake of contradiction, that $\Z[\sqrt{-3}]$ is a euclidean
    ring. Then, it is also a UFD. However, we have that
    \begin{equation*}
        4 = (2)(2) = (1 - \sqrt{-3})(1 + \sqrt{-3}).
    \end{equation*}

    We can also see that 2 and $1 \pm \sqrt{-3}$ are irreducible and not
    associated. This leads to a contradiction, and thus $\Z[\sqrt{-3}]$ is
    not a Euclidean ring.
\end{hwproof}

\end{document}