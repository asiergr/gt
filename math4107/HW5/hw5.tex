\documentclass{article}
\usepackage[utf8]{inputenc}
\usepackage{amsmath}
\usepackage{amssymb}
\usepackage{amsfonts}
\usepackage{amsthm}
\usepackage{parskip}
\usepackage{bm}
\usepackage{graphicx}

\newcommand{\N}{\mathbb{N}}
\newcommand{\Z}{\mathbb{Z}}
\newcommand{\Q}{\mathbb{Q}}
\newcommand{\R}{\mathbb{R}}
\newcommand{\C}{\mathbb{C}}
\newcommand{\gen}[1]{\left\langle #1 \right\rangle}

\newenvironment{hwproof}[1]
{
    #1
    \begin{proof}
}{
    \end{proof}
}

\title{HW5}
\author{Asier Garcia Ruiz}

\begin{document}
\maketitle

% Lauritzen Chapter 2, Exercises 13, 17, 19, 20, 21, 27, 28
\section*{Chapter 2}
\subsection*{13}
\begin{hwproof}
    {
        Let $N$ be a normal subgroup of a group $G$. Prove that $gN = Ng$ for every
        $g \in G$.
    }

    Consider $g \in G$, we know that $N$ is a normal subgroup and thus
    \begin{equation*}
        gNg^{-1} = \{gng^{-1}: n \in N\} = N.
    \end{equation*}

    $(\subseteq)$ Consider $gn \in gN$ for some $n \in N$. We know that
    $gng^{-1} \in N$, so there exists $n'$ such that $gng^{-1} = n'$ and thus
    $gn = n'g \in Ng$.

    $(\supseteq)$ Consider $ng \in Ng$ for some $n \in N$. Then using what we
    just proved we have that $g^{-1}n \in g^{-1}N \subseteq Ng^{-1}$ so there exists
    some $n' \in N$ such that $g^{-1}n = n'g^{-1}$. Multiplying by $g$ from both
    sides we get $ng = gn' \in gN$ and thus $Ng \subseteq gN$.

    Therefore, we have that $gN = Ng$ when $N$ is a normal subgroup.
\end{hwproof}

\subsection*{17}
%https://math.stackexchange.com/questions/1679174/every-subgroup-of-the-quaternion-group-is-normal 
\begin{hwproof}
    {
        Prove that the quaternion group $H$ from Exercise 2.16 is not abelian, but
        that all its subgroups are normal.
    }

    We can easily show that it is not abelian since $ij = k$ but $ji = -k$.
    Therefore $ij \neq ji$ and the group is not abelian.

    We see the subgroups of the quaternion group $Q_8$ are
    $\gen{i} = \{1,i,-1,-i\}$, $\gen{j}=\{1,j,-1,-j\}$, $\gen{k} = \{1,k,-1,-k\}$,
    $\gen{-1} = \{1,-1\}$, and the trivial subgroup $\{1\}$. Trivially,
    $\{1\}$ is normal. It is also easy to see that $\gen{-1}$ is normal
    because $-1, 1$ commute with every element in $Q_8$. Also,
    $g(-i)g^{-1} = -gig^{-1}$ so $g(-i)g^{-1} \in \gen{i}$ as soon as
    $gig^{-1} \in \gen{i}$.

    This is easy to see when $g = 1,-1,i,-i$. Doing the computations for $j$
    \begin{equation*}
        jij^{-1} = ji(-j) = -j(ij) = -jk = -i,
    \end{equation*}
    and for $g = k$
    \begin{equation*}
        kik^{-1} ki(-k) = -(ki)k = -jk = -i.
    \end{equation*}
    Thus, $\gen{i}$ is normal

    The pattern is the same for $\gen{j}$ and $\gen{k}$ and so all subgroups
    of $Q_8$ are normal.
\end{hwproof}

\subsection*{19}
\begin{hwproof}
    {
        (i) Compute the inverse of [3] in $(\Z / 8\Z)*$.

        (ii) Compute the inverse of [5] in $(\Z / 13\Z)*$.
    }

    (i) Using the extended Euclidean algorithm we get that
    $8*2 - 3*5 = 1$ and thus $[3]^{-1} = [3]$. We can verify by checking
    that $[3][3] = [9] = [1]$.

    (ii) Using the extended Euclidean algorithm we get that
    $13 * 2 - 5 * 5 = 1$ and thus $[5]^{-1} = [8]$. We can verify this by checking
    $[5][8] = [40] = [1]$.
\end{hwproof}

\subsection*{20}
\begin{hwproof}
    {
        Prove that the inverse map of group isomorphism is also a group
        homomorphism.
    }
    Consider a group isomorphism $f: G \to H$. We know that it is also a
    bijective homomorphism and so we consider the inverse mapping
    $h: H \to G$ such that $f\circ h = \text{id}_H$ and
    $h\circ f = \text{id}_G$. Consider $a,b \in H$, we write
    \begin{align*}
        f(h(a)h(b)) & = f(h(a))f(h(b)), \qquad (\text{by homomorphism}) \\
                    & = ab,                                             \\
                    & = f(h(ab)).
    \end{align*}
    Now, since we know $f$ is injective, we have that $h(a)h(b) = h(ab)$ and thus
    $h$ is a group homomorphism.
\end{hwproof}

\subsection*{21}
\begin{hwproof}
    {
        Prove that $G$ is abelian if and only if the map $f: G \to G$ given by
        $f(g) = g^2$ is a group homomorphism.
    }
    $(\Rightarrow)$ We assume that $G$ is abelian, that is $gh = hg$ for all
    $g,h \in G$. Now consider the map $f: G \to G$ such that $f(g) = g^2$. We
    have that
    \begin{align*}
        f(gh) & = (gh)^2,                          \\
              & = ghgh,                            \\
              & \stackrel{\text{abelian}}{=} gghh, \\
              & = g^2h^2,                          \\
              & = f(g)f(h).                        \\
    \end{align*}

    $(\Leftarrow)$ We assume that $f: G \to G$ given by $f(g) = g^2$ is a
    homomorphism. Then we have that $f(gh) = f(g)f(h)$ for all $g,h \in G$.
    Now we can write
    \begin{gather*}
        f(gh) = f(g)f(h),\\
        (gh)^2 = g^2h^2,\\
        ghgh = gghh,\\
        g^{-1}ghghh^{-1} = g^{-1}gghhh^{-1},\\
        ehge = eghe,\\
        hg = gh.
    \end{gather*}
    Therefore, we have that $G$ is abelian.

\end{hwproof}

\subsection*{27}
\begin{hwproof}
    {
        Let $\text{SL}_2(\R)$ be the subset of $\text{GL}_2(\R)$ (see Example 2.1.10)
        consisting of matrices with determinant 1. Show that $\text{SL}_2(\R)$ is a
        normal subgroup of $\text{GL}_2(\R)$. Use the isomorphism theorem to determine
        the group
        \begin{equation*}
            \text{GL}_2(\R) / \text{SL}_2(\R).
        \end{equation*}
    }

    Consider the function $\det: \text{GL}_2(\R) \to \R^*$. We can clearly see that it is a
    group homomorphism since $\det(AB) = \det(A)\det(B)$. Now, we have that
    \begin{equation*}
        \ker f = \{A \in \text{SL}_2(\R) : \det(A) = 1\} = \text{SL}_2(\R).
    \end{equation*}
    Since the kernel of a function is always a normal subgroup of the domain we
    have that $\text{SL}_2(\R) \triangleleft \text{GL}_2(\R)$.

    By the isomorphism theorem $\text{GL}_2(\R) / \text{SL}_2(\R)$ is isomorphic to
    $\R^*$.
\end{hwproof}

\subsection*{28}
\begin{hwproof}
    {
        Prove that $(\Z / 13\Z)^*$ is a cyclic group by finding a generator.
    }

    We have that
    \begin{equation*}
        (\Z / 13\Z)^* = \{[1],[2],[3],[4],[5],[6],[7],[8],[9],[10],[11],[12]\}.
    \end{equation*}
    Now we see that $[2]^1 = [2]$, $[2]^2 = [4]$, $[2]^3 = [8]$, $[2]^4 = [3]$,
    $[2]^5 = [6]$, $[2]^6 = [12]$, $[2]^7 = [11]$, $[2]^8 = [9]$,
    $[2]^9 = [5]$, $[2]^{10} = [10]$, $[2]^{11} = [7]$, $[2]^{12} = [1]$.
    So clearly $\gen{[2]} = (\Z / 13\Z)^*$ and thus $(\Z / 13\Z)^*$ is a cyclic
    group.
\end{hwproof}
\end{document}