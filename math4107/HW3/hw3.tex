\documentclass{article}
\usepackage[utf8]{inputenc}
\usepackage{amsmath}
\usepackage{amssymb}
\usepackage{amsfonts}
\usepackage{amsthm}
\usepackage{parskip}
\usepackage{bm}
\usepackage{graphicx}

\newcommand{\N}{\mathbb{N}}
\newcommand{\Z}{\mathbb{Z}}
\newcommand{\Q}{\mathbb{Q}}
\newcommand{\R}{\mathbb{R}}
\newcommand{\C}{\mathbb{C}}


\title{HW3}
\author{Asier Garcia Ruiz}

\begin{document}
\maketitle

\section*{Ch. 2 Ex. 1}
Let $G$ be a group and $g \in G$ an element of $G$. Prove that the map $\xi: G\to G$ given by $\xi(x) = xg$ is bijective.

\begin{proof}
    We will start by proving injectivity. Assume that $\xi(x) = \xi(y)$, that is to say
    $xg = yg$. Now, since every $g\in G$ has an inverse $g^{-1} \in G$ we can postmulitply
    both sides by $g^{-1}$ to get $xgg^{-1} = ygg^{-1}$ and thus $x=y$. Hence, $\xi$ is
    injective.

    Now we will prove surjectiveness. To do this we will show that for every
    $y \in G \ \exists x \in G$ such that $\xi(x) = y$. Using again the fact that every
    $g \in G$ has an inverse $g^{-1} \in G$ we now let $x = yg^{-1}$. Therefore, we have that
    $\xi(x) = yg^{-1}g = y$ and $\xi$ spans $G$.

    Hence, $\xi$ is bijecive as needed.
\end{proof}

\section*{1}
Prove or disprove that the following are groups:
\subsection*{(a)}
Even integers under addition.
\begin{proof}
    (\textbf{Closure:})
    If we have two even integers $2n, 2m \in 2\Z$ then addition yields
    $2n + 2m = 2(n + m) \in 2\Z$. Hence addition is a binary operation on the
    even integers.

    (\textbf{Associative:}) We have that $(2n + 2m) + 2s = 2n + (2m + 2s) \
        \forall n,m,s \in \Z$, hence associativity is met integers.

    (\textbf{Identity:}) The identity element is $0\in 2\Z$ since
    $0+ x = x = x + 0 \ \forall x \in 2\Z$.

    (\textbf{Inverse:}) Every $x \in 2\Z$ has an inverse $-x \in 2\Z$ since clearly $x + (-x) = 0$.

    Hence, this is a group.
\end{proof}

\subsection*{(b)}
Odd integers under addition.
\begin{proof}
    This is clearly not a group as there is no identity element. That is, there is no
    $x \in 2\Z + 1$ such that $x' + x = x' = x + x'$. This is because the additive
    identity is $0 \notin 2\Z + 1$.
\end{proof}

\subsection*{(c)}
$(3\Z, +)$ where $3\Z = \{3n: n\in\Z\}$.
\begin{proof}
    (\textbf{Closure:})
    Given $3n, 3m \in 3\Z$ we have that $3n + 3m = 3(n + m) \in 3\Z$. Hence,
    $+$ is a binary operator on $3\Z$

    (\textbf{Associative}) We can clearly see that $(3n + 3m) + 3n = 3n + (3m + 3x) \ \forall n,m,x \in Z$.

    (\textbf{Identity:}) We see that $0 \in 3\Z$ is the identity as $0 + 3n = 3n = 3n + 0, \ \forall n \in 3\Z$.

    (\textbf{Inverse:}) We define $x = 3n$ and $-x = 3(-n), n \in \Z$.
    We have that for every $x \in 3\Z$ there exists an inverse $-x \in 3\Z$
    such that $x + (-x) = 0$.

    Hence, this is a group.
\end{proof}

\subsection*{(d)}
$\{3^n: n \in \Z\}$ under multiplication.
\begin{proof}
    (\textbf{Closure:})
    For any two $3^n, 3^m \in \{3^n: n \in \Z\}$ we have that
    $3^n*3^m = 3^{n+m} \in \{3^n: n \in \Z\}$. Hence multiplication is a binary
    operation on $\{3^n: n \in \Z\}$.

    (\textbf{Associative:})
    We can write
    \[3^x(3^y3^z) = 3^x3^{y+z} = 3^{x+y+z} = 3^{x+y}3^z =  (3^x3^y)3^z.\]
    Thus, associativity is met.

    (\textbf{Identity:})
    We have that $3^0 = 1 \in \{3^n: n \in \Z\}$ is the identity since $3^n * 1 = 3^n = 1*3^n$
    for all $n \in \Z$.

    (\textbf{Inverse:})
    We see that for every $x = 3^n \in \{3^n: n \in \Z\}$ there exists $y = 3^{-n} \in \{3^n: n \in \Z\}$
    such that $x*y = 3^n3^{-n} = 3^{n-n} = 3^0 = 1 = 3^{-n+n} = 3^{-n}3^n = y*x$.

    Hence this is a group.
\end{proof}

\section*{2}
\subsection*{(a)}
$2\times 2$ diagonal matrices under matrix addition.

\begin{proof}
    We will denote these matrices in the general form as as $A = \begin{bmatrix}
            a_{11} & 0      \\
            0      & a_{22}
        \end{bmatrix}$

    (\textbf{Closure:})
    For any two diagonal $2\times 2$ matrices $A,B$ we have that
    \begin{align*}
        A + B & = \begin{bmatrix}
                      a_{11} + b_{11} & 0               \\
                      0               & a_{22} + b_{22}
                  \end{bmatrix}
    \end{align*}
    which is also a diagonal $2\times 2$ matrix.
    Therefore addition is a binary operator on diagonal $2\times 2$ matrices.


    (\textbf{Associative:})
    We have that
    \begin{align*}
        (A + B) + C & = \begin{bmatrix}
                            a_{11} + b_{11} & 0               \\
                            0               & a_{22} + b_{22}
                        \end{bmatrix} + C                   \\
                    & = \begin{bmatrix}
                            a_{11} + b_{11} + c_{11} & 0                        \\
                            0                        & a_{22} + b_{22} + c_{22}
                        \end{bmatrix} \\
                    & = A + \begin{bmatrix}
                                b_{11} + c_{11} & 0               \\
                                0               & b_{22} + c_{22}
                            \end{bmatrix} = A + (B+C).
    \end{align*}
    Hence, it is associative.

    (\textbf{Identity}) We have that the identity is the zero matrix $\bm{0}$ since
    $A + \bm{0} = A = \bm{0} + A$.

    (\textbf{Inverse:}) Any of these matrices $A$ has an inverse $-A = \begin{bmatrix}
            -a_{11} & 0       \\
            0       & -a_{22}
        \end{bmatrix}$ such that $A + (-A) = 0 = -A + A$.

    Therefore, this is a group.
\end{proof}

\subsection*{(b)}
$2\times 2$ matrices with determinant 1 under matrix addition.

\begin{proof}
    This is clearly not a group as there is no identity element. There is no matrix with
    determinant 1 such that $A + E = A = E + A$. The matrix identity under addition
    is $\bm{0}$, but this matrix does have determinant 1.
\end{proof}

\subsection*{(c)}
$2\times 2$ matrices with determinant 1 under matrix multiplication.

\begin{proof}
    (\textbf{Closure:})
    We have that for any $2\times 2$ matrices with determinant 1. We know by linear
    algebra that $\det(A\times B) = \det(A)\det(B)$. Hence, $\det(AB) = 1*1 = 1$
    and multiplication is a binary operation on $2\times 2$ matrices with determinant
    one.

    (\textbf{Associative:}) Since matrix multiplicaiton is associative in general, then this
    condition is met.

    (\textbf{Identity:}) The identity element is $I = \begin{bmatrix}
            1 & 0 \\
            0 & 1
        \end{bmatrix}.$ Since we have that
    \begin{equation}
        A\times I = \begin{bmatrix}
            a_{11} & a_{12} \\
            a_{21} & a_{22}
        \end{bmatrix} \times \begin{bmatrix}
            1 & 0 \\
            0 & 1
        \end{bmatrix} = A = I\times A.
    \end{equation}

    (\textbf{Inverse:}) We know a matrix is singular if and only if it has a determinant of 0.
    Since the matrices in the set all have determinant non-zero, they all have an inverse.

    Hence, this is a group.
\end{proof}

\subsection*{(d)}
$2\times 2$ matrices with trace 0 under matrix multiplication.

\begin{proof}
    We begin by noting that these are matrices of the form $\begin{bmatrix}
            0      & a_{12} \\
            a_{21} & 0
        \end{bmatrix}$.

    Consider an identity element $E = \begin{bmatrix}
            0      & e_{12} \\
            e_{21} & 0
        \end{bmatrix}$. Then we require that $A \times E = A$. Carrying out the computation we get
    \begin{equation*}
        A \times E  = \begin{bmatrix}
            a_{12}*e_{21} & 0             \\
            0             & e_{21}*a_{12}
        \end{bmatrix}.
    \end{equation*}
    Clearly it is impossible to make this equal to $A$. Hence, there is no identity element and this is not
    a group.
\end{proof}

\subsection*{(e)}
$2\times 2$ symmetric matrices with nonzero determinant under matrix multiplication.

\begin{proof}
    These are matrices of the form $A = \begin{bmatrix}
            a_{11} & a_{2}  \\
            a_2    & a_{22}
        \end{bmatrix}$

    (\textbf{Closure:})
    From linear algebra we know that $\det(A\times B) = \det(A)\det(B)$. Thus,
    if $\det(A) \neq 0$ and $\det(B) \neq 0$ then $\det(A\times B) \neq 0$.
    Therefore, matrix multiplication is a binary operation on matrices with
    nonzero determinant.

    (\textbf{Associative:}) Since matrix multiplication is associative by definition, so is
    matrix multiplication of this specific kind of matrix.

    (\textbf{Identity:}) The matrix $I = \begin{bmatrix}
            1 & 0 \\
            0 & 1
        \end{bmatrix}$ is clearly symmetric and also such that $A\times I = A = I \times A$.
    Hence, there is an identity element.

    (\textbf{Inverses:}) Since any matrix with non-zero determinant is invertible, there exists
    $A^{-1}$ such that $A\times A^{-1} = I = A^{-1}\times A$
\end{proof}

\subsection*{(f)}
$2 \times 2$ orthogonal matrices under matrix multiplication.

\begin{proof}

    (\textbf{Closure:})
    For any two orthogonal matrices $A,B$ we have that
    \begin{equation*}
        AB(AB)^T = ABB^TA^T = AIA^T = AA^T = A.
    \end{equation*}
    Therefore matrix multiplication is a binary operation on orthogonal matrices.

    (\textbf{Associative:}) Since matrix multiplication is associative by definition, so is
    matrix multiplication of this specific kind of matrix.

    (\textbf{Identity:}) The matrix $I = \begin{bmatrix}
            1 & 0 \\
            0 & 1
        \end{bmatrix}$ is clearly orthogonal as $I \times I = I$
    and is also such that $A\times I = A = I \times A$.

    (\textbf{Inverse:}) Every orthogonal matrix by definition has an inverse
    $A^{-1} = A^T$ such that $AA^T = I = A^T A$.

    Hence, this is a group.
\end{proof}

\section*{3}
\begin{proof}

    (\textbf{Closure:})
    We can write
    \begin{align*}
        z_k * z_s & = e^{\frac{2ki\pi}{n}}e^{\frac{2si\pi}{n}}, \\
                  & = e^{\frac{2(k + s)i\pi}{n}}.
    \end{align*}
    In the case where $0 \leq k + s < n$ we are done. If we have that
    $k + s > n$ then this is simply a repeated root and we can calculate
    $j = k + s \mod n$ to get a value lesser than $n$. Therefore,
    multiplication is closed on the $n$th roots of unity.

    (\textbf{Associative:}) We can write
    \begin{align*}
        (z_k * z_s) * z_l & = (e^{\frac{2ki\pi }{n_1}} * e^{\frac{2si\pi }{n_2}})
        *e^{\frac{2li\pi }{n_3}}                                                                          \\
                          & = e^{\frac{2ki\pi)}{n_1} + \frac{2si\pi}{n_2}} e^{\frac{2li\pi }{n_3}}        \\
                          & = e^{\frac{2ki\pi}{n_1}+\frac{2si\pi}{n_2}+\frac{2li\pi}{n_3}}                \\
                          & =e^{\frac{2ki\pi }{n_1}} e^{\frac{2si\pi}{n_2} + \frac{2li\pi}{n_3}}          \\
                          & = e^{\frac{2ki\pi }{n_1}} (e^{\frac{2si\pi }{n_2}} * e^{\frac{2li\pi }{n_3}}) \\
                          & = z_k*(z_s*z_l).
    \end{align*}
    Hence, it is associative.

    (\textbf{Identity:}) If we let $k = 0$ then we have that
    \begin{equation*}
        z_0 = e^0 = 1.
    \end{equation*}
    This is clearly the identity element since for any $z_k, 0 \leq k < n$ we have that
    $z_k * 1 = z_k = 1 * z_k$.

    (\textbf{Inverse:}) Consider $z_k = e^{\frac{2ki\pi }{n}}$. Then there exists
    $z_{n-k} = e^{\frac{2(n-k)i \pi }{n}}$ such that
    \begin{equation*}
        z_k*z_{n-k} = e^{\frac{2ki\pi }{n} + \frac{2(n-k)i\pi }{n}}
        = e^{\frac{2i\pi(k + n - k)}{n}} = e^{2i\pi} = 1.
    \end{equation*}
    Therefore, there is an inverse for every $z_k$.

    Thus, this is a group as needed.
\end{proof}

Now we can graph the nth roots of unity for $n = 2,3,4$.
$n=2$

\includegraphics[scale=0.5]{n2}

$n=3$

\includegraphics[scale=0.5]{n3}

$n=4$

\includegraphics[scale=0.5]{n4}
\end{document}