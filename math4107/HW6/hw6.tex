\documentclass{article}
\usepackage[utf8]{inputenc}
\usepackage{amsmath}
\usepackage{amssymb}
\usepackage{amsfonts}
\usepackage{amsthm}
\usepackage{parskip}
\usepackage{bm}
\usepackage{graphicx}

\newcommand{\N}{\mathbb{N}}
\newcommand{\Z}{\mathbb{Z}}
\newcommand{\Q}{\mathbb{Q}}
\newcommand{\R}{\mathbb{R}}
\newcommand{\C}{\mathbb{C}}
\newcommand{\gen}[1]{\left\langle #1 \right\rangle}

\newenvironment{hwproof}[1]
{
    #1
    \begin{proof}
}{
    \end{proof}
}

\title{HW6}
\author{Asier Garcia Ruiz}

\begin{document}
\maketitle

%Lauritzen Chapter 2: 26, 31, 32, 33, 35, 39 
\section*{Chapter 2}
\subsection*{26}
\begin{hwproof}
    {
        Let $G$ be an abelian group, $K$ a group, and $f: G \to K$ a group homomorphism.
        Prove that $f(G) \subseteq K$ is an abelian subgroup of $K$.
    }

    Proposition 2.4.9 tells us that $f(G)$ is a subgroup of $K$. We must only
    prove that $K$ is abelian.

    We know $G$ is abelian and $f$ is a homomorphism. Thus we have that
    \begin{equation*}
        f(ab) = f(ba),
    \end{equation*}
    and
    \begin{equation*}
        f(ab) = f(a)f(b).
    \end{equation*}
    combining these two facts
    \begin{equation*}
        f(ab) = f(a)f(b) = f(ba) = f(b)f(a).
    \end{equation*}
    Therefore
    \begin{equation*}
        f(a)f(b) = f(b)f(a)
    \end{equation*}
    and $f(G) \subseteq K$ is abelian as required.
\end{hwproof}

\subsection*{31}
\begin{hwproof}
    {
        (i) Write down all the elements of order 7 in $\Z / 28\Z$.

        (ii) How many subgroups are there of order 7 in $\Z / 28\Z$?
    }
    (i) The elements of order 7 in $\Z / 28\Z$ are
    $\{[4], [8], [12], [16], [20], [24]\}$

    (ii) By Proposition 2.7.4 we have that since $7|28$ then there exists
    one unique subgroup of order 7.
\end{hwproof}

\subsection*{32}
\begin{hwproof}
    {
        (i) Prove that the cyclic group $\Z / 15\Z$ is isomorphic to the
        product group $\Z / 3\Z \times \Z / 5\Z$.

        (ii) Prove that the group $(\Z / 15\Z)*$ is isomorphic to the product group
        $\Z / 2 \times \Z / 4\Z$. Conclude that $(\Z / 15/Z)*$ is not cyclic.
    }
    (i) We see that $\gcd(3,5) = 1$, so they are relatively prime, and $15=3*5$.
    Thus, by Proposition 2.8.2 we have that
    \begin{equation*}
        \tilde{\varphi}: \Z / 15\Z \to \Z / 3\Z \times \Z / 5\Z
    \end{equation*}
    given by $\varphi(x + N\Z) = (\varphi_1(x), \varphi_2(x))$ is a group isomorphism.
    This implies that $\Z / 15/Z \cong \Z / 3\Z \times \Z / 15/Z$.

    (ii)
\end{hwproof}

\subsection*{33}
\begin{hwproof}
    {
        Consider $\Z \subset \Q$ as abelian groups with $+$ as composition.
        Let $[q] = q + \Z \in \Q / \Z$, where $q \in \Q$.

        (i) Show that $\left[\frac{9}{4}\right]$ has order 4 in $\Q / \Z$.

        (ii) Determine the order of $\left[\frac{a}{b}\right]$ in $\Q / \Z$,
        where $a \in \Z, b \in \N / \{0\}$ and $\gcd(a,b) = 1$. Conclude that
        every element in $\Q / \Z$ has finite order and that there are elements
        in $\Q / \Z$ of arbitrary large order.

        (iii) Show that $\Q / \Z$ is an infinite grou pthat is not cyclic.
    }
    (i) Doing the computations we can see that
    \begin{equation*}
        \frac{9}{4} + \frac{9}{4} + \frac{9}{4} + \frac{9}{4} = 9.
    \end{equation*}


\end{hwproof}

\subsection*{35}
\begin{hwproof}
    {
        Give an example of a non-cyclic group of order 8.
    }
    The quaternion group $Q_4$ is an example of a non-cyclic group of order
    8.
\end{hwproof}

\subsection*{39}
\begin{hwproof}
    {
        Let $\tau \in S_3$ denote the 3-cycle
        \begin{equation*}
            \begin{pmatrix}
                1 & 2 & 3
            \end{pmatrix}
            = \begin{pmatrix}
                1 & 2 & 3 \\
                2 & 3 & 1
            \end{pmatrix}
            .
        \end{equation*}
        Show that the subgroup $\gen{\tau} = \{\tau^n : n \in \Z\}$ is normal
        in $S_3$.
    }

    We start by asserting that
    \begin{equation*}
        \gen{\tau} = \{(), (1 \ 2 \ 3), (3 \ 1 \ 2), (2 \ 3 \ 1)\}.
    \end{equation*}

    We want to show that for any $g \in S_3$
    \begin{equation*}
        gng^{-1} =
    \end{equation*}
\end{hwproof}

\end{document}