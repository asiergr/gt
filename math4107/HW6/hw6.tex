\documentclass{article}
\usepackage[utf8]{inputenc}
\usepackage{amsmath}
\usepackage{amssymb}
\usepackage{amsfonts}
\usepackage{amsthm}
\usepackage{parskip}
\usepackage{bm}
\usepackage{graphicx}

\newcommand{\N}{\mathbb{N}}
\newcommand{\Z}{\mathbb{Z}}
\newcommand{\Q}{\mathbb{Q}}
\newcommand{\R}{\mathbb{R}}
\newcommand{\C}{\mathbb{C}}
\newcommand{\gen}[1]{\left\langle #1 \right\rangle}

\newenvironment{hwproof}[1]
{
    #1
    \begin{proof}
}{
    \end{proof}
}

\title{HW6}
\author{Asier Garcia Ruiz}

\begin{document}
\maketitle

%Lauritzen Chapter 2: 26, 31, 32, 33, 35, 39 
\section*{Chapter 2}
\subsection*{26}
\begin{hwproof}
    {
        Let $G$ be an abelian group, $K$ a group, and $f: G \to K$ a group homomorphism.
        Prove that $f(G) \subseteq K$ is an abelian subgroup of $K$.
    }

    Proposition 2.4.9 tells us that $f(G)$ is a subgroup of $K$. We must only
    prove that $K$ is abelian.

    We know $G$ is abelian and $f$ is a homomorphism. Thus we have that
    \begin{equation*}
        f(ab) = f(ba),
    \end{equation*}
    and
    \begin{equation*}
        f(ab) = f(a)f(b).
    \end{equation*}
    combining these two facts
    \begin{equation*}
        f(ab) = f(a)f(b) = f(ba) = f(b)f(a).
    \end{equation*}
    Therefore
    \begin{equation*}
        f(a)f(b) = f(b)f(a)
    \end{equation*}
    and $f(G) \subseteq K$ is abelian as required.
\end{hwproof}

\subsection*{31}
\begin{hwproof}
    {
        (i) Write down all the elements of order 7 in $\Z / 28\Z$.

        (ii) How many subgroups are there of order 7 in $\Z / 28\Z$?
    }
    (i) The elements of order 7 in $\Z / 28\Z$ are
    $\{[4], [8], [12], [16], [20], [24]\}$

    (ii) By Proposition 2.7.4 we have that since $7|28$ and the group is cyclic
    then there exists one unique subgroup of order 7.
\end{hwproof}

\subsection*{32}
\begin{hwproof}
    {
        (i) Prove that the cyclic group $\Z / 15\Z$ is isomorphic to the
        product group $\Z / 3\Z \times \Z / 5\Z$.

        (ii) Prove that the group $(\Z / 15\Z)*$ is isomorphic to the product group
        $\Z / 2\Z \times \Z / 4\Z$. Conclude that $(\Z / 15\Z)*$ is not cyclic.
    }
    (i) We see that $\gcd(3,5) = 1$, so they are relatively prime, and $15=3*5$.
    Thus, by Proposition 2.8.2 we have that
    \begin{equation*}
        \tilde{\varphi}: \Z / 15\Z \to \Z / 3\Z \times \Z / 5\Z
    \end{equation*}
    given by $\varphi(x + N\Z) = (\varphi_1(x), \varphi_2(x))$ is a group isomorphism.
    This implies that $\Z / 15\Z \cong \Z / 3\Z \times \Z / 15\Z$.

    (ii)
    We begin by finding the Cayley table for $\Z / 15\Z$.
    \begin{center}
        \begin{tabular}{c | c  c  c  c  c  c  c  c }
            $\times$ & [1]  & [2] & [4]  & [7]  & [8]  & [11] & [13] & [14] \\
            \hline
            $[1]$    & [1]  & [2] & [4]  & [7]  & [8]  & [11] & [13] & [14] \\
            $[2]$    & [4]  & [2] & [8]  & [14] & [1]  & [7]  & [11] & [13] \\
            $[4]$    & [8]  & [2] & [1]  & [13] & [2]  & [14] & [7]  & [11] \\
            $[7]$    & [14] & [2] & [13] & [4]  & [11] & [2]  & [1]  & [8]  \\
            $[8]$    & [1]  & [2] & [2]  & [11] & [4]  & [13] & [14] & [7]  \\
            $[11]$   & [17] & [2] & [14] & [2]  & [13] & [1]  & [8]  & [4]  \\
            $[13]$   & [11] & [2] & [7]  & [1]  & [14] & [8]  & [4]  & [2]  \\
            $[14]$   & [13] & [2] & [11] & [8]  & [7]  & [4]  & [2]  & [1]. \\
        \end{tabular}
    \end{center}

    Now we look at the Cayley table for $\Z / 2\Z \times \Z / 4\Z$.
    \begin{center}
        \begin{tabular}{c | c c c c c c c c}
            +     & (0,0) & (1,0) & (2,0) & (3,0) & (0,1) & (1,1) & (2,1) & (3,1) \\
            \hline
            (0,0) & (0,0) & (1,0) & (2,0) & (3,0) & (0,1) & (1,1) & (2,1) & (3,1) \\
            (1,0) & (1,0) & (2,0) & (3,0) & (0,0) & (1,1) & (2,1) & (3,1) & (0,1) \\
            (2,0) & (2,0) & (3,0) & (0,0) & (1,0) & (2,1) & (3,1) & (0,1) & (1,1) \\
            (3,0) & (3,0) & (0,0) & (1,0) & (2,0) & (3,1) & (0,1) & (1,1) & (2,1) \\
            (0,1) & (0,1) & (1,1) & (2,1) & (3,1) & (0,0) & (1,0) & (2,0) & (3,0) \\
            (1,1) & (1,1) & (2,1) & (3,1) & (0,1) & (1,0) & (2,0) & (3,0) & (0,0) \\
            (2,1) & (2,1) & (3,1) & (0,1) & (1,1) & (2,0) & (3,0) & (0,0) & (1,0) \\
            (3,1) & (3,1) & (0,1) & (1,1) & (2,1) & (3,0) & (0,0) & (1,0) & (2,0) \\
        \end{tabular}
    \end{center}

    Ftom the tables we can tell that these groups are isomorphic given the mapping
    $\varphi(g)$ such that,
    \begin{align*}
        \varphi([1])  & = (0,0), \\
        \varphi([2])  & = (1,0), \\
        \varphi([4])  & = (2,0), \\
        \varphi([7])  & = (3,0), \\
        \varphi([8])  & = (0,1), \\
        \varphi([11]) & = (1,1), \\
        \varphi([13]) & = (2,1), \\
        \varphi([14]) & = (3,1), \\
    \end{align*}
\end{hwproof}

\subsection*{33}
\begin{hwproof}
    {
        Consider $\Z \subset \Q$ as abelian groups with $+$ as composition.
        Let $[q] = q + \Z \in \Q / \Z$, where $q \in \Q$.

        (i) Show that $\left[\frac{9}{4}\right]$ has order 4 in $\Q / \Z$.

        (ii) Determine the order of $\left[\frac{a}{b}\right]$ in $\Q / \Z$,
        where $a \in \Z, b \in \N / \{0\}$ and $\gcd(a,b) = 1$. Conclude that
        every element in $\Q / \Z$ has finite order and that there are elements
        in $\Q / \Z$ of arbitrary large order.

        (iii) Show that $\Q / \Z$ is an infinite group that is not cyclic.
    }
    (i) Doing the computations we can see that
    \begin{equation*}
        \frac{9}{4} + \frac{9}{4} + \frac{9}{4} + \frac{9}{4} = 9.
    \end{equation*}
    Furthermore, we have that $[9] = 9 + \Z = 0 + \Z = [0]$. Since
    $[0]$ is the identity of $\Q / \Z$. Hence, $\frac{9}{4}$ has order 4 in this
    group.

    (ii) Since $\gcd(a,b) = 1$ we get that $b$ is the smallest number such that
    $\frac{a}{b} * b = a$. From part (i)
    we observe that for any integer $a$ we have $[a] = a + \Z = 0 + \Z = [0]$.
    Hence, the order of any $\left[\frac{a}{b}\right]$ is $b$.
    Therefore, since $b$ must be a finite number, the order of any of these elements
    must also be finite.

    (iii) Any element in $\Q / \Z$ can be represented in the form $\frac{a}{b}$
    where $gcd(a,b) = 1$. We have seen from (ii) that any $\left[\frac{a}{b}\right]$
    has finite order $b$. Therefore, any element in $\Q / \Z$ has finite order.
    It is evident that $\Q / \Z$ is infinite, and thus since every element has
    finite order, it cannot be generated by any element. Thus, it is an infinite
    group that is not cyclic.


\end{hwproof}

\subsection*{35}
\begin{hwproof}
    {
        Give an example of a non-cyclic group of order 8.
    }
    The quaternion group $Q_4$ is an example of a non-cyclic group of order
    8.
\end{hwproof}

\subsection*{39}
\begin{hwproof}
    {
        Let $\tau \in S_3$ denote the 3-cycle
        \begin{equation*}
            \begin{pmatrix}
                1 & 2 & 3
            \end{pmatrix}
            = \begin{pmatrix}
                1 & 2 & 3 \\
                2 & 3 & 1
            \end{pmatrix}
            .
        \end{equation*}
        Show that the subgroup $\gen{\tau} = \{\tau^n : n \in \Z\}$ is normal
        in $S_3$.
    }

    We start by asserting that
    \begin{equation*}
        \gen{\tau} = \{e, (1 \ 2 \ 3), (1 \ 3 \ 2)\}.
    \end{equation*}

    We want to show that for any $g \in S_3$. We note that
    \begin{align*}
        (123)^{-1} & = (132), \\
        (12)^{-1}  & = (12),  \\
        (13)^{-1}  & = (13),  \\
        (23)^{-1}  & = (23).
    \end{align*}
    Since $\gen{\tau}$ is a subgroup, it is evident that
    $g\gen{\tau}g = \gen{\tau}$ for $g = e, (123), (132)$. We perform the
    coomputation for the remaining elements of $S_3$.
    \begin{align*}
        (12)\gen{\tau}(12)^{-1} & = \{(12)e(12)^{-1}, (12)(123)(12)^{-1}, (12)(132)(13)^{-1}\}
                                & = \{e, (123), (132)\},                                       \\
        (13)\gen{\tau}(13)^{-1} & = \{(13)e(13)^{-1}, (13)(123)(13)^{-1}, (13)(132)(13)^{-1}\}
                                & = \{e, (123), (132)\},                                       \\
        (23)\gen{\tau}(23)^{-1} & = \{(23)e(23)^{-1}, (23)(123)(23)^{-1}, (23)(132)(23)^{-1}\}
                                & = \{e, (123), (132)\}.                                       \\
    \end{align*}
    Therefore, for every $g \in S_3$ we have that $g\gen{\tau}g^{-1} = \gen{\tau}$
    and thus $\gen{\tau}$ is normal.
\end{hwproof}

\end{document}